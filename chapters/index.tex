索
引
索引中的页码为英文原书页码,与本书中页边标注的页码一致。读者可到图灵社区 (ituring.
com.cn/book/888)下载本索引完整版。
数字
Multimedia SIMID vs. GPUs, (多媒体 SIMID对 GPU),
312
2:1 cache rule of thumb, definition(缓存经验规则,定义),
SMP/DSM shared memory (SMP/DSM 共享存储器),
348
B-29
80x86,见 Intel 80x86处理器
virbual memory(虚拟存储器),B-40~B-41
Address specifier(地址标识符)
instruction set encoding(指令集编码),A-21
A
Address trace, cashe performance(地址跟踪,缓存性能),
B-4
ABI,见 Application binary interface
Address translation(地址变换)
Access bit, LA-32 descriptor table(访问位,IA-32 描述符
AMD64 pagcd virtual memory(AMD64分页虚拟存储器),
表),B-52
B-55~B-56
Access time,另见 Average Memory Access Time
Administrative costs, WSC vs.datacenters(管理成本,WSC
ACID,见 Atomicity-consistency-isolation-durability
对数据中心),455
Active low power modes, WSCe(活联低功率模式,WSC), Advanced directory protocol(高级目标协议)
472
basic function(基本功能),283
Address aliasing prediction(地址别名预测),213
case studies(案例研究),420~426
Address Coalescing Unit(地址接合单元),310
Advanced RisC Machine, 见 ARM(Advanced RISC
Address fault, virtual memory(地址错误,虚拟存储器),
Machine)
B-42
Advanced Technology Attachment disks, 见 ATA (Advanced
Addressing modes(寻址方式),A-11
Technology Attachment)disks
Address offiset, virtual mnemory(地址偏移量,虚拟存储器),
Advanced Vector Extensions (AVX)(高级向量扩展),
B-56
284
Address space(地址空间)
Afine, loop-level parallelism dependences(仿射,循环级
Ferni GPU architecture (Fermi GPU 体系结构),
并行相关性),318~320
306~307
Airflow(气流),466
remory hierarchy(存储器层次结构),B-48~B-49,
Airside econimization, WSC cooling systems(经济化供风,
B-57~B-58
WSC制冷系统),449
544
索
引
Akamai, as Content Delivery Network (Akamai, 作为内容
交付网络),460
Aliased variables, and compiler tecbnology(别名麥量和編
译器技术),A-27~A-28
Aliases, address transiation(别名,地址变換),B-38
Alignment, memory address interpretation(对齐,存储器地
址解读),A-7~A-8,A-8
shared-memory workloads(共享存储器工作负载),369,
370
ALUs,见 Arithmetic-logical units
AMAT,见 Average Merory Access Timne
Arazon,452~455
Amazon Elastic Computer Cloud (EC2)(Amazon 弹性计
算机云),456~457
Amazon Simple Storage Service (S3)(Amazon 简单存储
服务),456~457
Amazon Web Services (AWS)(Amazon Web 服务),
456~459
Amdahl's law (Amdahl定律)
computer design principles(计算机设计原则),46~48
computer system power consumption case study(计算机
系统功耗案例研究),63~64
DRAM,99
parallel computers(并行计算机),406~407
parallel processing calculations(并行处理计算),
349~350
pitfalls(易犯错误),$5-56
processor performance equation(处理器性能公式),
51
scalar performance(标量性能),331
WSC processor cost-performance(WSC 处理器成本性
能),472~473
AMD Barcelona microprocessor (AMD Barcelona 微处理
器),467
AMID Opteron
address translation(地址变换),B-38
Amazon Web Services (Amazot Web 服务),457
architecture(体系结构),15
cache coherence(缓存一致性),361
data cache example(数据缓存示例),B-12~B-15,B-13
Google WSC servers(Google WSC服务器),468~-469
inclusion(包含性),398
manufacturing cost(制造成本),62
misses per instruction(每条指令的缺失数),B-15
MOESI protocol (MOESI 协议),362
multicore processor performance(多核处理器性能),
400-401
multilevel exclusion(多级互斥),B-35
paged virtual memory example(分页虚拟存储器示例),
B-54~B-57
Pentium protection(Pentium保护),B-57
real-world server considerations(现实服务器考忠事
项),52~55
server energy savings(服务器节館),25
snooping limitations(监听局限性),363~364
SPEC benchmarks(SPEC基准测试),43
TLB during address translation(地址变换期间的TLB),
B~47
AMD processors(AMD处理器)
architecture flaws vs. success(体系结构缺陷与成功),
A-45
shared-mnemory multiprogramming workload(共享存储
器多重编程工作负载),378
teroninology(术语),313~315
tournament predictors(竞赛预测器),164
Virtual Machines(虚拟机),110
VMIMs,129
Android OS,324
Antialiasing, address translation(别名消去,地址变换),
B-38
Antidependences(反相关),1$2
Apogee Software(Apogee软件),A-44
AppleiPad
ARM Cortex-A8,114
memory hierarchy basics(存储器层次结构基础),78
Application binary interface (ABI)(应用程序二进制接口)
A-20
Architect-corpiler writer relationship(架构师—编译编
写人员关系),A-29~A-30
Architecturally visible registers(从体系结构角度可见的哥
存翻)
register renaming v5.ROB(寄存器重命名与ROB),
208~209
Architecture,另见 Computer architecture; CUDA (Compute
Unified Device Arcbitecture); Instruetion set architecture
(ISA); Vector architectures
Arithmetic intensity(算术密度)
Roofline model(屋顶轮廓线模型),326,326~327
Arithmetic-logical units (ALUs)(算术逻辑单元)
ARM Cortex-A8,234,236
basic MIPS pipeline(基本 MiPS流水线),C-36
branch condition evaluation(分支条件判断),A-19
data forwarding(数据转发),C-40~C-41
data hazards requiring stalls(需要停顿的数据冒险),
C-19~C-20
data hazard stall minimization(数据冒险停顿最小化),
C-17~C-19
effective address cycle(实际地址周期),C-6
hardware-based execution(基于硬件的扩展),185
hardware-bzsed speculation(基于硬件的推测),
200~201,201
immediate operands(立即操作数),A-12
Intel Core i7,238
ISA operands(ISA 操作数),A-4~A-5
ISA perfotmance and efficiency prediction (ISA 性能与
效率预测),241
load interlocks(载人互锁),C-39
microarchitectural techniques case study(微体系结构技
术案例研究),253
MIIS operations (MIPS操作),A-35,A-37
MIS pipeline control(MIPs 流水线控制),C-38-C-39
MIPS pipeline FP operations (MIPS 流水线浮点运算),
C-$2~C-53
MIPS R4000, C-65
operand forwarding(操作数转发),C-19
operands per instruction exaple(每个指令示例的操作
数),A-6
parallelism(并行度),45
pipeline branch issues(流水线分支发射),C-39~C-41
pipeline executionrate(流水线执行速率),C-10~C-11
power/DLP issues(功率/DLP发射),322
RISC classic pipcline(RISC经典流水线),C-7
RISC instruction set (RISC指令集),C-4
simple MIPS implementation(简单 MIPS实现),
C-31~C-33
ARM (Advanced RISC Machine)(高级 RISC机)
control flow instrictions(控制流指令),14
ISA class(ISA分类),11
memory addressing(存储器寻址),11
operands(操作数),12
Arrays(数组)
access age(访问阶段),91
blocking(分块),89~90
cluster server outage/anomaly statistics(集群服务器输出
1异常编洋效字),435
examples(举例),90
FFT kernel(FFT内核),1-7
Google WSC servers(Google WSC服务器),469
Layer 3 network linkage(L.3 网絡链路),445
1oop interchange(循环交换),88~89
lo0p-level parallelism dependences(循环级并行相关),
318~319
WSC memory bierarchy (WSC存储器层次结构),445
WSCs,443
Auray switch,WSCs,(阵列交换机,WSC),443~444
ASCIl character format (ASCII字符格式),12,A-14
Assembly language(江編语言),2
索
引
545
Associativity,另见 Set associativity
ATA (Advanced Technology Attachment)disks((高级技
术附件)磁盘)
server energy savings(服务器节能),25
ATM systers (ATM 系统)
server benchmarks(服务器基准测试),41
Atomic exchange(原子交换)
lock implementation(锁实现),389-390
synchronization(同步),387-388
Atomic instructions(原子指令)
Core i7, 329
Fermi GPU,308
T1 multithreading unicore performance (T1 多线程单核
性能),229
Atomicity-consistency-isolation-durab ility( ACID),ns.WSC
storage
(原子性--致性-隔离性-持久性与 WSC存储),439
Atomic operations(原子操作)
cache coherence(缓存一致性),360~361
snooping cache coherence implementation(监听缓存一
致性实现),365
Availability(可用性)
computer architecture(计算机体系结构),11,15
data on Internet(互联网上的数据),344
fault detection(错误检测),57~58
loop-level parallelism(循环级并行),217~218
mainstream computing classes(主流计算类别),5
modules(模块),34
open-source software(开源软件),457
RAID systems (RAID 系统),60
as server cbaracteristic(作为服务器特性),7
servers(服务器),16
source operands(源操作数),C-74
WSCs,8,433~435,438~439
Average Memory Access Time (AMAT)(存储器平均访
问时间),B-30-B-31
AVX,见 Advanced Vector Extensions (AVX)
AWS,见 Amazon Web Services (AWS)
B
Backside bus, centalized shared-mnemory tnultiprocessors(背
部总线,集中式共享存储器多处理器),351
Bandwidth,另见Througbput
Banked memory,另见 Memory banks
Banks Fermi GPUs(分组,Fermi GPU),
Barriers(屏障)
commercial workloads(商业工作负载),
hardware primitives(硬件基元),387
546
索
引
synchronization(同步),298,313,329
BARRNet,见 Bay Area Research Network
Base field, IA-32 descriptor table(基址字段,IA-32 描述符
表),B-52~B-53
Basic block, ILP(基本块,IP),149
Batch processing workloads(批处理工作负载)
WSC goals/requirements(WSC 目标/需求),433
WSC MapReduce and Hadoop (WSC MapReduce 和
Hadoop),437~438
Between instruction exceptions(指令间异常),C-45
Big Endian(大端)
memory address interpretation(存储髁地址解释),A-7
MIPS data transfers (MIPS 数据传输),A-34
Bigtable (Google),438, 441
Binary code compatibility(二进制代码兼容性)
VLIW processors (VLIW 处理器),196
Binary-coded decimal, definition(二进制编码十进制数,
定义),A-14
Bing search(必应搜索)
delays and user behavior(延迟与用户行为),451
latency effects(延迟效果),450~452
wsC processor cost-performance(WSC处理器成本性能),
473
Bisection bandwidth, WSC array switch(二分带宽,WSC
阵列交换机),443
Bit selection, block placement(位选择,块放置),B-7
Block addressing(块寻址)
block identification(块识别),B-7~B-8
interleaved cache banks(交错缓存组),86
memory hierarchy basics(存储器层次结构基础),74
Block identification(块识别)
memory hierarchy considerations(存储器层次结构考虑
事项),B-7~B-9
virtual mnemory(虚拟存储器),B-44~B-45
Blocking(分块)
benchmark fallacies(基础测试谬论),56
Blocking calls, shared-memory multiprocessor workload(分
块调用,共享存储器多处理器工作负载),369
Blocking factor, definition(分块因数,定义),90
Block offset(块偏移),B-7~B-8
Block replaceent(块替换)
mnemory hierarchy considerations(存储器层次结构考惠
事项),B-9~B-10
virtual memory(虚拟存储器),B-45
Blocks,另见 Cache block;Thread Block
Block size(块大小)
access time(访问时间),B-28
memory hierarcby basics(存储器层次结构基础),76
miss rate(缺失率),B-27
Body of Vectorized Loop(向量循环体),292,313
Bo8e-Einstein formula, definition(玻色-愛因斯坦公式,定
义),30
Bounds checking, segmented virtual mnemory(边界检查,分
段虚拟存储器),B-52
Branch delay slat(分支延迟时隙)
characteristics(特性),C-23~C-25
control hazards(控制冒险),C-41
MIPS R4000,C-64
scheduling(调度),C-24
Branches(分支)
canceling(取消),C-24~C-25
conditional branches(条件分支),300~303,A-17,
A~19~A-20,A-21
control flow instructions(控制流指令),A-16,A-18
delayed(延迟),C-23
dclay slot(延迟时隙),C-65
MIPS control flow instructions (MIPS 控制流指令),
A-38
MIPS operations(MIPS操作),A-35
mullifying(废除),C-24~C-25
RISC instruction set (RISC指令集),C-5
Branch folding,definition(分支折合,定义),206
Branch hazards(分支冒险)
basic considerations(基本考虑事项),C-2]
penalty reduction(降低代价),C-22~C-25
pipeline issues(流水线发射),C-39,C-42
scheme performance(方案性能),C-25~C-26
stall reduction(减少停顿),C-42
Branch history table, basic scheme(分支历史表,基本方
案),C-27~C-30
Branch oftfsets, control flow instructions(分支偏移 ,控制
流指令),A-18
Branch penalty(分支代价)
examples(示例),205
instruction fetch bandwidth(指令提取带宽),203~206
reduction(降低),C-22~C-25
simple scheme examples(简单方案举例),C-25
Branch prediction(分支预测)
accuracy(精度),C-30
branch cost reduction(降低分支成本),162~167
corelation(相关性),I62~164
cost reduction(降低成本),C-26
dynamic(动态),C-27~C-30
ideal processor(理想处理器),214
TLP exploitation(ILP 开发),201
instruction fetch bandwidth(指令提取带宽),205
integrated instraction fetch uoits(集成指令提取单元).
207
Intel Core i7, 166~167,239~241
mnisprediction rates on SPEC89(SPEC89测得的预测错
误率),166
static(静态),C-26~C-27
two-bit predictor comparison(两位预测比较),165
Branch-prediction buffers, basic considerations(分支预测缓
冲区,基本考忠事项),C-27~C-30,C-29
Branch stalls, MIIPS R4000pipeline(分支停顿,MIPS R4000
流水线),C-67
Branch-target address(分支目标地址)
branch hazards(分支冒险),C-42
MIPS control flow instructions (MIIPS 控制流指令),
A-38
MIrS pipeline(MIPS流水线),C-36,C-37
MIPS R4000,C-25
pipeline branches(流水线分支),C-39
RISC instruction set(RISC 指令集),C-5
Branch-target buffers(分支目标缓冲区)
ARM Cortex-A8, 233
branch hazard stalls(分支冒险停顿),C42
example(示例),203
instruction fetch bandwidth(分支提取带宽),203~206
instruction handling(指令处理),204
MIPS control flow imstructions(MIPS 控制流指令),
A-38
Branch-target cache,见 Branch-target buffers
Bubbles(冒泡)
stall(停顿),C-13
Buffers(缓冲区)
branch-prediction(分支预测),C-27~C-30,C-29
branch-target (分支目标),203~206,204,233,A-38,
C-42
memory(存储器),208
MIPS scoreboarding(MIPS 记分卡),C-74
ROB,184~192,188~189,199,208~210,238
TLB,见 Translation Lookaside buffer
translation buffer(转换缓冲区),B-45~B-46
write buffer(写缓冲区),B-11,B-14,B-32,B-35~B-36
Buses(总线),351
Bypassing,另见Forwarding
Byte offset(字节偏移量)
misaligued addresses(非对齐地址),A-8
PTX instructions(PTX 指令),300
Bytes(字节)
aligned/misaligned addresses(对齐/对齐地址),A-8
arithmnetic intensity example(算术密度示例),286
memory address interpretation(存储器地址解释),
A-7~A-8
MIPS data transfers (MIPS 数据传输),A-34
索
引
547
MIPS data types (MIPS数据类型),A-34
operand types/sizes(操作数类型/大小),A-14
per reference, vs. block size(每次引用对块大小),378
C
CAC,见 Computer aided design(CAD)tools
Cache bandwidth(缓存带宽)
caches(缓存),78
multbanked caches(多组缓存),85~86
nonblocking caches(非阻塞缓存),83~85
pipelined cache access(流水线缓存访问),82
Cache block(缓存块),B-2
Cache coherence(缓存一致性)
advanced directory protocol case study(高级目录协议
案例研究),420~426
basic considerations(基本考虑事项),112~113
directory-based,见 Directory-based cache coherence
enforcement(实施),354~355
extensions(扩展),362~363
hardware primaitives(硬件原语),388
latencry hiding with speculation(以推测隐藏延迟),396
lock implementation(锁实现),389-391
mechanism(方案),358
memory hierarchy basics(存储器层次结构基础),75
multiprocessor-optimized software(多处理器优化软件),
409
multiprocessors(多处理器),352~353
protocol definitions(协议定义),354~355
single-chip multicore processor case study(单芯片多核
处理器案例研究),412-418
single memory location example(单存储器位置示例),
352
snooping,见 Snooping cache coherence
state diagram(状态图),361
steps and bus traffic examples(步长.与总线缓冲区示
例),391
write-back cache(写回缓存),360
Cache definition(缓存定义),B-2
Cache hit(缀存命中),B-2
Cache latency,nonblocking cache 缓存延迟,非陌塞缓存),
83~84
Cache miss(缓存峡失),B-2
Cache optimnizations(缓存优化)
basic categories(基本类别),B-22
basic optimizations(基本优化),B-40
case studies(案例研究),131~133
compiler-controlled prefetching(编译器控制的预取)
92~95
548
索引
compiler optimizations(编译器优化),87~90
critical word first(关键字优化),86~-87
energy consumption(能耗),81
hardware instruction prefetching(硬件指令预取),
91~92,92
hit time reduction(缩短命中时间),B-36~B-40
miss categories(缺失类别),B-23~B-26
miss penalty reduction(降低缺失代价),B-30~B-36
miss rate reduction(降低缺失率),B-26~B-30
multibanked caches(多组缓存),
85~86,86
nonblocking caches(非阻塞缓存),83~85,84
overview(概述),78~79
pipelined cache access(流水线缓存访问),82
simple first-level caches(简单第一級缀存),79~80
techniques overview(技术概述),96
way prediction(路预测),81~82
write buffer merging(写缓冲区合并),87,88
Cache organization(缓存组织方式)
blocks(块),B-7,B-8
Opteron data cache(Opteron 数据级存),B-12~B-13,
B-13
optimization(优化),B-19
performance impact(性能影响),B-19
Cache performance(缓存性能)
average memory access time(存储器平均访问时间),
B-16~B-20
basic considerations(基本考虑事项),B-3~B-6,B-16
basic equations(基本公式),B-22
basic optimizations(基本优化),B-40
cache optimization(绥存优化),96
case study(案例研究),131~133
example calculation(示例计算),B-16~B-17
out-of-order processors(乱序处理器),B-20~B-22
prediction(预测),125~126
Cache prefetch, cache optimization(缓存预测,缓存优化),
92
Caches,另见Memory hierarchy
Cache size(缓存大小)
access time(缓存时间),77
AMD Opteron example (AMID Opteron 示例),
B-13~B-14
energy consumption(能耗),81
highly parallel memory systemns(高度并行存储器系
统),133
memory bierarchy basics(存储器层次结构基础),76
misses per instruction(每条指令的缺失数),126,
371
miss rate(缺失率),B-24~B-25
wsmiss rate(缺失率对比),B-27
miss rate reduction(降低敏失率),B-28
multilevel caches(多级缓存),B-33
relative execution time(相对执行时间),B-34
CACTI
cache optimization(缓存优化),79-80,81
memory access times(存储器访问题意),77
Caller saving, control flow instructions(调用方保存,控制
流指令),A-19~A-20
Call gate(调用门)
IA-32 segment descriptors(1A-32 段描述符),B-53
segmented virtual memory(分段虚拟存储器),B-54
Calls(调用)
compiler structure(编译器结构),A-25~A-26
control flow instructions(控制流指令),A-17,
A-19-A-21
CUDA Thread (CUDA线程),297
dependence analysis(相关性分析),321
high-level instuction set(高级指令集),A-42~A-43
ivocation options(调用选项),A-19
MIPS control flow instructions(MIPS 控制流指令),
A-38
MIPS registers(MIPS 寄存器),12
multiprogrammed workload(多重编程工作负载),378
NVIDIA GPU Memory structures (NVIDIA GPU 存储器
结构),304~305
return address predictors(返回地址预测器),206
shared-memory multiprocessor workload (共享存储器多
处理器工作负载),369
user-to-OS gates(“用户到OS” 门),B-54
Canceling branch, branch delay slots(取消分支,分支延迟
时隙),C-24~-C-25
Canonicai form, AMD64 paged virtual memory(规范形式,
AMD64 分页虚拟存储器),B-55
Capacity misses(容量缺失),B-23
Capital expenditures(CAPEX)(资本性支出)
WSC costs(WSC成本),452-455,453
WSC Flash mmemory(WSC闪存),475
WSC TCO case study( WSCTCO案例研究).476~478
Case statemnents(Case语句)
control flow instruction addressing modes(控制流指令
寻址方式),A-18
return address predictors(返回地址预测器),206
Case studies(案例研究)
advanceddirectoryprotocol(高级目录协议),420~426
cache optimization(缓存优化),131~133
chip fabrication cost(芯片制造成本),61~62
computer system power consumption(计算机系统功
耗),63~64
directory-based coberence(目录式一致性),418-420
highly parallel memory systems(高度并行的存储器系
统),133~136
instruction set principles(指令集原理),4-47-A-54
memory hierarchy(存储器层次结构),B-60~-B-67
microarchitectural techniques(微体系结构技术),
247~254
pipelining exazple(流水线示例),C-82~C-88
single-chip multicore processor( 单芯片多核心处理器),
412~418
vector kernel on vector processor and GPU(向量处理務
和GPU上的向量内核),334~~336
WSC resource allocation(WSC 资源分配),478-479
WSCTCO (WSCTCO),476478
C/C++ language(C/C++语言)
hardware immpact on software development(硬件对软件
开发的影响),4
1oop-level parallelisrn dependences(循环级并行相关
性),318,320~321
NVIDIA GPU programming (NVIDIA GPU 编程),289
return address predictors(返回地址预测器),206
CDC,见 Control Data Corporation
CDF,datacenter(CDF,数据中心),487
Centralized shared-memory multiprocessors(集中式共享存
储器多处理器)
basic considerations(基本考虑事项),351~352
basic structure(基本结构),346~347,347
cache coherence(缓存一致性),352~353
cache coherence enforcement(缓存一致性实施),
354~355
cache coherence example(缓存一致性示例),357~362
cache coherence extensions(缓存一致性扩展),362~363
invalidate protocol implementation(失效协议实现),
356~357
SMP and snooping limitations(SMP 和监听局限性),
363~364
snooping coherence implemnentation(监听一致性实现),
365~366
snlooping coherence protocols(监听一致性协议),
355~356
Central processing unit (CPU)(中央处理器)
Amdahl's law(Amdabl定律),48
average memory access time(存储器平均访问时间),
B-17
cache performance(缓存性能),B-4
coarse-grained multithreading(粗粒度多线程),224
exception stopping/restarting(异常停顿/重新启动),
C-47
extensive pipelining(全面流水线),C-81
Google server usage(Google服务器应用),440
索引
549
GPUs(GPU对比),288
instruction set complications(指令集复杂性),C-50
MIPS implementation (MIPS 实现),C-33~C-34
MIPS precise exceptions( MIPS 精确异常),C-59-C-60
MIPS scoreboarding (MIPS 记分卡),C-77
Pipeline branch issues(流水线分支发射),C41
pipelining exceptions(流水线异常),C-43~C-46
pipelining performance(流水线性能),C-10
SPEC server benchmarks(SPEC服务器基准测试),
40
Central processing unit (CPU)time(中央处理器单元时间)
execution time(执行时间),36
modeling(建模),B-18
processor performance calculations(处理器性能计算),
B-19~B-21
processor performance eqguation(处理器性能公式),
49~51
processor performance time(处理器性能时间),49
Chaining(链接)
VMITPS,268~269
Character(字符)
floating-point performance(浮点性能),A-2
operand type(操作数类型),A-13~A-14
operand types/sizes(操作数类型/大小),12
Chillers(冷凝机)
Google WSC, 466,468
WSC containers(WSC 集装箱),464
WSC cooling systems(WSC制冷系统),448~449
Chime(钟鸣),309
Chipkill
memory dependability(存储器可性),104~105
WSCs, 473
Circulating water system (CWS)(循环水系统(CWS))
cooling system design(制冷系统设计),448
WSCs,448
Clean block, definition(清洁块,定义),B-11
Climate Savers Corputing Initiative, power supply efficiencics
(叶算产业拯救气候计划,电源效率),462
Clock cycles(时钟周期)
basic MIPS pipeline(基本 MIIPS流水线),C-34-C-35
and branch penalties(与分支代价),205
cache performance(缓存性能),B-4
FP pipeline(浮点流水线),C-66
fhull associativity(全相联),B-23
GPU conditional branching(GPU条件分支),303
ILP exploitation (ILP开发),197,200
ILP exposure(揭示 ILP),157
imstrnuction fetch bandwidth(指令提取带宽),202~203
instruction steps(指令步骤),173~175
550
索引
Iatel Core i7 bramch predictor (Intel Core i7 分支预测
器),166
MIPS exceptions(MIPS 异常),C-48
MIPS pijpeline(MIPS 流水线),C-52
MIPS pipeline FP operations(MIPS 流水线浮点运算),
C-52~C-53
MIPS scoreboarding(MIPS记分卡),C-77
miss rate calculations(缺失率计算),B-31~B-32
multithreading approaches(多线程方法),225~226
pipelining performance(流水线性能),C-10
processor performance equation(处理器性能公式),
49
RISC classic pipeline(RISC 经典流水线),C-7
Sun T1 multithreading(Sun T1 多线程),226-227
vector execution time(向量执行时间),269
vector multiple lanes(向量多车道),271~273
VLIW processors(VLIW处理器),195
Clock cycles per instruction (CPI)(每条指令的时钟周期数)
addressing modes(寻址方式),A-10
branch schemes(分支方案),C-25~C-26,C-26
cache behavior impact(缓存行为影响),B-18~B-19
cache bit calculation(缓存命中计算),B-5
data hazards reguiring stalls(需要停顿的数据冒险),
C-20
extensive pipelining(全面流水线),C-81
floating-point calculations(浮点计算),50~52
ILP concepts (ILP 概念),148~149,149
ILP exploitation(ILP开发),192
MIPS R4000 perfornance(MIPS R4000性能),C-69
miss penalty reduction(降低缺失代价),B-32
multiprocessing/multithreading-based performance(基于
多重处理/多线程的性能),398-400
multiprocessor commnunication calculations(多处理器通
信计算),350
;ipeline branch issues(流水线分支发射),C41
pipeline with talls(具有停顿的流水线),C-12~C-13
pipclne structural hezards(流水线结构性冒险),
(15~C-16
pipcining concept(流水线概念),C-3
proc;essor perfornance calculations(处理器性能计算),
218~212
proce:sor perfornance time (处理器性能时间),49~51
processor speed(处理器速度),244
sharci-memory workloads(共享存储器工作负载),
369~370
simple MIPS implementation(简单 MIPS实现),
C-33~C-34
structural hazarf(结构性冒险),C-13
Sun Tl multith.:ing unicore performance (Sun T! 多
线程单核心性能),229
Sun Tl processor(Sun T1处理器),399
Tomasulo's algorithm(Tomasulo 算法),181
Clock cycle time(时钟周期时间)
assoCiativity(相联度),B-29
average memnory access time(存储器平均访问时间),
B-21~B-22
cache optimization(缓存优化),B-19~B-20,B-30
cache perforance(缓存性能),B-4
CPU time equation(CPU 时间公式),49-50,B-18
MIPS implementation(MIPS实现),C-34
miss penalties(敏失代价),219
pipeline performance(流水线性能),C-12,C-14~C-15
pipelining(流水化),C-3
Clock periods, processor performance equation(时钟周期,
处理器性能公式),48~49
Clock rate(时钟速率)
DDR DRAMS and DIMIMS (DDR DRAMS 和 DIMMS),
101
ILP for realizable processors(可实现处理器的ILP),218
microprocessors(微处理器),24
MIPS pipeline FP operations(MIPS流水线深点),C-$3
multicore processor performance(多核处理器性能),
400
processor speed(处理器速度),244
Clocks, processor performance equation (时钟,处理器性能
公式),48-49
Clock skew, pipelining performance(时钟偏移,流水线性
能),C-10
Clock ticks(时钟嘀嗒)
cache coherence(缓存一致性),391
processor performance equation(处理器性能公式),
48~49
Cloud computing(云计算)
basic considerations(基本考虑事项),455-461
clusters(集群),345
provider issues(提供商同题),471~472
Clusters(集群)
cloud computing(云计算),345
computer class(计算机类别),5
Google WSC servers(Google WSC服务器),469
outage/anomaly statistics(储运/异常编译数字),435
WSC forerunners(WSC先驱),435~436
WSC storage(WSC 存储),442~443
CMOS
DRAM,99
Coarse-grained multithreading, definition(粗粒度多线程,
定义),224~226
Code generation(代码生成)
compiler structure(编译器结构),A-25~A-26,A-30
dependences(相联性),220
general-purpose register computers(通用寄存器计算
机),A-6
ILP Limitation studies(IP局限性研究),220
loop unrolling/scheduling(循环展开/调度),162
Code size(代码大小)
architect-compiler considerations(体系结构编泽器考虑
事项),A-30
benchmark information(基准测试信息),A-2
comparisons(比较),A-44
flawless architecture design(无瑕疵体系结构设计),
A-45
instruciton set encoding(指令集编码),A-22~A-23
ISA and compiler technology (ISA 和编译器技术),
A-43~A-44
loop unrolling(循环展开),160~161
multiprogramming(多重编程),375~376
PMDS,6
RISCs, A-23~A-24
VAX design (VAX设计),A-45
VLIW model(VLIW模型),195~196
Coerced exceptions(强制异常),C-45
Coherence,见 Cache coherence
Coherence misses(一致性缺失),366
Cold-start misses, definition(冷启动映失,定义),B-23
Collision misses,definition(冲突映失,定义),B-23
Column access strobe (CAS),DRAM(列访同选通,
DRAM),98~99
Column major order(列主序)
blocking(分块),89
stride(步幅),278
COMA,见 Cache-only memory architecture
Commercial workloads(商业工作负载)
execution time distribution(执行时间分布),369
symmetric shared-memory multiprocessors(对称共享存
储器多处理器),367~374
Comnit stage, ROB instruction( 提交阶段,ROB 指令),
186~187,188
Commodities(大众化商品)
Amazon Web Services(Amazon Web服务),456~457
array switch(阵列交换机),443
cloud computing(云计算),455
cost vs. price(成本对价格),32~33
cost trends(成本趋势),27~28,32
Ethernet rack switch(以太网机架交换机),442
HPC hardwvare(HPC硬件),436
shared-memory multiprocessod(共享存储器多处理器),
441
索
引
5$1
Common data bus (CDB)(通用数据总线(CDB))
dynamnic scheduling with Tomasulo's algorithm(用
Tomasulo 算法实现动态调度),172,175
FP unit with Tomasulo's algorithm(采用Tomasulo算.
法的浮点单元),185
reservation stations/register tags(保留站点/寄存器标
签),177
Tomasulo's algorithm(Tomasulo 算法),180,182
Communication mechanist(通信机制)
multiprocessor communication calculations(多处理器
通信计算),350
SMP limitations (SMP 限制条件),363
Communication subnets,见 Interconnection networks
Communication subsystems,见 Interconnection networks
Compiler-controlled prefetching, miss penalty/tate reduction
(编译器控制的预取,降低缺失代价/缺失率),92~95
Compiler optimizations(编译器优化)
blocking(分块),89~90
cache optimization(缓存优化),131~133
compiler assumptions(编译器假设),A-25~A-26
consistency model(一致性模型),396
loop interchange(循环交换),88-89
miss rate reduction(降低缺失率),87~90
passes(扫描遍数),A-25
performance impact(性能影响),A-27
types and classes(类型与分类),A-28
Compiler scheduling(编译器调度),C-71
Compiler technology(編译器技术)
architecture decisions(体系结构决策),A-27~A-29
ISA and code size (ISA 和代码大小),A-43~A-44
multimedia instriction support(多媒体指令支持),
A-31~A-32
register allocation(寄存器分配),A-26~A-27
structure(结构),A-24~-A-26, A-25
Compiler writer-architect relationship(编译器编写人员-架
构师关系),A-29~A-30
Compulsory misses(强制缺失),B-23
Computer aided design(CAD)tools, cache optimization(计
算机辅助设计工具,缀存优化),79-80
Computer architecture,另见 Architecture
Computer chipfabrication(计算机芯片制造)
Cost case study(成本案例研究),61~62
Computer classes(计算机分类),5
Computer design principles(计算机设计原理)
Amdahl's law(Amdahl定律),46~48
common case(常见情景),45~46
parallelism(并行度),44~45
principle oflocality(局域性原理),45
$52
索
引
processor performance equation(处理器性能公式),
48~52
Computer history,technology and architecture(计算机历史,
技术与体系结构),2~5
Computer room air-conditioning (CRAC),WSC infrastructure
(计算机室空调,WSC 基础设施),448-449
Compute Unified Device Architecture, 见 CUDA (Compute
Unified Device Architecture)
Conditional branches(条件分支)
branch folding(分支折合),206
compate frequcncies(比较频率),A-20
compiler performance(编译器性能),C-24~C-25
control flow instructions(控制流指令),14,A-16,
A-17,A-19,A-21
evaluation(求值),A-19
GPUs,300~303
ideal processor(理想处理器),214
ISAs,A-46
MIPS control flow instructions (MIPS 控制流指令),
A-38, A-40
predictor misprediction rates(预测器错误预测率),
166
PTX instruction set(PTX 指令集),298~299
static branch prediction(静态分支预测),C-26
types(类型),A-20
vector-GPU comparison(向量-GPU比较),311
Condition codes(条件码),C-5
Conflict misses(冲突缺失),B-23
Connection Multiprocessor(连接多处理器),2
Consistency,见 Memory consistency
Containers(集装箱)
airflow(气流),466
Google WSCs, 464~465,465
ontext Switching(上下文换),106,B-49
definition(定义),106,B-49
Lontrol Data Corporation (CDC) 6600(控制数据公司6600)
dynamically scheduling with scoreboard(用记分卡动态
调度),C-71~C-72
MIps scoreboarding (MIPS记分卡),C-75~C-77
Control Data Corporation (CDC)STAR-100(控制数据公
司 STAR-100)
peak performance vs. start-up overhead(蜂值性能对启
动开销),331
Control dependences(控制相关性)
data dependence(数据相关性),150
hardware-based speculation(基于硬件的推测),163
ILP, 154~156
ILP hardware mnodel(ILP硬件模型),214
Tomasulo's algonithm(Tomasulo算法),170
vector mask registers(向量遮罩寄存驟),275~276
Control flow instructions(控制流指令)
addressing modes(寻址方式),A-17~A-18
basic considerations(基本考虑事项),A-16~A-17,
A-20~A-21
ciasses(分类),A-17
conditional branch options(条件分支选项),A-19
hardiwvare vs. softwvare speculation(硬件对软件推测),
221
procedure invocation options(过程调用选项),
A-19~A-20
Control hazardis(控制冒险),C-11
Control instructions(控制指令)
VAX,B-73
Controller transitions(控制器转換)
directory-based(基于目录),422
snooping cache(监听缓存),421
Control Processor(控制处理器),309
Conventional datacenters, vs. WSCs(传统数据中心对
WSC),436
Cooling systems(制冷系統)
Google WSC, 465~468
mechanical design(机械设计),448
WSC infrastructure(WSC 基础设施),448~449
Core definition(核心定义),15
Correlating branch predictors, branch costs(相关分支預测
器,分支成本),162~163
Cost(成本)
Amazon EC2, 458
Amazon Web Services(Amazon Web 服务),457
branch predictors(分支预测器),162-167,C-26
chip fabrication case study(芯片制造案例研究),61~-62
cloud computing providers(云计算提供商),471~472
MapReduce calculations(MapReduce 计算),458~459,
459
memory hierarchy design(存储器层次结构设计),
72
multiprocessor cost relationship(多处理器成本关系),
409
mmultiprocessor linear speedup(多处理器线程加速),
407
server calculations(服务器计算),454,454~455
server usage(服务器使用),7
speculation(推测),210
tournament predictors(竞赛预测器),164~166
WSC auray switch(WSC阵列交換机),443
WSC vs.datacenters(WSC对数据中心),455~456
WSC efficiency(WSC效率),450-452
WSC facilities(WSC设施),472
WSC network bottleneck(WSC 网络瓶颈),461
WsCs vs.servers(WSC对服务器),434
WSC TCO case study(WSC TCO案例研究),476~478
Cost associativity, cloud conputing(成本柑关佳,云计算),
460~461
Cost-perfornance(成本一性館)
computer trends(计算机趋势),3
extensive pipelining(全面流水化),C-80~C-81
IBM eServer p5 processor(IBM eSever pS 处理器),
409
WSC Flash mnemory(WSC闪存),474~475
WSC goals/requirements(WSC目标/需求),433
WSC hardwace inactivity (WSC 硬件不活联),474
WSC processors(WSC处理器),472~473
Cost trends(成本趋势)
integrated circuits(集成电路),28~32
manufacturing vs. operation(制造对运行),33
overview(概述),27
price(价格),32~33
time, volume,commoditization (时间、体积、大众商品
化),27~28
CPI,见 Clock cycles per instruction
CPU,见 Central processing unit
CRAC, 见 Computer room air-conditioning
Cray-1
peak performance vs. start-up overhead(峰值效率对启动
开销),331
vector performance(向量性能),332
as VMIPS basis(作为 VMIPS 基础),264,270-271,
276~277
Cray,C-90
Cray T90, memory bank calculations (Cray T90,存储器组
计算),276
Cray Y-MP
vector architecture programming(向量体系结构编程),
281,281~282
Critical word first(关键字优先),86~87
Cary-skip adder (CSA)(进位跳跃加法器)
C# language, hardware impact on software development(C#
语言,硬件对软件开发的影响),4
CUDA(Compute Unified Device Architecture)(计算统一
设备体系结构)
GPU conditional branching(GPU条件分支),303
GPUs vs.vector architectures(GPU对向量体系结构),
310
NVIDIA GPU programming (NVIDIA GPU编程),
289
PTX,298, 300
sample program(示例程序),289~290
索
553
SIMD instructions (SIID 指令),297
terminology(术语),313~315
CUDA Thread(CUDA线),292,313
CYBER 180/990, precise exceptions (CYBER180/990,精
确异常),C-$9
Cycles, processor performance equation(周期,处理器性能
讨算),49
D
DaCapo benchmarks (DaCapo 基准测试)
ISA,242
SMT,230~231,231
DAMQS, 见 Dynamically allocatabie multi-queues
Database program speculation, via multiple branches(数据
库程序推测,通过多分支),211
Data cache(数据缓存)
ARM Cortex-A8,236
cache optimization(缓存优化),B-33,B-38
cache performance(缓存性能),B-16
GPU Memory(GPU存储器),306
ISA,241
locelity principle(局域性原理),B-60
MIIPS R4000 pipeline( MIPS R4000流水线),C-62~C-63
multiprogramning(老重编程),374
page level write-through(页级直写),B-56
RISC processor(RISC处理器),C-7
structural hazards(结构性冒险),C-IS
TLB,B-46
Data cache miss(数据缓存缺失)
epplications vs.OS(应用程序对操作系統),B-59
cache optimization(缓存优化),B-25
Intel Core i7,240
Opteron,B-12~B-15
sizes and associativities(大小与相联性),B-10
wites(写人),B-10
Data cache size, multiprogramming(数据缓存大小,多重
编程),376~377
Datacenters(数据中心)
CDF,487
cooling systems(制冷系统),449
layer 3 network example(L.3 网络示例),445
PUE statistics (PUE 统计数字),451
tier classitications(层分类),491
WSC costs(WSC成本),455~456
WSC eficiency measurement(WSC效率测量),
450~452
Data dependences(数据相关性)
data hazards(数据胃险),167~168
554
索
引
dynarnically scheduling with scoreboard(以记分卡进行
动态调度),C-71
hazards(冒险),153~154
ILP,150~152
IP harcdware model(ILP硬件模型),214-21S
ILP limitation studies (正P局限性研究),220
vector execution time(向量执行时间),269
Data fetching(数据提取)
ARM Cortex-A8,234
directory-based cache coberence protocol example(目录
式缓存一致性协议示例),382~383
dynamically scheduled pipelizes (动态调度流水线),
C-70~C-71
ILP, instruction bandwidth (ILP,擔令带宽),202~207
MIPS R4000,C-63
snooping coherence protocols(监听一致性协议),
355~356
Data flow(数据流)
control dependence(控制相关性),154~156
dynamic scheduling(动态调度),168
ILP limitation studies (ILP 局限性研究),220
Data flow execution, hardware-based speculation(数据流执
行,基于硬件的推测),184
Data hazards(数据冒险)
stall minimization by forwarding(通过转发将停顿降至
最低),C-16~C-19,C-18
stall requirements(停顿需求),C-19~C-21
VMIPS, 264~267
Data-level parallelism (DLP)(数据级并行),9
Data-race-free, synchronized programs(无数据竞赛,同步
程序),394
Data races, synchronized programs(数据竞赛,同步程序),
394
Data transfers(数据传送)
cache miss Tate calculations(缓存续失率计算),B-16
computer architecture(计算机体系结构),15
gather-scater(集中/分散),281,291
instruction operators(指令运算符),A-15
MIS, addressing modes (MIIPS,寻址模式),A-34
MIPS64 instruction subset (MIPS64 指令子集),A-40
MIPS64 ISA formats (MIPS64 ISA格式),14
MIPS operations(MIPS 操作),A-36~A-37
MIMX,283
multimedia instruction compiler support(多媒体指令编
译器支持),A-31
operands(操作数),A-12
PTX,305
SIMD extensions(SIMD 扩展),284
“typical programs(“典型”程序),A-43
VAX,B-79
vector vs.GPU(向量对CPU),300
Data trunks, MIPS scoreboarding 数据干线,MIIS记分卡),
C-75
Data types(数据类型)
architect-compiler Writer relationship(架构师-编泽器编
写人员关系),A-30
desktop computing(桌面计算),A-2
MIPS,A-34,A-36
MIPS64 atrchitecture(MIPS64 体系结构),A-34
multimedia compiler support(多媒体编译器支持),
A-31
operand types/sizes(操作数类型/大小),A-14~A-15
SIMD Multimedia Bxtensions(SIMD 多媒体扩展),
282~283
DAXPY loop(DAXPY循环)
memory bandwidth(存储器带宽),332
MIPS/VMIPS calculations(MIPS/VMIPS 计算),
267~268
peak performance vs. start-up overhead(峰值性能对启
动开销),331
VLRs,274~275
D-caches (D缓存)
case study examples(案例研究示例),B-63
way prediction(略预测),81~82
DDR, 见 Double data rate
Deadlock(死锁)
cache coherence(缓存一致性),361
directory protocols(目录协议),386
synchronization(同步),388
Decimal operands, formats(十进制操作数,格式),A-14
Decision support system( DSS),
shared-memory
workloads(决策支持系统,共享工作负载),368~369,
369,369~370
DEC PDP-11, address space (DEC PDP-11,地址空间),
B-57~B-58
Defect tolerance, chip fabrication cost case study (缺陷容忍,
芯片制造成本案例研究),61~62
Delayed branch(延退分支)
basic scheme(基本方案),C-23
stalls(停顿),C-65
Dell Poweredge servers, prices (Dell Poweredge 服务器,价
格),53
Dell servers (Dell 服务器)
economies of scale(规模经常),456
real-world considerations(现实考虑事项),$2-55
WSC services(WSC服务),441
Dependences(相关),152~153,315~316
Descriptor priviege level ( DPL), segmented virtbral mernory(描
述符权限级别,分段虚拟存储器),B-$3
Descriptor table, IA-32(描述符表,IA-32),B-$2
Desktop computers(台式计算机)
characteristics(特性),6
compiler structure(编译器结构),A-24
as computer class(作为计算机类别),5
memory hierarcby basics(存储器层次结构基础),78
multiprocessor importance(多处理器重要性),344
performance benchmarks(性能基准测试),38-40
processor comparison(处理器比较),242
RAID bistory (RAID历史),C-80
Destination offset, IA-32 segment(目标偏移量,IA-32段),
B-53
Dies(晶片)
integrated circuits(集成电路),28~30,29
Nehaler floorplan(Nehalem 平面图),30
wafer example(晶圆示例),31,31~32
Die yield(晶片正品率),basic eguation(基本公式),
30~31
Digital Alpha
branches(分支),A-18
Digital Alpha 21264
cache bierarchy(缓存层次结构),368
floorplan(平面图),143
Digital Alpha processors (Digital Alpha如理)
control flow instruction branches(控制流指令分支),
A-18
displacement addressing mode(位移量寻址方式),4-12
exception stopping/restarting(异常停止/重新启动),
C-47
immediate value distribution(立即值分布),A-13
MIPS precise exceptions (MIPS 精确异常),C-59
shared-memory workload(共享存储器工作负载),
367~369
DIMMs,见 Dual inline memory modules
Direct-mnapped cache(直接映射缓存)
address parts(地址部分),B-9
address translation(地址转换),B-38
block placement(块布置),B-7
memory hierarchy basics(存储器层次结构基础),74
memory hierarchy(存储器层次结构),B-48
optimization(优化),79~80
Directory-based cache coherence(目录式缓存一致性),
354
Dirty bit(脏位(重写位)),B-11
Dirty block(脏块),B-11
Disk technology(磁盘技术)
failure rate calculation(故障率计算),48
Google WSC servers (Google WSC 服务器),469
索
引
555
performance ttends(性能趋勢),19~20,20
WSC Flash memory(WSC闪存),474~475
Dispatch stage(分派级)
instruction steps(指令步骤),174
mnicroarchitectural techniques case shudy(微体系结构技
术案例研究),247-254
Displacement addressing mode(位移量寻址方式)
basic considerations(基本考虑事项),A-10
MPS,12
MIPS data transfers(MIPS 数据传输),A-34
MIPS instruction format (MIPS 指令格式),A-35
value distributions(值分布),A-12
Distributed sbared memory(DSM)(分布式共享存储器)
basic considerations(基本考忠事项),378~380
basic structure(基本结构),347~348,348
directory-based cache coherence(目录式缓存一致性),
354, 380,418~420
mmuitichip multicore mnultiprocessor(多芯片多核多处理
器),419
snooping coherence protocols(监听一致性协议),355
DLP,见 Data-level parallelism
Double data rate (DDR)(双数据率)
ARM Cortex-A8,117
DRAM performance (DRAM性能),100
DRAMs and DIMIMS (DRAM 和 DIMMS),101
Google WSC servers(Google WSC服务),468~469
Intel Core i7, 121
SDRAMs,101
Double data rate 2 (DDR2),SDRAM tirning diagram(双
数据率2,SDRAM时序图),139
Double data rate 3 (DDR3)(双数据率3)
DRAM internal organization(DRAM 内部组织方式),
98
GDRAM,102
Iatel Core i7,118
$DRAM power consutijption(SDRAM功耗),102,
103
Double data rate 4(DDR4), DRAM(双数据率 4, DRAM),
99
Double data rate s (DDRs),GDRAM(双数据率5,
GDRAM),102
Double-precision floating point(双精度浮点)
add-divide(加除),C-68
AVX for ×86(x86的AVX),284
data access beachmarks(数据访问基准测试),A-15
Fermi GPU architecture(FermiGPU体系结构),306
foating-point pipeline(浮点流水线),C-65
GTX 280,325,328-330
IBM360,171
556
索引
MIIPS,285,A-38~A-39
MIIPS data transfers (MIPS 教据传输),A-34
MIPS registers (MIIS 寄存器),12,A-34
Moltimedia SIMD vs.GPUs(多媒体SIMID对GPU),
312
operand sizes/types(操作数大小/类型),12
as operand type(作为操作数类型),A-13~A-14
operand usage(操作数便用),297
pipeline timing(流水线时序),C-54
Roofline model(屋顶轮廓线模型),287,$26
SIMD Extensions(SMID扩展),283
VMIPS, 266, 266-267
Double words(双字)
aligned/misaligned addresses(对齐/未对齐地址),A-8
data access benchmarks(数据访问基准测试),A-15
memory address interpretation(存储器地址解释),
A-7~A-8
MuPS data types(MIPS数据类型),A-34
operand types/sizes(操作数类型/大小),12,A-14
stride(步幅),278
DRAM,见 Dynamic random-access memory
Driver domains, Xen VM(驱动程序域,Xen VM),111
DSM,见 Distributed shared memory
DSS,见 Decision support system
Dual inline memory modules (DIMMs)(双列直插存储器
模块)
clock rates, bandwidth, names(时钟频率,带宽,名称),
101
DRAM basics(DRAM基础),99
Google WSC server(Google WSC服务器),467
Google WSCservers(Google WSC 服务器),468~469
eraphics memory(图形存储器),322~323
Intel Core i7, 118,121
SDRAMs,101
WSC memory(WSC存储器),473-474
Dual SIMID Thread Scheduler, example(双 SIMD线程调度
程序,示例),305~306
DVFS,见 Dynamic voltage-frequency scaling
Dynamically scheduled pipelines(动态调度流水线),C-56
Dynamically shared libraries, control flow instruction
addressing modes(动态共享库,控制流指令寻址方式),
A-18
Dynamic energy,definition(动态能量,定义),23
Dynamic power(动态功率)
energy efficiency(能量效率),211
mnicroprocessors(微处理器),23
static power(静态功率),26
Dynamic random-access memory (DRAM)(动态随机访
问存储器 DRAM)
bandwridth issues(带宽问题),322~323
characteristics(特性),98~100
clock rates, bandwridth,rames( 时钟频率,带宽,名称),
101
cost vs. access time(成本对访问时间),D-3
cost trends(成本趋势),27
CUDA,290
dependability(可靠性),104
Flash memory(闪存),103~104
Google WSC servers(Google WSC服务器),468-469
GPU SIMD instructions(GPU SIMD指令),296
irprovement over time(随时间的改进),17
intcgrated circuit costs(集成电路成本),28
Intel Core i7,121
interal organization(内部组织),98
memory hierarchy design(存储器层次结构设计),73,
73
memnory performance(存储器性能),100~102
multibanked caches(多组缓存),86
NVIDIA GPU Metnory structures (NVIDIA GPU 存储
器结构),305
performance milestones(性能里程碑),20
power consumption(功耗),63
real-world server considerations(现实服务器考忠事
项),$2~55
Roofline model(屋顶轮廓线模型),286
server energy savings(服务器节能),25
speed trends(速度趋势),99
technology trends(技术趋势),17
WSC efficiency measurement(WSC效率測量),450
WSC memory costs (WSC 存储器成本),473~474
WSC memory hierarchy (WSC 存储器层次结构),
444-445
WSC power modes(WSC 功率模式),472
yield(正品率),32
Dynamie scheduling(动态调度)
ILP,168
MIPS scoreboarding (MIPS记分卡),C-79
SMT on superscalar processors(超标量处理器上的
SMT),230
unoptirized code(未优化代码),C-81
Dynamic voltage-frequency scaling (DVFS)(动态电压-
频率调整)
energy efficiency(能耗效率),25
Google WSC,467
processor pertfommance eqyation(处理器性能公式),
52
Dynamo(Amazon),438,452
E
Early restart, mniss penalty reduction(早期重启,降低敏失
代价),86
EBS,见 Elastic Block Storage
EC2,见 Amazon Elastic Computer Cloud
ECC,见 Erot-Corecting Code
Economies of scale(规模经济),434-456
EEMBC,见 Electronic Desigon News Embedded Microprocessor
Benchmark Consortium
EEPROM(Electonically Erasable Programmable Read-Only
Memory,(电子可擦除可编程只读存储器)
compiler-code size considerations(编译器代码大小考
虑事项),A-44
Effective address(实际地址),A-9
Eight-way set associativity(八路組相联)
ARM Cortex-A8, 114
cache optimization(缓存优化),B-29
conflict misses(冲褒缺失),B-23
data cache misses(数据缓存续失),B-10
Elapsed time, execution time(消逝时间,执行时间),36
Elastic Block Storage(EBS),MapReduce cost calculations
(弹性分块存储,MapReduce 成本计算),458~460,459
Electronically Erasable Programmable Read-Only Memory,
见 EEPROM(Electronically Erasable Programmable
Read-Only Memory)
Electronic Desigo News Embedded Microprocessor Benchrark
Consortium(EEMBC)(电子设计新闻嵌人式处理器
基准测试协会)
ISA code size(ISA代码大小),A-44
performance benchmatks(性能基准测试),38
Element group, definition(元素组,定义),272
Encoding(编码)
control flow instructions(控制流指令),A-18
erasure encoding(删除编码),439
instruction set(指令集),A-21~A-24,A-22
ISAs,14,A-5~A-6
MIPS ISA,A-33
MIPS pipeline (MIPS 流水线),C-36
opcode(操作码),A-13
VLIW mmodel(VLIW模型),195~196
Energy efficiency,另见 Powerconsumption
Energy proportionality, WSC servers(能量比例,WSC.服
务器),462
EPIC approach(EPIC方法)
VLIW processors(VLIW处理器),194,196
ERA,见 Engineering Research Associates
Erasure encoding, WSCs(消除编码,WSC),439
.
索
引
557
Error-Corecting Code (ECC)(纠错码)
tault detection pitfalls(错误检测易犯错误),58
Fermi GPU architecture (Fermi GPU 体系结构),307
memory dspendability(存储器可靠性),104
WSCs,473~474
Ethernet switches(以太网交換机)
architecture considerations(体系结构考虑因素),16
Dell servers(Deil 服务器),53
Google WSC, 464~465,469
historical performance milestones(历史性能里程磚),
20
EX,见 Execution address cycle
Exceptions(异常)
ALU instructions (ALLU指令),C-4
architecture-specific examples(体系结构特有示例),
C-44
categories(类别),C-46
control dependence(控制相关性),154~155
hardware-based speculation(基于硬件的推测),190
imprecise(非精确),169~170,188
long latency pipelines(长延迟流水线),C-$5
MIPS,C-48,C-48~C-49
out-of-order completion(乱序完成),169~170
precise(精确),C-47, C-58~C-60
return address buffer(返回地址缓冲区),207
ROB instructions (ROB 指令),190
speculative execution(推测执行),222
stopping/restarting(停止/重新启动),C-46~C47
types and requirements(类型与需求),C-43~C-46
Execute step(执行步骤)
instruction steps(指令步骤),174
ROB instruction (ROB 指令),186
Execution address cycle (EX)(执行地址周期)
basic MIPS pipeline(基本MIPS流水线),C-36
data hazards requiring stalls(需要停顿的数据冒险),
C-21
data hazard stall minimization(数据冒险停顿最小化),
C-17
exception stopping/restarting(异常住/重新启动),
C-46~C-47
hazards and forwarding(冒险与转发),C-56~C-57
MIPS FP operations, basic considerations (MIPS 浮点运
算,基本考虑事项),C-51~C-$3
MIS pipeline (MIPS流水线),C-52
MIPS pipeline control (MIPS 流水线控制),C-36-C-39
MIPS R4000, C-63~C-64, C-64
MIPS scoreboarding(MIPS记分卡),C-72,C-74,
C-77
out-of-order execution(乱序执行),C-71
558
索引
pipeline branch issues(流水线分支发射),C-40,C-42
RISC classic pipeline (RISC经典流水线),C-10
simple MIPs implementation(简单MIPS 实现),
C-31~C-32
simple RISC implementation(简单RISC实现),C-6
Execution time(执行时间)
Amdahl's law(Amdahi 定律),46~47,406
application/OS misses(应用程序/OS缺失),B-59
cache perfornance(缀存性能),B-3~B-4,B8-16
calculation(计算),36
commercial workJoads(商业工作负载),369~370,370
energy efficiency(能量效率),211
integrated circuits(集成电路),22
loop unrolling(循环展开),160
rmultilevel caches(多级缓存),B-32~B-34
multiprocessor perfomnance(多处理器性能),405~406
mltiprogrammed parallel "make” workload(多重编程
的并行“make” 工作负载),375
multitbreading(多线程),232
performnance equations(性能公式),B-22
pipelining performance(流水化性能),C-3,C-10~-C-I1
PMD8,6
principle oflocality(局域性原理),45
processor comparisons(处理器对比),243
processor performance equation(处理器性能公式),
49, 51
reduction(缩减),B-19
second-level cache size(第二级缓存大小),B-34
SPEC benchmarks (SPEC基准测试),42~44,43,56
stall time(停顿时间),B-21
vector mask registers(向量遮單寄存器),276
vector operations(向量运算),268~271
Expand-down field(向下扩展字段),B-53
Explicit operands, ISA classifications(显式操作数,ISA 分
类),A-3~A-4
Explicit unit stride, GPUs vs. vector architectures(显式单位
步橱,GPU对向量体系结构),310
Extended accummulator(扩展累加器)
flaved architectures(有缺陷体系结构),A-44
ISA classification (ISA分类),A-3
F
Facebook, 460
Failures,另见 Mean time between failures (MTBF);Mean
time to failure(MTTTF)
False sharing(伪共享),366~367
FarmVille,460
Fault detection, pitfalls(故障检测,易犯错误),57~58
Faulting prefetches, cache optimization(故障预取,缓存优
化),92
Feature size(特征尺寸)
dependability(可靠性),33
FEC,见 Forward eror correction
Fermi GPU
architectural innovations(体系结构创新),305-308
future features(未来特征),333
Grid mapping(网格映射),293
multithreaded SIMID Processor(多线程 SIMD处理器),
307
NVIDIA,291,305
SIMID,296~297
SIMD Thread Scheduler(SEMID 线程调度程序),306
Fermi Tesla GTX 280
GPU comparison(GPU比较),324~325,325
meinory bandwidth(存储器带宽),328
rav/relative GPU performance(原始/相对GPU性能),
328
synchronization(同步),329
weaknesses(弱点),330
Fermi Tesla GTX 480
foorplan(平面图),295
GPU comparisons(GPU 比较),323~330,325
Fetch-and-increment(提取并递增)
synchronization(同步),388
Fetching,见 Data fetching
Field-programmable gate arrays ( FPGAs), WSC array
switch),现场可编程阵列,WSC阵列交換机),433
Fine-grained multithreading(细粒度多线程),224~226
First-level caches,另见 L.1 caches
technology trends(技术趋势),18
Ftrst-referenoe mnisses, de finition(首次引用缺失,定义),B-23
FIT rates, WSC memory(FIT率,WSC 存储器),473-474
Fixed-field decoding, simple RISC implementation(固定字
段解码,简单 RISC实现),C-6
Fixed-length encoding(定长编码)
gereral-purpose registers(通用寄存器),A-6
instruction sets(指令集),A-22
ISAs,14
Fixed-length vector(定长向量),264~284
Flags(标志)
performance bencbmarks(性能基准测试),37
performance reporting(性能报告),41
scoreboarding(记分卡),C-75
Flash memory(闪存)
characteristics(特性),102~104
depeadability(可靠性),104
memory hierarcby design(存储器层次结构设计),72
索
引
559
technologytrends(技术趋势),18
WSC cost-perfommance(WSC 成本一性能),474~475
Flexible chaining(灵活链接)
vector execution time(向量执行时间),269
Floating-point(FP)operations(浮点运算)
arithmetic intensity(算术密度),285~288,286
branch condition evaluation(分支条件判断),A-19
branches(分文),A-20
cache misses(缓存缺失),83-84
control flow instructions(控制流指令),A-21
CPI calculations(CPI计算),50~51
data access benchmarks(数据访问基准测试),A-15
data dependences(数据相关性),1S1
data hazards(数据冒险),169
dynaic scheduling with Tomasulo's algoritho(用
Tomastulo算法进行动态调度),171~172,173
exception stopping/restarting(异常停止/重新启动),
C-47
ILP exploitation(ILP开发),197~199
ILP exposure(揭示LLP),157~158
ILP in perfect processor(完美处理器中的ILP),215
ILP for realizable processors(可实现处理器的IL.P),
216-218
independent(独立),C-54
instruction operator categories(指令运算符类别),A-15
Intel Core i7,240,241
ISA performance and efficiency prediction (ISA 性能与
效率预测),241
latencies(延迟),157
MIPS,A-38~A-39
Tomasulo's algorithm(Tomasulo算法),173
MIPS exceptions(MIIPS 异常),C-49
MIPS operations(MIPS 操作),A-35
MIPS pipeline(MIIPS 流水线),C-52
MIPS precise exceptions(MIPS 精确异常),C-58~C-60
MIPS R4000,C-65~C-67,C-66-C-67
MIPS scoreboarding(MIPS 记分卡),C-77
MIPS with seoreboard(具有记分卡的MIPS),C-73
misspeculation instructions(推测错误指令),212
Multimedia SIMID Extensions(多媒体 SIMID扩展),
285
multiple lane vector unit(多车道向量单元),273
multiple outstanding(多条未完成指令),C-54
operand sizes/types(操作数大小/类型),12
parallelista vs. window size(并行度对窗口大小),217
pipeline hazards and forwarding(流水线冒险与转发),
C-S5~C-S7
pipeline structural hazards(流水线结构冒险),C-16
ROB comrit (ROB提交),187
SMT, 398~400
SPEC bencharks (SPEC基准测试),39
stalls from RAW hazards(RAW冒险导致的停顿),
C-SS
Static branch prediction(静态分支預测),C-26~C-27
Tomasulo's algorithm(Tomasulo 算法),185
VAX,8-73
vector seguence chimes(向量序列钟鸣),270
VLIW processors(VLIW处理器),195
VMIRS,264
Floating-point registers (FPRs)(浮点寄存器)
MIIPS data transfers (MIIPS 数据传输),A-34
MIPS operations(MIIPS操作),A-36
MIP$64 architecture (MIPS54体系结构),A-34
write-back(写回)、C-36
Floating-point sguare root (FPSQR)(浮点平方根)
calculation(计算),47~48
CPI caloulations(CPI计算),50~51
Flush, branch penalty reduction(冲刷,降低分支代价),
C-22
FORTRAN
compiler types and classes(编译器类型与分类),A-28
loop-level parallelismn dependences(循环级并行度相
关),320~321
MIPS scoreboarding(MIPS记分卡),C-77
returo address predictors(返回地址预測器),206
Forwarding,另见Bypassing
Four-way confict misses, definition(四路冲突缺失,定义),
B-23
FP,见Floatingpoint operations
FPRs,见 Floating-point registers
FPSQR,见 Floating-point square root
Freeze, branch penaity reduction(冻结,降低分支代价),
C-22
FU,见Functionalunit
Fully associative cache(全相联缓存)
block placement(块放置),B-7
conflict rnisses(冲突峡失),B-23
direct- mapped cache(直接映射缓存),B-9
memory hierarchy basics(存储器层次结构基础),74
Functional hazards(功能冒险)
ARM Cortex-A8, 233
microarchitectural techniques case study (徽体系结构技
术案例研究),247~254
Functional unit (FU)(功能单元)
FP operations(浮点运算),C-66
Function calls(功能调用)
GPU programming (GPU编),289
NVIDIA GPU Memory structures (NVIDIA GPU 存储器
560
索
引
结构),304~305
PTX asserbler(PTX汇编程序),301
Function pointers, control flow instruction addressing modes
(功能指针,控制流指令寻址方式),A-18
Future fle, precise exceptions (未来文件,精碗异常),
C-59
G
Gather-Scatter(集中/分散),309
GCD, 见 Greatest common divisor test
GDDR,见 Grapbics double data rate (GDDR)
GDRAM,见 Graphics dynamic random-access memory
General-purpose registers(GPRs)(通用寄存器)
advantages/disadvantages(优点/缺点),A-6
ISA classification(ISA 分类),A-3~A-5
MIPS data transfers(MIPS 数据传输),A-34
MIPS operations(MIPS操作),A-36
MIP$64,A-34
VMIPS, 265
Geometric means,cxample calculations(几何均值,示例计
算),43~44
GFS,见 Google File System
Giga Thread Engine, definition(Giga 线程引擎,定义),
292,314
Global address space, segmented virtual memory(全局地址
空间,分段虚拟存储器),B-$2
Global common subexpression elimiation, compiler structure
(全局公共子表达式消去法,编泽器结构),A-26
Global data area, and compiler technology(全局数据区域,
与编译器技术),A-27
Global load/store, definition(全局载人/存储,定义),309
Global Memory(全局存储器),292,314
Global mniss rate(全局缺失率),B-31
Global optimizations(全局优化)
compilers(编译器),A-26,A-29
optimization types(优化类型),A-28
Global predictors(全局预测器)
Intel Core i7,166
tournatnent predictors(竞赛预测器),164~166
Global scheduling, ILP, VL.IW processor(全局调度,IP,
VLIW处理器),194
Google
Bigtable,438, 441
cloud computing(云计算),455
MapReduce, 437, 458~459, 459
server CPUs(服务器CPU),440
server power-performance benchmarks(服务器功率—性
能基准测试),439~441
WSCs,432,449
containers(集装箱),464-465,465
cooling and power(制冷与功率),465-468
monitoring and repairing(监控与修复),469~470
PUE,468
servers(服务器),467,468~469
Google Clusters
memory dependability(存储器可靠性),104
Google File System (GFS)(Google 文件系统)
MapReduce,438
WSC storage(WSC存储),442~443
Google Goggles
PMDS,6
user experience(用户体验),4
Google search(Google 搜索)
shared-Iemory workloads(共享存储器工作负载),369
workload demands(工作负载需求),439
GPRs,见 General-purpose registers
GPU(Graphics Processing Unit ) banked and graphics
memory,(图形处理单元)分組与图形存储器),
322~323
GPU Memory(GPU存储器),292,309, 314
GPU programming(GPU编程),288
Graph coloring,register allocation(图形着色,寄存器分
派),A-26~A-27
Graphics double data rate(GDDR)(图形双数据率)
characteristics(特性),102
Fermi GTX 480 GPU, 294,324
Graphics dynamic random-access memory (GDRAM)(图
形动态随机访问存储器)
bandwidth iasues(带宽问题),322~323
characteristics(特性),102
Graphics-intensive benchmarks, desktop performance(图形
密集基准测试,桌面性能),38
Graphics Processing Unit,见GPU
Graphics synchronous dynamic random-access mnemory
(GSDRAM),characteristics
(图形同步动态随机访问存储器,特性),102
Greatest common divisor (GCD) test, loop-level parallelism
dependences(最大公约数测试,循环级并行度相关
性),319
Grid(网格),292,309,313
Guest definition(来宾定义),108
Guest domains, Xen VM(来宾域,Xen VM),111
H
Hadoop, WSC batch processing (Hadoop,WSC批处理),
437
Half words(半字)
aligned/misaligaed addresses(对齐/非对齐地址),A-8
menory address interpretation(存储器地址解释),
A-7~A-8
MIIPS data tyes(MIPS 数据类型),A-34
operand sizes/types(操作数大小/类型),12
Operand type(数据类型),A13~A14
Hard drive, power consumption(硬盘,功耗),63
Hardware(硬件)
architecture component(体系结构分量),15
cache optimization(缓存优化),96
energy/performance fallacies(能量/性能谬论),56
ILP approaches (ILP 方法),148,214~215
pipeline hazard detection(流水线冒险检测),C-38
Virtual Machines protection(虚拟机保护),108
WSC cost-performance (WSC成本一性能),474
WSC running service(WSC 运行服务),434~435
Hardware-based speculation(基于硬件的推测)
basic algorithm(基本算法),191
data flow execution(数据流执行),184
FP unit using Tomasulo's algorithm(使用Tomasulo算
法的深点单元),185
key ideas(关键思想),183~184
Hardware prefetching(硬件预取)
cache optimization(缓存优化),131~133
miss penalty/rate reduction(降低缺失代价/鳅失率),
91~92
NVIDIA GPU Memory structures (NVIDIA GPU 存储器
结构),305
SPEC benchrarks (SPEC 基准测试),92
Handware primitivies(硬件原语)
basic types(基本类型),387~389
synchronization mechanistns(同步机制),387~389
Header(标头)
Heap, and compiler technology(堆和编译器技术),
A-27~A-28
Heterogeneous architecture, definition(异构体系结构,定
义),262
Hewlett-Packard Prol.jant SL2x170z G6, SPECPower
benchrarks(Hewlett-Packard ProLiant SL.2x170z G6,
SPECPower 基准测试),463
High-level optimizations, compilers(高级优化,编译),
A-26
Highly parallel memory systems, case studies(高并行存储
器系统,案例研究),133~136
High-order functions, control flow jnstruction addressing
modes(高阶函数,控制流指令寻址方式),A-18
High-performance computing (HPC)(高性能计算)
write strategy(写人策略),B-10
索
引
$61
WSCs,432,435-436
History file, precise exceptions(历史文件,精确异常),
C-59
Hit time(命中时间)
average memory access time(存储器平均访问时间),
B-16~B-17
first-level caches(第一级缓存),79~80
memory hierarchy basics( 存储器层次结构基础),77~78
reduction(缩减),78,B-36~B-40
way prediction(路预测),81~-82
Home node, directory-based cache coherence protocol basics(本
地节点,目录式缀存一致性协定基础),382
Host definition(主机定义),108,305
HPC,见Higb-perfomnance computing
HP-Compaq servers(HP-Compag 服务器)
price-pertformance differences(性价比差别),441
SMT,230
Eypertranspart, AMID Opteron cache coherence(Hlypertransport,
AMID Opteron缓存一致性),361
Hypervisor, characteristics (Hlypervisor,特性),108
|
IBM
Chipkill,104
IBM 360
address space(地址空间),B-58
instruction operator categories(指令操作符类别),A-15
instruction set complications(指令集复杂性),C-49~C-50
protection and ISA(保护与ISA),112
IBM 360/91
dynarnic scheduling with Tomasulo's algorithm(用
Tomasulo算法进行动态调度),170~171
IBM 370
protection and ISA(保护与ISA),112
Virtual Machines( 拟机),110
IBM CodePack, RISC code size (IBM CodePack, RISC 代码
大小),A-23
TBM eServer p5 processor (IBM eServer p5 处理器)
performance/cost benchmarks(性能/成本基准测试),
409
SMT and ST performance(SMT 和 ST 性能),399
speedup benchmarks(加速比基准测试),408,408~409
IBM J9 JM
real-world server considerations(现实服务器考虑事项),
52~55
WSC performance(WSC 性能),463
IBM PCs, architecture flaws vs. success (IBM PC,体系结
构缺陷对成功),A-45
562
索
引
IBM Power processors(IBM Power处 器)
branch-prediction bufters(分支预测缓冲区),C-29
characteristics(特性),247
exception stopping/restarting(异常停止/重新启动),
C47
MIPS precise exceptions (MIPS精度异常),C-59
shared-me mory multiprogramming workload(共享存储
器多重编程工作负载),378
1BM Power 4
peak performance(峰值性能),$8
IBM Power 5
manufacturing cost(制造成本),62
multiprocessing/ multithreading-based performance(基
于多重处理/多线程的性能),398~400
IBM Power 7
Google WSC, 436
ideal processors(理想处理器),214~215
mnulticore processor performance(多核处理器性能),
400~401
multithreading(多线程),225
TBM servers, economies of scale(TBM 服器,规模经济),
456
IC,见 Instruction count
I-caches(1-缓存)
case sturdy examples(案例矿究示例),B-63
way prediction(路预测),81~82
ID,见 Instruction decode
Ideal pipeline cycles per instruction, ILP concepts (理想的每
指令流水线周期,ILP概念),149
Ideal processons, ILP hardware mnodel(理想处理器,ILP硬
件模型),214~215,219~220
IF,见 Instruction fetch cycle
IF statement handling(IF语句处理)
control dependences(控制相关性),154
GPU conditional branching (GPU 条件分支),300,
302~303
memory consistency(存储器一致性),392
vectorization in code(代码中的向量化),271
vector-mask registers(向量遮單寄存器),267,275~276
ILP.见 Instruction-level parallelism
Immediate addressing mode(立即数寻址方式)
ALU operations (ALU运算),A-12
basic considerations(基本考虑事项),A-10~A-11
MIPS,12
MIPS instruction format (MIIPS 指令格式),A-35
MIIPS operations(MIPS操作),A-37
value distribution(值分布),A-13
Implicit operands, ISA classifications(隐式操作数,ISA分
类),A-3
Implicit unit stride, GPUs vs. vector architectures(隐式单位
步幅,GPU对向量体系结构),310
Imprecise exceptions(非精确异常),167~170
Inactive power modes,WSCs(非活跌功率模式,WSC),
472
Inclusion(包含)
cache hierarcby(缓存层次结构),397~398
implementation(实现),397~398
invalidate protocols(失效协议),357
Indexed addressing(麥址寻址)
Indexes(索引)
address translation during(地址变换期间),B-36~B-40
AMD Opteron data cache(AMD Opteron 数据缓存),
B-13~B-14
ARM Cortex-A8,115
size equations(大小公式),B-22
Index field, block identification(索引字段,块识别),B-8
Index vectot, gather/scatter operations(索引向量,集中/分
散操作),279~280
In flight instructions, ILP hardware model(使用中的指令,
IP硬件模型),214
Information tables, examples(信息表,示例),176~177
Infrastructure costs(基础设施成本)
WSC, 446~450, 452-455,453
WSC efficiency(WSC效率),450~452
Initiation interval, MIPS pipeline FP operations (启动间隔,
MIPS 流水线浮点运算),C-52~C-$3
Initiation rate(启动速率)
floating-point pipeline(浮点流水线),C-65-C-66
memory banks(存储器组),276~277
vector execution time(向量执行时间),269
In-order commit(循序提交)
hardware-based speculation 基于硬件的推测),188~189
In-order execution(循序执行)
average memory access time(存储器平均访同时间),
B-17~B-18
cache behavior calculations(缓存行为计算),B-18
cache miss(缓存峡失),B-2~B-3
dynamic scheduling(动态调度),168~169
IBM Power processors (IBM Power处理),247
ILP exploitation(ILP开发),193~194
multiple-issue processors(多发射处理器),194
superscalar processors(超标量处理器),196
In-order floating-point pipeline, dynamic scheduling(循序浮
点流水线,动态调度),169
In-order issue(循序发射)
ARM Cortex-A8,233
dynamic scheduling(动态调度),168~170,C-71
In-order scalar processors,VMIPS(循序标量处理器,
VMIPS),267
Instruction cache(指令缓存)
AMD Opteron exa.mple (AMD Opteron示例),B-15
antialiasing(别名消失),B-38
application/OS misses(应用程序/操作系统缺失),B-59
branch prediction(分支预测),C-28
commercial workload(商业工作负载),373
GPU Memory(GPU 存储器),306
instruction fetch(指令提取),202~203,237
ISA, 241
MIPS R4000 pipeline(MIPS R4000流水线),C-63
miss rates(缺失率),161
multiprogramming workload(多重编程工作负载),
374~375
prefetch(预取),236
RISCs, A-23
Instruction commit(指令提交)
harctware-based speculation( 基于硬件的推测),184~185,
187~188,188,190
imstruction set complications(指令集复杂性),C-49
Intel Core i7,237
speculation support(推测支持),208-209
Instruction count (IC)(指令计数器)
addressing modes(寻址方式),A-10
cache performance(缓存性能),B4,B-16
compiler optimization(编译器优化),A-29,A-29~A-30
processor perfornance time(处理器性能时间),49~51
Instruction decode (ID)(指令译码(ID))
basic MIPS pipeline(基本 MIPS 流水线),C-36
branch hazards(分支冒险),C-21
data hazards(数据冒险),169
hazards and forwarding(冒险与转发),C-55-C-$7
MIIPS pipeline(MIPS 流水线),C-71
MIPS pipeline control (MIPS 流水线控制),C-36~C-39
MIPS pipeline FP operations(MIPS 浮点运算),C-53
MIPS scoreboarding(MIS 记分卡),C-72~C-74
out-of-order execution(乱序执行),170
pipeline brancb issues(流水线分支发射),C-39~C-41,
C-42
RISC classic pipeline(RISC经典流水线),C-7~C-8,
C-10
simple MIPS implementation(简单MIPS实现),C-31
simple RISC implementation(简单RISC实现),C-5~C-6
Instruction fetch (IF)cycle(指令提取周期)
basic MIIPS pipeline(基本MIPS 流水线),C-35~C-35
branch hazards(分支冒险),C-21
branch-prediction buffers(分支预测缓冲区),C-28
exception stopping/restarting(异常停顿/启),C-46~C47
MIPS exceptions(MIPS 异常),C-48
索引
563
MIPS R4000(MIPS R4000),C-63
pipeline branch issues(流水线分支发射),C-42
RISC classic pipeline (RISC经典流水线),C-7,C-10
simple MIPS implementation(简单MIPS实现),C-31
simple RISC implementation(简单.RISC实现),C-5
Instruction fetch units(指令提取单元)
integrated(集成),207~208
Intel Core i7, 237
Iastruction issue(指令发射),C-36
Instruction-level parallelism(ILP)(指令级并行),9,
149~150
Instruction path length, processor Pperformance time(指令路
径长度,处理器性能时间),49
Instruction prefetch(指令预取)
integrated instruction fetch units(集成指令提取单元),
208
rniss penalty/rate reduction(降低缺失代价/缺失率),
91~92
SPEC beachmarks(SPEC基准测试),92
Instruction register (IR)(推令寄存器(IR))
basic MIPS pipeline(基本 MPS流水线),C-35
dynamic scheduling(动态调度),170
MIPS implementation(MIPS 实现),C-31
Instruction set architecture (ISA ),另见 Intel 80x86 processors;
Reduced Instruction Set Computer (RISC)
Instructions per clock (IPC)(每时钟周期的指令数)
ARM Cortex-A8,236
flawless architecture design(无缺陷体系结构设计),
A-45
ICP for realizable processors(用于可实现处理器),
216~218
MIPS scoreboarding (MIPS 记分卡),C-72
multiprocessing/multithreading-based performance(基于
多重处理/多线程的性能),398-400
processor performance time(处理器性能时间),49
Sun T1 multichreading unicore performance (Sun TI 多
线单核性能),229
Sun Tl processor(Sun T1 处理器),399
Instruction status(指令状态)
dynamic scheduling(动态调度),177
MIPS scoreboarding(MPs记分卡),C-75
Integer operand(整数操作数)
flawed architecture(有续陷体系结构),A-44
GCD,319
eraph coloring(图形着色),A-27
instruction set encoding(指令集编码),A-23
MIPS data types(MIPS 数据类型),A-34
as operand type(作为操作数类型),12,A-13~A-14
Integer operations(整数运算)
564
索
引
addressing modes(寻址方式),A-11
ALUs,A-12,C-54
ARM Cortex-A8, 116,232.235,236
benchmarks(基准测试),167,C-69
branches(分支),A-18~A-20,A-20
cache misses(缀存缺失),83~84
data access distribution(数据访问分布),A-1S
data dependences(数据相关性),151
dependences(相关性),322
desktop benchmarks(桌南基准测试),38~39
displacement values(位移值),A-12
exceptions(异常),C-43,C-45
hardware ILP model(硬件ILP模型),215
hardware vs. sofiware speculation(硬件推测对软件推
测),221
hazards(冒险),C-57
ILP,197~200
instruction set operations(指令集运算),A-16
Intel Core i7,238,240
ISA,242,A-2
longer latency pipelines(长延迟流水线),C-SS
MIPS,C-31-C-32,C-36,C-49,C-51~C-53
MIPS64 ISA,14
MIPS FP pipeline(MIPS 淨点流水线),C-60
MIPS R4000 pipeline(MIPS R4000流水线),C-61,
C-63,C-70
misspeculation(错误预测),212
MVL, 274
pipeline scheduling(流水线调度),157
precise exceptions(精确异常),C-47,C-58, C-60
processor clock rate(处理器时钟频率),244
R4000 pipeline (R4000流水线),C-63
realizable processor [LP(可实现处理器 ILP),216~218
RISC,C-5,C-11
scoreboarding(记分卡),C-72-C-73,C-76
SIMD processor(SIMD处理器),307
SPEC benchmarks(SPEC 基准测试),39
speculation through multiple branches(多分支推测),
211
static branch prediction(静态分支预测),C-26~C-27
T1 multithreading unicore performance(T1 多线程单核
性能),227~229
Tomasulo's aigorithm(Tomasulo算法),181
tourament predictors(竞赛预测器),164
VMIPS,265
Integer registers(整数寄存器)
hardware-based speculation(基于硬件的推测),192
MIP'S dynamic instructions(MIPS 动态指令),A41~A-42
MIPS floating point operations(MIPS 浮点运算),A-39
MIPS64 architecture (MIPS 体系结构),A-34
VLIW,194
Integrated circuit basics(集成电路基础)
cost ttends(成本趋势),28~32
dependability(可靠性),33~36
logic technology(逻辑技术),17
microprocessor developments(微处理器发展),2
power and energy(功率与能量),21~23
scaling(缩放),19~21
Intel Atom,230
Intel Atom processors (Intel Atom处理器)
ISA performance and efficiency prediction (ISA 性能和
效率预测),241~243
performance measurement(性能测量),405~406
SMT,231
WSC memory(WSC存储器),474
WSC processor cost-performance(WSC处理器成本一性
能),473
Intel Core i7
Alpha processors (Alpha处理器),368
architecture(体蒸结构),15
basic function(基本功能),236~238
“big and dumb" processors (“大而被动”处理器),245
branch predictor(分支预测器),166~167
clock rate(时钟速率),244
dynamic scheduling(动态调度),170
GPU corparisons (GPU 对比),324~330,325
hardware prefetching(硬件预取),91
ISA perfortnance and efficiency prediction (ISA 性能与
效率预测),241~243
L2/L3 miss rates (L.2/L3 毓失率),125
memory hierarchy basics(存储器层次结构基础),78,
117~124,119
merory hierarchy design(存储器层次结构设计),73
mrlemory perfornance(存储器性能),122~124
MESIF protocol(MESF 协议),362
microprocessor die example(处理器晶片示例),29
miss rate benchmarks(缺失率基准测试),123
multibanked caches(多组缓存),86
multithreading(多线程),225
nonblocking cache(无阻塞缀存),83
performance(性能),239,239~241,240
perforance/energy efficiency(性能/能量效率),
401~405
pipelined cache access(流水化缓存访问),82
pipeline structure(流水线结构),237
processor comparison(处理器对比),242
raw/relative GPU performance(原始/相对GPU 性能),
328
Roofline model(屋顶轮廓线模型),286~288,287
Inter corei7(continued)
single-threaded benchmatks(单线程基准测试),243
SMP limitations(SMP 局限性),363
SMT, 230~231
snooping cache coherence implementation(监听缓存一
致性实现),365
three-level cache hierarchy(三级缓存层次结构),118
TLB structure(TLB 结构),118
write invalid protocol(写失效协议),356
Intel 80x86 processors(Intel 80x86处理器)
address space(地址空间),B-58
architecture flaws vs. success(体泰结梅缺陷与不足),
A-44~A-5
Atom,231
cache performance(缓存性能),B-6
common exceptions(常见异常),C-44
instruction set encoding(措令集编码),A-23
Intel Core i7,117
ISA,11~12,1415,A-2
memory accesses(存储器访问),B-6
memory addressing(存储器寻址),A-8
process protection(进程保护),B-50
RISC, 2, A-3
top instructions(使用较多的指令)•A-16
variable encoding(变量编码),A-22~A-23
virtualization issues(虚拟化问题),128
Virtual Machines ISA support(虚拟机 ISA 支持),109
Virtual Machines and virtual memory and I/O(虚拟机与
存储器、1/0),110
Intel LA-32 architecture (Intel IA-32体系结构)
call gate(调用门),B-54
descriptor table(描述符表),B-52
instruction set complications(指令集复杂性),C-49-C-51
segment descriptors(段描述符),B-53
segmented virtual memory(分啟虚拟存储器),
B-51~B-54
Intel 1A-64 architecture (Intel IA-64 体系结构)
multiple issue processor approaches(多发射处理器方
法),194
Intel Itanium 2
“big and dumb” processors(“大而被动”处理器),245
clock rate(时钟频率),244
peak performance( 值特性),58
SPEC benchmarks(SPEC基准测试),43
Lntel MMX, multimedia instruction compiler support
(Intel MMX,多媒体指令编译器支持),A-31~A-32
Intel Nehalem,411
Intel Pentium 4
索
引
565
Intel Pentium Il
pipelined cache access(流水化缓存访问),82
Tntel Peatium Pro,82
Intel Pentium processors (Intel Pentium 处理器)
“big and dumb"processors(“大而被动”处理器),245
clock rate(时钟频率),244
Opteron memory protection(Opteron 存储器保护),B-57
pipelining perforrance(流水线性能),C-10
segmented virtual memory example(分段虚拟存储器示
例),B-51~B-54
SMT,230
Intel processors( Intel 处理器)
carly RISC designs(早期RISC设计),2
Intel Streaming SIMD Extension (SSE)(Intel 流式 SIMD
扩展)
basic function(基本功能),283
Multimedia SIMD Extensions(多媒体SIMD扩展),
A-31
vector architectures(向量体系结构),282
Interactive workloads, WSC goals/ requirements(交互式工
作负载,WSC目标/需求),433
Interconnection networks(互连网络)
multicore single-chip multiprocessor(多核单芯片多处
理器),364
Internal fragmentation, virtual memory page size selection(内
部碎片,虚拟存储器页大小选择),B-47
Internal Mask Registers, definition(内部遮單寄存器,定
义),309
Interrupt,见 Exceptions
Invalidate protocol(使协议失效)
directory-based cache coherence protocol cxample(目录
式缓存一致性协议示例),385~383
example(示例),359, 360
immplementation(实现),356~357
snooping coherence(监听一致性),355,355~356
Inverted page table, virbual memory block identification (反
转分页表,虚拟存储器块识别),B-44-B-45
VO bound workload, Virtual Machines protection (1/0限工
作负载,虚拟机保护),108
VO cache coherency, basic considerations (1/O缓存一致性,
基本考虑事项),113
1/O devices(LO设备)
address translation(地址转换),B-38
average memory access time(存储器平均访问时间),
B-17
cache coherence enforce ment(缓存一致性实施),354
centralized shared-memory multiprocessors(集中共享存
储器多处理器),351
future GPU features(未来GPU特征),332
566
索
引
inchusion(包含),B-34
Multimedia SIMID vs.GPUs(多媒体SIMD对 GPU),
312
multiprocessor cost effectiveness(多处理器成本效率),
407
Virtual Machines impact(虚拟机影响),110~111
write strategy(写策略),B-11
Xen VM,111
1/O Latency, shared-memory workloads(1/O延迟,共享存
储器工作负载),368~369,371
VO registers, write buffer merging(1/O寄存器,写缓冲区
合并),87
Issue logic(发射逻辑)
ARM Cortex-A8(ARM. Cortex-A&),233
ILP,197
Longer latency pijpelines(长延退流水线),C-57
mnultiple issue processor(多发射处理器),198
register renaming vs.ROB(寄存器重命名对ROB),
210
speculation support(推測支持),210
Issue stage(发射级)
ID pipe stage(ID 流水級),170
instruction steps(指令步骤),174
MIPS with scoreboard( 具有记分卡的MIIPS),C-73~C-74
out-of-order execution(乱序执行),C-71
ROB instruction (ROB 指令),186
J
Java benchmarks(Java 基准测试)
Intel Core i7, 401~405
SMT on superscalar processors(趙标量处理器上的
SMT),230-232
without SMT(没有 SMT),403~404
Java language(Java 语言)
hardware impact on software development(硬件对软件
开发的影响),4
return address predictors(返回地址预测器),206
SMT,230~232,402-405
sPECjbb, 40
SPECpower,$2
virtual functions/methods(虛拟函数/方法),A-18
Java Virtual Machine (JVM)(Java 虚拟机)
IBM,463
multicore processor performance(微处理器性能),400
multithreading-based speedup(基于多线程的加速比),
232
SPECjibb,$3
JBOD, 见RAID O
Jump prediction(Jump 预测)
hardware model(硬件模型),214
ideal processor(理想处理辦),214
Jumps(跳转)
control flow instructions(控制流指令),14,A-16,
A-17,A-21
GPU conditional branching(GPU条件分支),301~302
MIPS control flow instructions (MIRS 控制流指令),
A-37~A-38
MIPS operations(MIIPS 操作),A-35
return address predictors(返回地址预测器),206
RISC instruction set (RISC 指令集),C-S
JM,见 Java Virtual Machine
K
Kerneis(内核)
arithmetic intensity(算术密度),286,286~287,327
benchmarks(基准测试),56
bytes per reference, vs. block size (每次引用的字节数,
与块大小),378
caches(缓存),329
comrercial workload(商业工作负载),369~370
compilers(编译器),A-24
compute bandwidth(计算带宽),328
via computing(通过计算),327
EEMIBC benchmarks (EEMBC 基准测试),38
FP benchmarks(浮点基准测试),C-29
Livermore Fortran kernels (Livernore Fortran 内核),
331
multimedia instructions(多媒体指令),A-31
multiprocessor architecture(多处理器体系结构),408
muitiprogramming workload(多重编程工作负载),
375~378,377
pertormance benchmarks(性能基准测试),37,331
primitives(原语),A-30
protecting processes(保护进程),B-50
segmented virtual memmory(分段虚拟存储器),B-51
SIMD exploitation (SIMD开发),330
vector, on vector processor and GPU(向量,在向量处
理器和 GPU上),334~336
virtual memory protection(虚拟存储器保护),106
WSCs.438
L
L1 caches,另见 First-level caches
L2 caches,另见 Second-level caches
L3 caches,另见 Third-level caches
Lanes(车道)
GPUs vs.vector architecture(GPU 与向量体系结构),
310
Sequence of STMID Lane Operations (SIMID 车道操作
序列),292,313
SIMD Lane Registers(SIMID 车道寄存將),309,314
SINID Lanes(SIMID车道),296~297,297,302~303,
308,309,311~312,314
vector execution time(向量执行时间),269
vector instruction set(向量指令集),271~273
Vector Lane Registers(向量车道寄存器),292
Vector Lanes(向量车道),292,296~297,309,311
Latency, 另见 Response time
Layer 3 network, array and Internet linkage(L.3 网络,阵列
与互联网连接),445
Layer 3 network, WSC mnerory hierarchy(L3网络,WSC
存储器层次结构),445
Leaming curve, cost trends(学习曲线,成本直接),27
Least recently used (LRU)(最近使用)
AMD Opteron data cache (AMD Opteron 数据缓存),
B-12,B-14
block replacement(块替换),B-9
virtual memory block replacemnent(虚拟存储器块替
换),B-45
Level 3, as Content Delivery Network(L3,作为内容交付
网络),460
Limit field, IA-32 descriptor table(界限宇段,1A-32 描述
符表),B-52
Line, memory hierarchy basics(行,存储器层次结构基础),
74
Linear speedup(线性加速比)
cost effectiveness(成本效率),407
IBM eServer p5 multiprocessor (IBM eServer p5 多处理
器〉,408
multicore processors(多核处理器),400,402
performance(性能),405~406
Link register(链接寄存器)
MIPS control flow instructions (MIPS控制流指令),
A-37~A-38
procedure invocation options(过程调用选项),A-19
synchronization(同步),389
Linpack benchmark(Linpack 基准测试)
vector processor example(向量处理器示例),267-268
Linux operating systems(Linux 操作系統)
Amazon Web Services(Amazon Web 服务),456-457
architecture costs(体系结构成本),2
protection and ISA(保护与ISA),112
WSC services(WSC服务),441
索
引
567
Lisp
ILP,215
aS MapReduce inspiration(为 Mapkeduce启发),437
Literal addressing mode, basic considerations(直接数寻址
方式,基本考虑事项),A-10~A-11
Little Endian(小端)
memory address interpretation(存销器地址解释),
A-7
MIPS data transfers(MIPS 数据传输),A-34
Liveness, control dependence(活性,控制相关性),156
Livermore Fortran kernels, performance(Livermore Fortran
内核,性能),331
LMD,见 Load memory data
Load instructions(载人指令)
control dependences(控制相关性),15S
data hazards requiring stall(需要停顿的数据冒险),
C-20
dynamic scheduling(动态调度),177
ILP,199,201
loop-level parallelism(循环级并行),318
memory port conflict(存储器端口冲突),C-14
pipelined cache access(流水化缓存访问),82
RISC instruction set(RISC 指令集),C-4~C-5
Tomasulo's algorithm(Tomasulo算法),182
VLIW samuple code(VLIW示例代码),252
Load linked(链接载人)
locks via coherence(通过一致性锁定),391
synchronization(同步),388~389
Load locked, synchronization(锁定载人,同步),388~389
Load memory data(LMD), simple MIPs implementation
(载人存储器数据,简单 MITPS实现),C-32~C-33
Load stails, MIPS R4000 pipeline(载人停顿,MIPS R4000
流水线),C-67
Load-store instruction set architecture(载人-存储指令集体
系结构)
basic concept(基本概念),C-4-C-5
Intel Core i7,124
ISA,11
ISA classification(ISA分类),A-5
MIPS operations(MIPS 运算),A-35~A-36,A-36
simple MIPS impiementation(简单 MIPS实现),C-32
VMIPS, 265
Load/store unit(载人/存储单元)
Fermi GPU, 305
ILP hardware model(ILP硬件模型),215
multiple lanes(多车道),273
Tomastulo's algorithm(Tomasulo 算法),171~173,
182,197
vector units(向量单元),265,276~277
568
索引
Losd upper immediate (LUI),MIPS operations(载入高位
立即数(LUI),MIPS操作),A-37
Local address space, segmented virtual memory(载人地址
空间,分段拟存储器),8-52
Locality,见 Principle of locality
Local Memory(本地存储器),292,314
Local miss rate,definition(局部缺失率,定义),B-31
Local node, directory-based cache coherence protocol basics
(本地节点,目录式缓存一致性协议基础),382
Local optimizations, compilers(局部优化,编译器),A-26
Local predictors,tournarnentpredictors(局部预测器,竞赛
预测器),164~166
Local scheduling, ILP, VLIW processor(局部调度,LP,
VLIW处理器),194195
Locks(锁)
via coherence(通过一致性),389-391
hardware primitives(硬件原语),387
multiprocessor so ftwvare development(多处理器软件开
发),409
Lock-up free cache(自由查询缓存),83
Long integer(长整数)
operand sizes/types(操作数大小/类型),12
SPEC benchmarks(SPEC 基准测试),A-14
Loop-carried dependences(循环间相关),315~316
Loop exit predictor, Intel Core i7(循环退出预测器,Intel
Core i7),166
Loop interchange, compiler optimizations(循环交换,编译
器优化),88~89
Loop-level parallelism(循环级并行),149~150
Loop stream detection, Intel Core i7 micro-op buffer(循环流
检测,Iatel Core i7 微操作缓冲区),238
Loop unrolling(循环展开)
basic considerations(基本考虑事项),161~162
ILP exposure(揭示 ILP),157~161
ILP limitation studies(IP限制性研究),220
Tomasulo's algorithm(Tomasulo算法),179,181~183
VLIW processors(VLIW处理器),195
LRU,见 Least recently used
Lucas
compiler optimizations(编译器优化),A-29
data cache misses(数据缓存敏失),B-10
LUI,见 Load upper immediate
M
Machine memory, Virtual Machines(机器存储器,虚拟机),
110
Macro-op fusion, Intel Core i7(微操作融合,Intel Corei7),
237~238
Main Memory(主存储器),292,309
Manufacturing cost(制造成本)
chip tabrication cnse study(芯片制造案例研究),61~62
cost tends(成本趋势),27
moder processors(现代处理器),62
operation cost(运行成本),33
MapReduce
cloud computing(云计算),455
cost calculations(成本计算),458~460,459
Google usage(Google应用),437
reductions(减少),321
WSC batch processing (WSC批处理),437~438
WSC cost-performance(WSC成本性能),474
Mask Registers(遮罩寄存器),309
Matrix300 kermel (Matrix 300内核),56.
Matrix multiplication(矩阵乘法)
benchmarks(基准测试),56
muktidimensional artays in vector architectures(向量体
系结构中的多维数组),278
Maximum vector length(MVL.)(最大向量长度)
Multimedia SIMD extensions(多媒体 SIMID扩展),282
vector vs.GPU(向量对GPU),311
VLRs,274~275
M-bus,见 Memory bus
MCF
compiler optimizations(编译器优化),A-29
data cache raisses(数据缓存缺失),B-10
Intel Core i7,240~241
Mean tirne between failures (MTBF)(平均故障间隔时间)
fallacies(谬论),56-57
SLA states(SLA状态),34
Mean time to failure (MTTF)(平均无故障时间)
computer system power consumption case study(计算
机系统功耗案例研究),63-64
example calculations(示例计算),34~35
SLA states(SLA状态),34
WSCs vs.servers(WSC对服务器),434
Memory access(存储器访问)
ARM Cortex-A8 example(ARM Cortex-A8示例),117
basic MIPS pipeline(基本MIPS流水线),C-36
block size(块大小),B-28
cache hit calculation(缓存命中计算),B-S
data hazards reguiring stall(需要停顿的数据冒险),
C-19~C-21
data hazard stall mninimization(数据冒险停顿最小化),
C-17,C-19
integrated instruction fetch onits(集成指令提取单元),
208
MIPS data transfers (MIPS 数据传输),A-34
mwltimedia instruction compiler suppont(多媒体指令编
译嬲支持),A-31
pipeline branch issues(流水线分支发射),C-40,C-42
RISC claseic pipeline (RISC经典流水线),C-7,C-10
shared-mnemory workdloads (共享存储器工作负载),
$72
simple MIPS implementation(简单 MIPS实现),
C-32~C-33
simple RISC implementation(简单RISC实现),C6
huchural hazards(结构性冒险),C-13~C-14
Memory addressing(存储器寻址)
ALU imnmediate operands(ALU 立即数操作数),A-12
basic considerations(基本考虑事项),A-11~A-13
displacement values(位移值),A-12
immediate value distribution(立即数值分布),4-13
interpretation(解粹),A-7~A-8
ISA, 11
Memory banks,另见 Banked memory
Memory bus(M-bus)(存储器总线(M-总线)),351
Memnory consistency(存储器一致性)
basic considerations(基本考虑事项),392~393
cache coherence(缓存一致性),352
compiler optimization(编译优化),396
directory-based cache coherence protocol basics(目录
式缓存一致性基础),382
mruiltiprocessor cache coherency(多处理將缓存一致性),
353
relaxed consistency models (宽松连贯性模型),394~395
single-chip multicore processor case study(单芯片多核
处理器案例研究),412-418
speculation to hide latency( 以推測隐藏延迟),396~397
Memnory hierarchy(存储层次结构)
adidress space(地址空间),B-57~B-58
basic questions(基本公式),B-6~B-12
block identification(块识别),B-7~B-9
block placement issues(块放置向題)B-7
block replacement(块替換),B-9~B-10
cache optimization(缓存优化),B-22~B-40
cache performance(缓存性能),B-3~B-6
case studies(案例研究),B-60~B-67
inclusion(包含),397~398
levels in slow down(减缓级别),B-3
Opteron data cache exannple ( Opteron 数据缓存示例),
B-12~B-15,B-13
Opteron L.1/L.2, B-57
OS and page size(OS 和页大小),B-58
overview(概述),B-39
Peantinm vs. Optexon pacotection(Peasttum 与 Opteron 保护),
B-57
索
引
569
processor exasnples(处理舉示例),B-3
process protection(进程保护),B-50
terminology(术语),B-2~B-3
virtual memory(虚拟存储器),B-40~B-54
wite strategy(写筆略),B-10~B-12
WSCs, 443, 443~446,444
Memory hieratchy design(存储器层次结构设计)
access times(访问时间),77
Alpba 21264 floorplan (Alpha 21264平面图),143
ARM Cortex-A8 example (ARM Cortex-A8示例),
114~117,115-117
cache cohereney(缓存一致性),112~113
cache optimization(缓存优化),81~133
cache performnance prediction(缓存性能预測),
125~126
cache size and misses per instruction(缓存大小与每指
令的觖失数),126
DDR2 SDRAM tining diagram(DDR2 SDRAM时序图),
139
highly parallel memory systers(高度并行存储器系
统),133~136
high memory bandwvidth(高存储器带宽),126
instruction mniss benchmarks(指令缺失基准测试),
127
instruction simulation(指令模拟),126
Intel Core i7, 117~124,119,123~125
Intel Core i7 three-level cache hierarchy (Intel Core i7
三级缓存层次结构),118
Intel Core i7 TLB structure (Intel Core i7 TLB 结构),
118
Intel 80x86 virtualization issues (Intel 80x86 處拟问
题),128
memory basics(存储器基础),74~78
overview(概述),72~74
protection and ISA(保护与ISA),112
server vs. PMID(服务器对PMD),72
system call virua lization/ paravirtualizationper formance
(系统调用處拟/都分虚拟化性餡),141
virtual mmachine mmonitor(虚拟机监视器),108~109
Virtual Machines ISA support(虛拟机 ISA 支持),109~110
Virtual Machines protection(越拟机保护),107~-108
Virtual Machines and virtual mnemory and V/O(虚拟机
和虚拟存储器、L/O),110~111
virtual memory protection( 虚拟存储器保护),105~107
VMM on nonvirtualizable ISA(不可虚拟化ISA上的
VMM),128~129
Xen VM example(Xen VM示例),111
Memory Interface Unit(存储器接口单元)
NVIDIA GPU ISA,300
570
索引
vector processor example(向量处理器示例),310
Memory mapping(存储器映射)
memory hierarchy(存储器层次结构),B-48~B-49
segmented virtual memory(分啟虚拟存储器),B-52
TLBS,323
vixtual memory definition(虛拟存储器定义),B-42
Memory-memory instnuction set anchitecture, ISA classification,
•(存储器-存储器指令集体系结构,ISA分类),A-3,
A-5
Memory protection(存储器保护?
control dependence(控制相关性),155
Pentium vs. Opteron(Pentium 对 Opteron),B-57
processes(进程),B-$0
safe calls(安全调用),B-54
segmented virtual memory example(分段虚拟存储器
示例),B-51~B-54
vintual memory(虚拟存储器),B-41
Memory stall(存储器停顿),B-4~B-5
Memory technology basics(存储器技术基础)
DRAM,98,98~100,99
DRAM and DIMM characteristice(DRAM 和 DIMM 特
性),101
DRAM performance(DRAM性能)
,100~102
Flash memory(闪存),102~104
overview(概述),96-97
performance trends(性能趋勢),20
SDRAM power consutption (SDRAM功耗),102,
103
SRAM, 97~98
MESI,见 Modified-Exclusive-Shared-Invalid(MESI)
protocol
Messages(消息)
coherence maintenance(一致性保持),381
Microarchitecture(微体系结构)
architecture component(体系结构组件),15~16
ARM Cortex-A8,241
data hazards(数据冒险),168
ILP exploitation(ILP开发),197
Intel Core i7,236~237
Nehalem,411
out-of-order example(乱序示例),253
PTX vs.x86(PTX对x86),298
study,47~254
Microfusion, Intel Core i7 mnicro-op buffer(微合并,Intel
Core i7微操作缓冲区),238
Microinstructions(微指令)
Complications(复杂性),C-50~C-51
×86,298
Micro-ops(微操作)
Intel Core i7, 237,238~240,239
processor clock rates(处理器时钟频率),244
Microprocessor overview(微处理器概述)
clock rate trends(时钟频率趋势),24
cost trends(成本趋势),27~28
desktop computers(台式计算机),6
embedded computers(嵌人式计算机),8~-9
energy and power(能量与功率),23~26
integrated circuit improvements(集成电路改进),2
Moore's Lav(康尔定律),3-4
performance trends(性能趋势),19~20,20
power and energy system trends(功率与能量系統趋
势),21~23
bechnology trends(技术趋势),18
Microprocessor without Interlocked Pipeline Stages, 见 Mis
( Microprocessor without Interlocked Pipeline Stages)
Microsot(微软)
cloud computing(云计算),455
Intel support (Intel 支持),245
WSC,464~465
Microsoft Azure, 465
Microsoft Windows
benchmarks(基本测试),38
mrultithreading(多线),223
time/vohume/commoitization impact(时间/数量/大众
商品化影响),28
WSC workloads(WSC 工作负载),441
Microsoft Windows 2008 Server
real-world consicerations(现实考虑事项),52~55
SPECpower benchmark(SPECpower 基准测试),463
Migration, cache coherent multiprocessors(迁移,缓存一致
性多处理器),354
MIMD( Multiple Instruction Streams, Multiple Data Steans)
and Amdabl's law(多指令流,多数据流)与 Amdahl
定律),406~407
Minicomputers, replacement by microprocessors (小型计算
机,被微处理器代替),3~4
Minniespec benchmatks(Minniespec 基准测试)
ARM Cortex-A8, 116,235
ARM Cortex-A8 mnemory (ARM Cortex-A& 存储器),
115~116
MIPS(Microprocessor without Interlocked Pipeline Stages)
(没有互连流水级的多级互连)
addressing modes(寻址方式),11~12
basic pipeline(基本流水线),C-34~C-36
branch predictor correlation(分支预測器相关),163
cache performance(缓存性飽),B-6
control flow instructions(控制流指令),14
data dependences(数据相关),151
data hazards(数据冒险),169
dynamic scheduling with Tomasulo's algorithm(采用
Tomasulo算法的动态调度),171,173
encoding(编码),14
exceptions(异常),C-48,C-48~C-49
exception stopping/res tarting(异常停止/重启),C46-C-47
FP pipeline performance(浮点流水线性飴),C-60~C-6l,
C-62
FP unit with Tomasulo's algorithm(采用Tomasulo 算法
的浮点单元),173
hazard checks(冒險检测),C-71
ILP,149
ILP exposure(揭示[LP),157~158
ILP hardware model(ILP硬件模型),215
instruction set complications(指令集复杂性),C-49-C-51
ISA class (ISA类),11
ISA example(ISA 示例),A-22~A-42
Livermore Fortran kernel performancd Livermore Fortran
内核性能),331
memory addressing(存储器寻址),11
basic considerations(基本考思事项),C-51~C-54
hazards and forwarding(冒险与转发),C-54~C-58
precise exceptions(精确异常),C-58~C-60
operands(操作数),12
pipeline branch issues(流水线分支发射),C-39~C-42
pipeline control(流水线控制),C-36-C-39
pipe stage(流水級),C-37
processor performance calculations(处理器性能计算),
218~219
registers and usage conventions(寄存器与使用约定),12
RISC code size(RISC代码大小),A-23
scoreboard components(记分卡组件),C-76
scoreboarding(记分卡),C-72
scoreboarding steps(记分卡步骤),C-73,C-73~C-74
Simple implementation(简单实现),C-31~C-34,C-34
unpipelined functional units(非流水化功能单元),C-52
write strategy(写策略),B-10
MIPS16
instruction set arcbitecture formats(指令集体系结构结
构),14
instruction subset(指令子集),13,A-40
in MIPS R4000(在 MIPS R4000中),C-61
RISC instruction set (RISC 指令集),C-4
MIPS R4000
FP pipeline(浮点流水线),C-65~C-67,C-66
iteger pipeline(整数流水线),C-63
pipeline overview(流水线概述),C-61~C-65
pipeline performance(流水线性能),C-67~C-70
pipeline structure(流水线结构),C-62-C-63
索引
571
• MIPS R8000 precise exceptions (MIPS R8000 精确异常),
C-59
MIPS R10000,81
latency hiding(隐藏延迟),397
precise exceptions(精确异常),C-59
Misalignment, memory address interpretation(非对齐,存
储器地址解释),A-7~A-8,A-8
Misprediction rate(错误预测率)
branch-prediction buffers(分支预测缓冲区),C-29
predictors on SPEC89(关于 SPEC89的预测器),166
profle-based predictor(基于一览数据的预测器),C-27
SPECCPU2006 benchmarks ( SPECCPU2006 基准测
试),167
Mispredictions(错误预测)
ARM Cortex-A&,232,235
branch predictors(分支预测器),164~167,240,C-28
branch-target buffers(分支目标缓冲区),205
hardwvare-based speculation(基于硬件的推测),190
hardware vs. software speculation(硬件推测与软件推
测),221
integer Vs. FP programs(整数程序与浮点程序),212
Intel Core i7, 237
prediction buffers(预测缓冲区),C-29
static branch prediction(静态分支预测),C-26~-C-27
Misses per instruction(每指令觖失数)
application/OS statistics(应用程序/操作系统统计数
字),B-59
cache performance(缓存性能),B-5~B-6
cache protocols(缓存协议),359
cache size cffect(缀存大小影响),126
L.3cache block size(L3緩存块大小),371
memory hierarchy basics(存储器层次结构基础),75
performance impact calculations(性能影响计算),B-18
shared-memory workloads(共享存储器工作负藏),
372
SPEC benchmarks(SPEC基准测试),127
strided access-TLB interactions(步骤访问-TLB交互),
323
Miss penalty(缺失代价)
average memory access time(存储器平均访问时间),
B-16~B-17
cache optimization(缓存优化),79,B-35~B-36
cache performance(缓存性能),B-4,B-21
compiler-controled prefetching(编译器控制预取),
92~95
critical word first(关键字优化),85~87
hardware prefetching(硬件预取),91~92
ILP speculative execution(ILP 推测执行),223
memory hierarchy basics(存储器层次结构基础),75~76
572
索
引
nonblocking cache(非阻塞缓存),83
out-of-order processors(乱序处理器),B-20~B-22
processor performance calculations(处理器性館计算),
218~219
reduction via multilevel caches(通过多级缓存来降低),
B-30~B-35
write buffer merging(写缓冲区合并),87
Miss rate(缺失率)
AMD Opteron data cache (AMI Opteron 数据缓存),
B-15
ARM Cortex-A8,116
average mnemory access time(存储器平均访问时间),
B-16~B-17,B-29
basic categories(基本类别),B-23
block size(块大小),B-27
cache optimization(缓存优化),79
cache performance(缀存性能),B-4
cache size(缓存大小),B-24~B-25
compiler-controlled prefetching(编译器控制预取),
92~95
compiler optimizations(编译器优化),87~90
example calculations(示例计算),B-6,B-31~B-32
hardware prefetching(硬件预取),91~92
Intel Core i7, 123, 125, 241
memory hierarchy basics(存储器层次结构基础),75~76
multilevel caches(多級缓存),B-33
processor performance calculations(处理器性能计算),
218~219
shared-mnemory multiprogramming workload(共享存储
器多重编程工作负载),376,376~377
shared-memory workload(共享存储器工作负载),
370~373
single vs.multiple thread executions(单线程与多线程执
行),228
Sun Tt moltithreading unicore perforance (Sun T1 多
线程单核性能),228
Virtual addressed cache size(虚拟寻址缓存大小),B-37
Mixed cache(混合式缓存)
AMID Opteron exaruple(AMID Opteron 示例),B-15
commercial workload(商业工作负载),373
MMX,见 Multimedia Extensions
Mobile clients(移动客户端)
data usage(数据使用),3
GPU features(GPU 特征),324
server GPUs(服务器 GPU),323~330
Modified-Exclusive-Shared-Invalid (MESI) protocol(修
改-排除-共享-废弃(MESI)协议)
characteristics(特性),362
Modified-Owned-Exchusive-Shared-Invalid (MOESI)protocol
characteristics(修改-拥有-排除-共享-废弃(MOESI)
协议特性),362
Modified state(经修改状态)
coherence protocol(一致性协议),362
directory-based cache coherence(目录式缓存一致性)
protocol basics(协议基础),380
snooping cohereace protocol(监听一致性协议),
358~359
Module availability, definition(模块可用性,定义),34
Module reliability, definition(模块可靠性,定义),34
Moore's law(摩尔定律)
DRAM,100
flawed architectures(有缺陷体系结构),A-45
microprocessor dominance(处理器优势),3~4
RISC, A-3
softwvare importance(软件重要性),$5
technology trends(技术趋势),17
Mortar shot graphs, multiprocessor performance measurement
(迫击炮射击曲线,多处理器性能测量),405~-406
Multibanked caches(多组緩存)
cache optinization(缀存优化),85-86
example(示例),86
Multicomputers(多计算机),345,2-59
Multicore processors(多核处理器)
architechre goals/reqguirements(体系结构目标需求),15
cache coherence(缓存一致性),361~362
centralized shared-memory Imultiprocessor structure(樂
中式共享存储器多处理器结构),347
directory-based cache coherence(目录式缀存一致性),
380
directory-based coherence(目录式一致性),381,419
DSM architecture(DSM 体系结构),348,379
multichip(多芯片),419
multiprocessors(多处理器),345
perfornance(性能),400~401,401
performance gains(性能增益),398-400
performance milestones(性能里程碑),20
single-chip case study(单芯片案例研究),412-418
SMT,404~405
snooping cache coherence implementation(监听缓在一
致性实现),365
SPEC benchmarks(SPEC基准测试),402
uniforn memory access(一致存储器访问),364
write ivalidate protocolimplementation(写失效协议实
现),356~357
Milticycle operations, MIPS pipeline(多周期操作,MIPS
流水线)
basic considerations(基本考感事项),C-51~C-54
bazards and forwarding(冒险与转发),C.54-C-58
precise exceptions(精确异常),C-58~C-60
Multidirensional arrays(多维数组)
dependences(相关性),318
in vector architectures(向量体系结构中),278~279
Multilevel caches(多級級存)
cache optimizations(缓存优化),B-22
centralized shared-memory architectures(集中式共享存
储器体系结构),351
mnemory hierarchy basics(存储器层次结构基础),76
miss penalty reduction(降低缺失代价),B-30~B-35
miss rate vs. cache size(缺失率与缓存大小),B-33
Multimedia SIMID vs. GPU(多媒体 SIMD与 GPU),312
Multilevel exchusion, definition(多级互斥,定义),B-35
Multilevel inclusion(多级包含),397, B-34
Multimedia applications(多媒体应用程序)
GPUs,288
ISA support (ISA 支持),A-46
MIPS FP operations(MIPS浮点运算),A-39
vector architec tures(向量体系结构),267
Multimnedia Extensions (MIMXX)(多媒体扩展(MIMIX))
compiler support(编译器支持),A-31
SIMID history (SIMID历史),262
vector architectures(向量体系结构),282-283
Multimedia instructions(多媒体指令)
ARM Cortex-A8,236
compiler support(编译器支持),A-31~A-32
Multimedia SIMID Extensions(多媒体 SIMDD扩展)
basic considerations(基本考虑事项),262,282~284
compiler suppont(编译支持),A-31
DLP, 322
GPUs, 312
MIMD, vs. GPU (MIMDD 与 GPU),324~330
parallelism ciasses(并行分类),10
programming(编程),285
Roofine visual performance model(屋顶轮廓线虚拟性
能模型),285~288,287
256-bit-wide operations(256位宽操作),282
vector(向量),263~264
Multiredia user interfaces, PMIDs(多媒体用户接口,PMDD),
6
Multiple Instruction Streams, Multiple Data Streams, 见
MIIMID
Multiple Instruction Streams, Single Data Stream (MISD),
definitio(多指令流,单数据流(MISD),定义),10
Multiple-issue processors(多发射处理器)
basic VLIW approach(基本VLIW方法),193~196
dynamic scheduling and speculation(动态调度与推测),
197~202
instruction fetch bandwidth(指令提取带宽),202-203
索
引
573
integrated instruction fetch units(集成指令提取单元),
207
loop unrolling(循环展开),162
microarchitecbural techniques case study(微体系结构技
术案例研究),247~254
primary approaches(主要方法),194
SMT,224,226
speculation(推测),198
Tomasulo's algorithm(Tomasulo算法),183
Multiple lanes technique (多车道技术)
vector instruction set(向量指令集),271~273
Multiple paths, ILP limaitation studies(多路径,ILP局限性
研究),220
Multiply operations(乘法运算)
unfinished instructions(未完成指令),179
Multiprocessot basics(多处理器基础),354
Multiprograming(多重编程),345
Multithreaded SIMID Processor(多线程 SIMD处理器),
292,309,313-314
Multithreaded vector processor(多线向量处理器),292
Multithreading(多线程),223~-225
MVL,见 Maxirnum vector leagth
N
Nare dependences(名称相关性)
ILP,152~153
locating dependences(定位相关性),318~319
loop-level parallelism(循环级并行),315
scoreboarding(记分卡),C-79
Tomasulo's algorithm(Tomasulo算法),171~172
Nameplate power rating,WSCs(名牌额定功率),449
NAND Flash, definition (NAND Flash,定义),103
Natural parallelism(自然并行)
multiprocessor importance(多处理器重要性),344
multithreading(多线程),223
NEC Earth Simulator, peak performance( NEC Earth 模拟器,
值性能),58
NEC SX-9
Roofline model(屋顶轮廓线模型),286-288,287
Nested pege tables(嵌套页表),129
NetApp,见 Network Appliance
Netflix, AWS,460
Network Appliance(NetApp)(网络设备)
Network attached storage (NAS)(网络连接存储)
WSCs,442
Network File System (NFS)(网络文件系统)
server benchmarks(服务器基准测试),40
Networking costs,WSC vs. datacenters(联网成本,WSC
574
索
引
与数据中心),455
Network interface card (NIC)(网络接口卡(NIC))
Google WSC servers(Google WSC 服务器),469
Network technology,另见 Interconnection netwozks
NFS,见 Network File System
NIC,见 Network interface card
NMOS, DRAM,99
Nodes(节点)
coherence mnaintenance(一致性维护),381
directory-based cache coherence(目录式级存一致性),
380
parallel(并行),336
Nonatomic operations(非原子操作)
cache coherence(缓存一致性),361
Nonbinding prefetch, cache optimization(非绑定预取,缓
存优化),93
Nonblocking caches(非阻塞緩存)
cache optimization(缓存优化),83~85,131~133
effectiveness(有效性),84
ILP speculative exeoution(IP推测执行),222~223
Intel Core i7, 118
Nonfaulting prefetches,cache optirization(无错误预取,
缓存优化),92
Nonumiform memory access (NUMA)(非一致存储器访问)
DSM as, 348
snooping limitations(监听局限性),363~364
Non-unit strides(非单位步幅)
multidimnensional arrays in vector architectures(向量体
系结构中的多维数组),278~279
vector processor(向量处理器),310,310~311
No-write allocate(非写入分派),B-11
Nullifying branch, branch delay slots(废弃分支,分支延迟
时隙),C-24~C-25
NUMA,见 Nonuniform memory access
NVIDIA systems (NVIDIA 系统)
fine-grained multithreading(细粒度多线程),224
GPU comparisons(GPU对比),323~330,325
GPU computational structures (GPU 计算结构),
291~297
GPU ISA,298~300
GPU Memmory structures (GPU 存储器结构),304,
304~305
GPU programming(GPU编程),289
terminology(术语),313~315
N-way set associative(N路组相联)
block placement(块放置),B-7
conflict mnisses(冲突觖失),B-23
memory hierarchy basics(存储器层次结构基础),74
TLBs,B-49
◎
Observed performance, fallacies(观测性能,營论),57
Offiset(偏移量)
addressing modes(寻址方式),12
AMD64 paged virtual memory(AMD64分页虚拟存備
器),B-$$
block identification(块识别),B-7~B-8
cache optimization(缓存优化),B-38
cell gates(调用门),B-54
control flow instructions(控制流指令),A-18
directory-based cache coherence protocols(目录式缓存
一致性协议),381~382
example(示例),B-9
gather-scatter(集中-分散),280
IA-32 segment(IA-32分段),B-53
instruction decode(指令译码),C-5~C-6
main memory(主存储器),B-44
memory mapping(存储器映射),B-52
MIPS, C-32
MIPS control flow instructions (MIPS 控制流指令),
A-37~A-38
misaligned addresses(非对齐地址),A-8
Opteron data cache(Opteron 数据缓存),B-13~B-14
pipelining(流水化),C-42
PTX instructions(PTX 指令),300
RISC,C-4~C-6
RISC instruction set(RISC指令集),C-4
TLB, B-46
Tomasulo's approach(Tomasulo方法),176
virtual memory(虚拟存储器),B-43~B-44,B-49,
B-55~B-56
OLTP,见 On-Line Transaction Processing
OMINETPP, Intel Core i7, 240~241
On-chip cache(片上缀存)
optirnization(优化),79
SRAM,98~99
One-way con flict misses,definition (1路冲突鲛失,定义),
B-23
On-Line Transaction Processing (OL.P)(联机事务处理)
commerciai workiload(商业工作负载),369,371
server benchmatks(服务器基准测试),41
shared-mmemoryWorkeloads (共享存储器工作负载),
368-370,373~374
OpenCL
GPU programming(GPU 编程),289
GPU terminology (GPU 术语),292,313~315
NVIDIA terminology(NVIDIA 术语),291
processor comparisons(处理器对比),323
Open source software(开源软件)
Amazon Web Services(Amazon Web服务),457
WSCs,437
Operands(操作数)
Forvarding(转发),C-19
instruction set encoding(指令集编码),A-21~A-22
ISA,12
ISA classification(ISA分类),A-3~A-
MIPS data types(MIPS 数据类型),A-34
MIPS pipeline(MIPS流水线),C-71
MIPS pipeline FP operations(MIPS 流水线浮点运算),
C-52~C-53
NVIDIA GPU ISA,298
per ALU instruction example(每ALU指令示例),4-6
type and size(类型与大小),A-13~A-14
vector execution time(向量执行时间),268~259
Operating systems (general)(操作系统(通用))
address translation(地址转换),B-38
architecture development(体系结构开发),2
memory protection performance(存储番保护性能),
B-58
miss statistics(缺失统计数字),B-59
multiprocessor software development(多处理器软件开
发),408
page size(页大小),B-58
segmented virtual merory(分段虚拟存储器),B-54
server benchmarks(服务基准测试),40
shared-memory workloads(共享存储器工作鱼载),
374~378
Operational costs(操作成本)
basic considerations(基本考虑事项),33
WSC434,438,452,456,472
Operational expenditres (OPEX)(运行费用)
WSC costs(WSC戚本),452~455,454
WSCTCO case siudy(WSCTCO案例研究),476-478
Oracle database(Oracle 数据库)
commercial workioad(商业工作负载),368
miss statistics(续失统计数字),B-59
multithreading benchraarks(多线程基准试),232
single-(hreaded benchmarks(单线程基准测试),243
WSC services(WSC服务),441
Organization(组织)
cache, performance impact (缓存,性能影响),B-19
cache blocks(缓存块),B-7~B-8
cache optimization(缓存优化),B-19
coherence extensions(一致性扩展),362
computer architecture(计算机体系结构),11,15~16
DRAM,98
索
引
575
•MIPS pipeline (MIPS流水线),C-37
multiple-issue processor(多发射处理器),197,198
Opteron data cache(Opteron 数据缓存),B-12~B-13,B-13
Pipelines(流水线),152
processor history(处理器历史),2~3
procesor petformence equation(处理器性能公式),49
shared-memory multijprocessors(共享存储器多处理
器),346
TLB,B-46
Orthogonality, compiler writing-architecture relationship
(正交性,编译器编写-体系结构关系),A-30
Out-of-order execution and cache miss(乱序执行与缓存峽
失),B-2~B-3
cache performance(缓存性能),B-21
diata hazards(数据冒险),169~170
bacdware-based ex ecution(基于硬件的执行),184
ILP,245
memory hierarchy(存储器层次结构),B-2~B-3
mnicroarchitectural tecbniques case study(微体系结构
技术案例研究),247-254
MIPS pipeline(MIIPS 流水线),C-71
mniss penalty(缺失代价),B-20~B-22
performance milestones(性能里程碑),20
power/DLP issues(功能/DLP 问题),322
processor comparisons(处理器对比),323
R10000, 397
SMT,246
Tomasulo's algorithm(Tomasulo算法),183
Out-of-order processors(乱序处理器)
DLP,322
Intel Core i7,236
multithreading(多线程),226
vector architecture(向量体系结构),267
Out-oforder write, dynamic scheduling(乱序写,动态调
度),171
Output dependence(输出相关),152~153
Overclocking(超频模式)
microprocessors(微处理器),26
processor performance equation(处理器性能公式),
52
Oversubscription(超额认购)
auray switch(阵列交換机),443
Google WSC,469
WSC architecture(WSC体系结构),441,461
P
Packed decimal, defioition(压缩十进制,定
Page coloring,definition(页面着色,定义),
Paged segments, characteristics(分页段,特性),B
A-14
576 索引
Paged virtual memory(分页虚拟存储器)
Opteron exatple(Opteron示例),B-54~B-57
protection(保护),106
segmnented(分段),F-43
Page faults(页面错误)
cache optimization(缓存优化),A-46
exceptions(异常),C-43~C-44
haxcdware-based specwlation(基于硬件的推测),188
mernory hierarchy(存储器层次结构),B-3
MIPS excsptions(MIPS异常),C48
Miultiredia SIMID Extensicns(多媒体 SIMD扩展),284
stopping/restarting execution(停止/重启执行),C-46
virtual memory definition(虚拟存储器定义),B-42
virual memory miss (虚拟存储锦峡失),B-45
Page offset(页偏移)
cache optimization(缓存优化),B-38
main memory(主存储器),B-44
TLB, B-46
virtual memory(虚拟存储器),B-43,B-49,B-$5~B-$6
Pages(页),B-3
Page size(页大小),B-56
Page Table Bntry (PTE)(页表项)
AMD64 paged virtual memory (AMD64分页成拟存储
器),B-56
IA-32 equivalent (IA-32 等价),B-52
Intel Core i7,120
msin memory block(主存储器块),B-44~B-45
paged virtual memory(分而虛拟存储器),B-56
TLB, B-47
Page tables(页表)
address translation(地址转換),B-46~B-47
AMD64 paged virtual memory (AMID64分页虚拟存储
器),B-5S~B-56
descriptor tables as(描述符表),B-52
IA-32 segment descriptors (IA-32 段描述符),B-53
main memory block(主存储器块),B-44~B-45
multiprocessor software development(多处理器软件开
发),407~409
multithreading(多线程),224
protection process(保护进程),B-50
segmented virtual memory(分段虚拟存储器),B-51
virbual memory block identification(虚拟存储器块识
别),B-44
virtual-to-physical address mapping(虚拟地址-物理地
址映射),B-45
Palt, definition(页错误,定义),B-3
Parallelism(并行)
cache optimization(缓存优化),79
challenges(挑战),349-351
classes(类),9~10
computer design principles(计算机设计原理),44~45
loop-level(循环级),149~150,215,217~218,315~322
MIPS scoreboarding(MIPS 记分卡),C-77~C-78
multiprocessors(多处理器),345
natural(自然),223,344
request-level(请求级),4~5,9, 345,434
RISC development (RISC开发),2
speedup(加速比),263
task-level(任务级),9
window size(窗口大小),217
WSCa vs. servers(WSC与服务器),433-434
Parallel programming(并行编程)
Paralle Thread Execution (PTX)(并行线程执行)
basic GPU thread instructions(基本GPU线程措令),
299
GPU conditional branching(GPU条件分支),300~303
GPUs Vs. vector architectures(GPU 对向量体系结构),
308
NVIDIA GPU ISA,298~300
NVIDIA GPU Memory structures (NVIDIA GPU 存储
器结构),305
Parallel Thread Execution (PTX)Instruction(并行线程执
行指令),292,309,313
Paravirhualization(部分虚拟化)
system call performance(系统调用性能),141
Xen VM, 111
Parity(奇偶校验)
fault detection(错误检测),$8
memory dependability(存储器可靠性),104~105
WSC memory(WSC存储器),473~474
PARSEC benchmarks (PARSEC 基准测试)
Intel Core i7, 401~405
SMT on superscalar processors(超标量处理器上的
SMT),230~232,231
speedup without SMT(没有 SMT 的加速比),403~404
Partial store order, relaxed consistency models(部分存储顺
序,宽松的一致性模型),395
Partitioning(划分)
Multimedia SIMD Extensions(多媒体 SIMD扩展),
282
virtual mnemory protection(虚拟存储器保护),B-50
WSC memory hierarcby(WSC存储器层次结构),445
Pascal programs(Pascal 程序)
compiler types and classes(编译器类型和类),A-28
PC,见 Program counter
PDP-11,2-10,2-17~2-19,2-56
Peak performance(蜂值性能)
DLP,322
fallacies(谬论),57~58
multiple lanes(多车道),273
multiprocessor scaled programs(多处理器缩放程序),58
Roofline model(屋顶轮廓线模型),287
vector architectures(向量体系结构),331
WSC operational costs(WSC运行成本),434
Perfect Chub benchmarks (Perfect Chub 基准测试)
vector architecture programming(向量体系结构编程),
281,281~282
Perfect processor,ILP hardware model(完美处理器,ILP
硬件模型),214~215,215
Performance, 另见 Peak performance
Personal mobile device (PMDD)(个人移动设备)
characteristics(特性),6
computer class(计算机类),5
embedded computers(嵌入式计算机),8~9
Flash memory(闪存),18
integrated circuit cost trcnds(集成电路成本趋勢),28
ISA performance and efficiency prediction (ISA 性能与
效率预测),241~243
memory bierarchy basics(存储器层次结构基础)。78
memory hierarchy design(存储器层次结构设计),72
power and energy(功耗与能量),25
processor comparison(处理器比较),242
Phase-ordering problem, compiler structure(阶段排序问题,
编译器结构),A-26
Physical addresses(物理地址)
address translation(地址变换),B-46
AMD Opteron data cache (AMD Opteron 数据缓存),
B-12~B-13
ARM Cortex-A8,115
directory-based cache coberence protocol basics(目录
式缓存一致性协议基础),382
main memory block(主存储器块),B-44
memory hierarcby(存储器层次结构),B-48~B-49
memory hierarcehy basics(存储器层次结构基础),
77~78
memory mapping(存储器映射),B-52
paged virtual memory(分页虚拟存储器),B-55~B-56
page table-based mapping(基于分页表的映射),B-45
safe calls(案例调用),B-54
segmented virtual memory(分段虚拟存储器),B-51
sharing/protection(共享/保护),B-52
translation(转换),B-36~B-39
virtual memory definition(虚拟存储器定义),B-42
Physical cache, definition(物理缓存,定义),B-36~B-37
Physical memory(物理存储器)
centralized shared-memory tultiprocessors(集中式共
享存储器多处理器),347
索引
577
directory-based cache coherence(目录式缓存一致性),
354
futbure GPU features(未来GPU特征),332
GPU conditional branching(GPU条件分支),303
main memory block(主存储器块),B-44
memory hierarchy basics(存储器层次绪构基础),
B-41~B-42
multiprocessors(多处理器),345
paged virtual memory(分页虚拟存储器),B-56
procesgor comparison(处理器比较),323
segmented virtual memory(分段虚拟存储器),B-51
unified(统一),333
Virtual Machines(虚拟机),110
Pipeline bubble, sta.ll as(流水线气泡,作为停顿),C-13
Pipeline cycles per instruction(每条指令的流水线周期)
basic equation(基本公式),148
ILP,149
processor performance calculations(处理器性能计算),
218~219
R4000 performance(R4000性能),C-68~C-69
Pipeline delays(流水线延迟),228
Pipeline interlock(流水线互锁)
data dependences(数据相关性),151
data hazards requiring stall(需要停顿的数据冒险),
C-20
MIPS R4000,C-65
MIPS vs. VMIPS (MIPS 与 VMIPS),268
Pipeline latcbes(流水线闩锁),C-35
Pipeline organization(流水线组织)
dependences(相关性),152
MIPS,C-37
Pipeline registers(流水线寄存器),C-35
Pipeline schedulig(流水线调度)
basie considerations(基本考虑事项),161~162
dynamic scheduling(动态调度),168~169
ILP exploitation(ILP开发),197
LP exposure(揭示 ILP),157~161
microarchitectural techniques case study (微体系结构技
术案例研究),247~254
MITPS R4000, C-64
Pipeline stall cycles(流水线停顿周期)
branch scheme performance(分支方案性能),C-25
pipeline perforrmance(流水线性能),C-12~C-13
Pipelining(流水化),C-2~C-3
Pipe segment, defioition(流水段,定义),C-3
Pipe stage(流水级),C-3
PMD,见 Personal mobile device
Point-to-point mnultiprocessor, example(点对点多处理器,
示例),413
578
索
引
Point-to-point networks(点对点网络)
directory-based coherence(目录式一致性),418
directory protocol(目录协议),421~422
SMP limitations (SMP 局限性),363~364
Portable computers(便携式计算机)
processor comparison(处理器对比),242
Position independence, control flow instruction addressing
mode(位置独立性,控制流指令寻址方式),A-17
Power(电力)
distribution for servers(服务器配电),490
distribution overview(配电概述),447
DLP,322
first-level caches(一级缓存),79~80
Google server benchmatks(Google 服务器基准测试),
439~441
Google WSC, 465~468
PMDs,6
real-world server considerations(现实服务器考虑事项),
52~55
WSC infrastructure(WSC 基础设施),447
WSC power modes (WSC功率模式),472
WSC resource allocation case study (WSC 资源分配案
例研究),478~479
WSC TCO case study (WSCTCO案例研究),476479
Power consuption,另见 Energy efficiency
Power distribution unit (PDU), WSC infrastructure(配电
单元,WSC基础设施),447
Power failure(电力故障)
exceptions(异),C-43~C-44,C-45
utilities(公用设施),435
WSCstorage(WSC存储),442
Power gating, transistors(电源闸控,最体管),26
Power modes, WSCs(功率模式,WSC),472
PowerPC
AltiVec multirmedia instruction compiler support( AltiVec
多媒体指令编译器支持),A-31
consistency model(一致性模型),395
TBM Blue Gene/L
multimedia compiler support(多媒体编译器支持),A-31
precise exceptions(精确异常),C-59
RISC architecture (RISC体系结构),A-2
RISC code size(RISC代码大小),A-23
Power-performance(功率 性能)
low-power servers(低功率服务器),477
servers(服务器),54
Power Supply Units (PSUs), efficiency ratings(电源单元
(PSU),额外效率),462
power utilization effectiveness (PUE)(电力利用效率
(PUE))
datacenter comparison(数据中心对比),451
Google WSC,468
Google WSC containersd Google WSC集装箱),464~465
WSC,450~452
WSCs vs. datacenters (WSC与数据中心),456
WSCserver energy eficiency(WSC服务器飴量效率),
462
Precise exceptions(精确昇常),C47
Predicate Registers(断言寄存器),309
Predicted-not-taken scheme(预测未选中方案)
branch penalty reduction(降低分支代价),C-22,
C-22~C-23
MIPS R4000 pipeline (MIPS R4000 流水线),C-64
Predictions,另见Mispredictions
Prefetching(预取)
integrated instruction fetch units(集成指令提取单元),
208
Intel Core i7,122,123~124
NVIDIA GPU Memory structures (NVIDIA GPU 存储器
结构),305
parallel processing challenges(并行处理挑战),351
Present bit, IA•32 descriptor table(存在位,IA-32描述符
表),B-52
Price vs.cost(价格与成本),32~33
Price-perfonnance ratio(性价比)
cost trends(成本趋勢),28
Dell PowerEdge servers(Dell PowerEdge 服务器),$3
desktop comptuers(台式计算机),6
processor comparisons(处理器对比),55
WSCs,8,441
Primitives(原语)
architeet-compiler writer relationsbip(架构师-编译器编
写人员关系),A-30
basic hardware types(基本硬件类型),387~389
compiler writing-architecture relationship(编译器编写
人员-架构师关系),A-30
CUDA Thread (CUDA线),289
dependent computation elirnination(消除相关计算),
321
GPU vs. MIMD (GPU 与 MIMD),329
locks via coberence(通过一致性锁定),391
operand types and sizee(操作数类型与大小),A-14~A-15
synchronization(同步),394
Principle oflocality(局域性原理),45,B-2
Private data(私有数据)
cache protocols(缓存协议),359
centralized shared-mnemory Imultiprocessors(集中共享
存储器多处理器),351-352
Private Memory(专用存储器),292,314
Private vatiables, NVIDLA GPU Memory(私有变量,NVIDIA
GPU存销器),304
Procedure calls(过程调用)
compiler structure(编译器结构),A-25~A-26
control flow instructions(控制流措令),A-17,
A-19~A-21
dependence analysis(相关性分析),321
high-level instruction set(高级指令集),A-42~A-43
invocation options(调用选项),A-19
ISAs,14
MIPS control flow instructions (MIPS 控制流指令),
A-38
return address predictors(返回地址预测器),206
VAX,B-73~B-74
VAX swap(VAX交換),B-74~B-75
Process concept(进概念),160,B-49
Process-identifier(PID)tags, cache addressing(进程识别
符标签,缓存寻址),B-37~B-38
Process IDs, Virtual Machines(进程1D,虚拟机),110
Processor consistency(处理器一致性)
latency hiding with speculation(用推测隐藏延迟),
telaxed consistency models(宽松连贯性模型),395
Processor cycles(处理器周期),C-3
Processor-derpendeat optimizations(处理器相关优化)
compiler(编译器),A-26
performance impact(性能影响),A-27
types(类型),A-28
Proces8or-intensive benchmarks, desktop performance(处理
器密集基准测试,桌台性能),38
Processor performance(处理器性能)
average memory access time(存储器平均访问时间),
B-17~B-20
cache performance(缓存性能),B-16
clock rate trends(时钟频率趋勢),24
desktop benchmarks(桌面基准测试),38,40
historical tends(历史趋势),3,3~4
multiprocessors(多处理器),347
uniprocessors(单处理器),344
Processor perforance equation, computer design principles
(处理器性能公式,计算机设计原理),48~$2
Processor speed(处理器速度)
clock rate(时钟频率)
CPI,244
snooping cache coherence(监听缓存一致性),364
Process switch(进开关),106,B-49
Productivity(生产能力)
CUDA,290~291
NVIDIA programmers (VINIDIA 程序员),289
索
引
579
software developmeot(软件开发),4
virtual memory and programing(虚拟存储器与编
程),B-41
WSC,450
Profile-based predictor, misprediction rate (基于一览数据的
预测器,错误预测率),C-27
Program counter (PC)(程序计数器)
addressing modes(寻址方式),A-10
ARM Cortex-A8,234
branch hazards(分支胃险),C-21
branch-target bufiers(分支目标缓冲区),203,
203~204,206
control flow instriction addressing mnodes(控制流指令
寻址方式),A-17
dynamic branch predictiond 动态分支预测),C-27~C-28
excepdion stoypinghestarting(异常停止/重启),C46-C-47
GPU conditional branching (GPU条件分支),303
Intel Core i7, 120
MIPS control flow instructions(MIPS 控制流指令),
A-38
multithreading(多线程),223~224
pipeline branch issues(流水线分支同題),C-39-C-41
pipe stages(流水級),C-35
precise exceptions(精确异常),C-59-C-60
RISC classic pipeline (RISC经典流水线),C-8
RISC instruction set(RISC指令集),C-5
simple MIIPS implementation(简单 MIPS笑现),
C-31~C-33
TLP, 344
virtual memory protection(虚拟存储器保护),106
Progratn counter-relative addnessing(程序计算机相对寻
址),A-10
Programmning primitive, CUDA Thread(编程原语,CUDA
线程),289
Program order(程序顺序)
cache coherence(缓存一致性),353
control dependences(控制相关性),154~155
data hazards(数据冒险),153
dytamic scheduling(动态调度),168~169,174
hardware-based speculation(基于硬件的推测),192
ILP exploitation (ILP开发),200
name dependences(名称相关),152~153
Tomasulo's approach(Tomasulo 方法),182
Protection schemes(保护机制)
control dependence(控制相关性),155
ISA,112
Peatium vs. Opteron (Pentium 与 Opteron),B-57
processes(进程),B-50
safe calls(安全调用),B-54
580
索引
segmented virtual memory example(分段虚拟存储器
示例),B-51~B-$4
Virtual Machines(虚拟机),107~108
virtual memory(肆拟存储器),105~107,B-41
Pseudo-Jeast recently used (LRU)(伪最近使用(LRU))
block replacement(块放置),B-9~B-10
Intel Core i7,118
PSUs,见 Power Supply Units
PTX,见 Parallel Thread Execution
PUE,见 Power utilization effectiveness
Python Language, hardware imnpact on sofwvare development
(Python 语言,硬件对软件开发的影响),4
Q
Quickpath (Intel Xeon),cache coherence (Intel Xeon),
Quickpatb,缓存
一致性,361
R
Race-to-halt, definition(竞相暂停,定义),26
Rack units (U),WSC architechure(机架单元,WSC体系
结构),441
RAID (Redundant auray of inexpensivye disks)(RAID(廉
价磁盘冗余阵列))
data replication(数据复制),439
memory dependability(存储器可靠性),104
WSC storage(WSC存储),442
Random replacement(随机替換),B-9
RAR,见 Read after read
RAS,见 Row access strobe
RAW,见Read after write
Ray casting (RC)(射线造型)
GPU comparisons(GPU 比较),329
throughput computing kernel(吞吐量计算内核),327
RDMA,凡 Remote direct memory access
Read after read (RAR), absence of data hazard(读后读,
没有数据冒险),154
Read after write(RAW)(写后读)
data hazards(数据臂险),153
dynamic scheduling with Tomnasulo's algorithm(用
Tomasulo算法进行动态调度),170~171
hazards, staills(冒险、停顿),C-55
hazards and forwarding(冒险与转发),C-5$~C-57
instruction set complications(指令集复杂性),C-50
microarchitecturl techniques case study(徽体系结构技
术案例研究),253
MIIPS FP pipeline performance(MIPS 深点流水线性
館),C-60~C-61
MIPS pipeline control (MIPS 流水线控制),C-37~C-38
MIPS pipeline FP opecations (MIPS 流水线深点运算),
C-$3
MIS scoreboarding (MIPS记分卡),C-74
ROB,192
Tomasulo's algorithm(Tomnasulo 算法),182
unoptimized code(非优化代码),C-81
Read miss(读取缺失)
AMD Ophearon data cache (AMI Optson 数据缓存),B-14
cache coherence(缓存一致性),357,358,359~361
'coherence extensions(一致性扩展),362
directory-based cache coherence(目录式緩存一致性)
Protacol example(协议示例),380,382~386
memory bierachy basice(存储器层次结构基础),76~77
memory stall clock cycles(存储器停顿时钟周期),B4
miss penalty reduction(缩小觖失代价),B-35~B-36
Opteron data cache (Opteron 数据缓存),B-14
wtite-through(直写),B-11
Read operands stage(读操作数级)
ID pipe stage(ID 流水級),170
MIIPS scoreboarding(MIIS 记分卡),C-74~C-75
out-of-order execution(乱序执行),C-71
Realizable processors, ILP limitations(可实现处理器,IP
局限性),216~220
Real memory, Virtual Machincs(实际存储器,虚拟机),110
Real-time performance, PMDs(实时性能,PMD),6
RECN,见 Regional explicit congestion notification
Reduced Jnstruction Set Computer, 见 RISC (Redluced
Instruction Set Computer)
Reductions(降低)
commercial workloads(商业工作负载),371
cost trends(成本趋勢),28
loop-level parallelism dependences( 循环级并行相关),
321
multiprogramming workloads(多重编程工作负载),
377
T1 moltithreading unicore performance(T1多线程单核
性能),227
WSCs, 438
Redundancy(冗余)
Amdabl's law (Amdahl 宛律),48
chip fbrication cost case study(芯片制造成本案例研
究),61~62
computer systero power consumption case stucy(计算
机系统功耗案例研究),63~64
index checks(索引检查),B-8
integrated circuit cost(集成电路成本),32
integrated circuit failure(集成电路故障),35
simple MIPS implementation(简单MIPS实现),C-33
WSC,433,435,439
WSC bottleneck(WSC瓶颈),461
WSC storage(WSC存储),442
Redundant power supplies, example calculations(冗余电源,
示例计算),35
Reference bit(引用位)
memory hierarcby(存储器层次结构),B-52
virtual memory block replacement(虛拟存储器块放
置),B-$4
Register addressing mode(寄存器寻址方式)
MIPS,12
Register allocation(寄存器分配)
compilers(编译器),396,A-26~A-29
Register definition(寄存器定义),314
Register fetch (RF)(寄存器提取)
MIPS data path(MIIPS 数据路径),C-34
MIPS R4000,C-63
pipeline branches(流水线分支),C-41
simple MIPS implementation(简单 MIPS实现),C.31
simgple RISC implementatiork简单RISC实现),C-S~C-6
Register file(寄存器堆)
data hazards(数据冒险),C-16,C-18,C-20
dynarnic scheduling(动态调度),172,173,175,
177~178
Fermi GPU,306
field(宇段),176
bardware-based speculation(基于硬件的推测),184
longer latency pipelines(长延退流水线),C-55~C-57
MuPS exceptions(MIPS 异常),C-49
MIPS implementation (MPS 实现),C-31,C-33
MIPS R4000(MIPS R4000),C-64
MIPS scoreboarding (MIPS 记分卡),C-75
Multimedia SIMID Extensions(多媒体 SIMID扩展),
282,285
multiple lanes(多车道),272,273
multithreading(多线程),224
precise exceptions(精确异常),C-59
RISC classic pipeline(RISC 经典流水线),C-7~C-8
RISC instruction set(RISC 指令集),C-5~C-6
scoreboarding(记分卡),C-73,C-75
speculation support(推测支持),208
structural hazards(结构性冒险),C-13
Tomasulo's algorithm(Tomasulo 算法),180,182
vector architechure(量体系结构),264
VMIPS,265,308
Register-memory instruction set architecture(寄存器-存储
器指令集体系结构)
architect-compiler writer relationship(架构师编译器编
索引
581
写人员关系),A-30
dynarnic scheduling(动态调度),171
ISA classitication(ISA分类),11,A-3~A-6
Register prefetch, cache optimization(寄存器预取,缓存优
化),92
Register renaming(寄存器重命名)
dyzamic scheduling(动态调度),169~172
hardware vs. software speculation(硬件与软件推测),
222
jdeal processor(理想处理器),214
ILP hardware model(ILP硬件模型),214
IL.P limitations(ILP局限性),213,216~217
ILP for realizable processors(可实现处理器的ILP),
216
instruction delivery and speculation( 指令提交与推测),
202
microarchitectural techniques case study (微体系结构
技术案例研究),247~254
name dependences(名称相关性),1$3
ROB,208~210
ROB instruction(ROB 指令),186
sample code(示例代码),250
SMT,225
speculation(推测),208~210
superscalar code(超标量代码),251
Tomasulo's algorithm(Tomasulo算法),183
WAW/WAR hazards(WAW/WAR冒险),220
Register result status, MIIPS scoreboard(寄存器结果状态,
MIPS记分卡),C-76
Registers(寄存群)
instructions and hazards(指令与冒险),C-17
pipe stages(流水級),C-3s
VAX swap(VAX交换),B-74~B-75
Register tag example(寄存器标签示例),177
Regularity(规则)
compiler writing-architecture relationship(编译器编写
与体系结构的关系),A-30
Relative speedup, multiprocessor performance(相对加速比,
多处理器性能),406
Relaxed consistency models(宽松连贯性模型)
basic considerations(基本考虑事项),394~395
compiler optimization(编译器优化),396
WSC storage softare(WSC 存储软件),439
Release consistency, relaxed consistency models(释放连贯
性,宽松连贯性模型),395
Reliability(可靠性)
Amdahl's law calculetions (Amdahl定律计算),56
example calculations(示例计算),48
modules, SL.As(模块,SLA),34
582
索
引
MTTF, 57
redundant power supplies(元余电源),34~35
transistor scaling(晶体管缩放),21
Relocation, virtual memory(重定位,虚拟存储器),B-42
Remote node, directory-based cache coherence protocol basics
(远程节点,目录式一致性协议基础),381~382
Reorder bufter(ROB)(重排序缓冲区)
dependent instructions(相关指令),199
dymnamic scheduling(动态调度),175
FP unit with Tomasulo's algorithm(采用 Tomasulo算法
的浮点单元),185
hardware-based speculatior( 基于硬件的推测),184~192
ILP exploitation (ILP 开发),199~200
TLP limitations(ILP 局限性),216
Intel Core i7,238
register renaming(寄存器重命名),208~210
Repeat interval, MIPS pipeline FP operations(重复间隔,
MIPS 流水线浮点运算),C-$2~C-53
Replication(复制)
cache coherent multiprocessors(缓存一致性多处理器),
354
centralized shared-memory architectures(集中式共享存
储器体系结构),351~352
coherence enforcement(一致性实施),354
R4000 perfomance (R4000 性能),C-70
RAID storage servers(RAID 存储服务器),439
TLP, 344
virtual memory(虚拟存储器),B-48~-B-9
WSCs,438
Reproducibility, perforaance results reporting(再生能
力,性能结果报告),41
Requested protection level, segmented virtual memory (所请
求的保护级别,分啓虚拟存储器),B-54
Request-level parailelism (RLP)(请求級并行),9
Reservation stations(保留站)
dependent instructions(相关指令),199~-200
dynamic scheduling(动态调度),178
example(示例),177
fields(字段),176
hardware-based speculation(基于硬件的推测),184,
186,189-191
ILP exploitation(ILP 开发),197,199~200
Intel Core i7,238-240
loopiteration example(迭代示例),181
microarchitechural techniques case study(微体系结构
技术案例研究),253~254
speculation(推测),208~209
Tomasulo's algorithm(Tomasulo 算法),172,173,
174~176, 179,180,180~182
Resource allocation(贺源分配)
coxputer desigo principies(计算机设计原理),45
WSC case study(WSC案例研究),478~479
Response time,另见 Latency
performance considerations(性能考忠事项),36
performance trends(性能趋势),18~19
server benehmarks(服务器基准测试),40~41
user experience(用户体验),4
WSCs, 450
Responsiveness(响应),C-45
Restorations, SLA states(恢复,SL.A状态),34
Resumne events(恢复事件)
control dependences(控制相关性),156
exceptions(异常),C-45~C-46
hardwate-based speculation(基于硬件的推测),188
Return address predictor(返回地址预测器)
instruction fetch bandwidth(指令提取带宽),206~207
prediction accuracy(预测精度),207
Returns(返回)
Amdahl'slaw(Amdahl定律),47
cache coherence(缓存一致性),352~353
compiler technology and architectural decisions(编译器
技术与体系结构决策),A-28
control flow instructions(控制流指令),14,A~17,A-21
hardware primitives(硬件原语),388
invocation options(调用选项),A-19
procedure invocation options(过程调用选项),A-19
retum address predictors(返回地址预测器),206
RF,见 Register fetch
Rings(环)
process protection(进程保护),B-50
RISC (Reduced Instruction Set Computer)(精简指令集计
算机)
architecture flaws vs.success(体系结构缺陷与成功),
A45
basic concept(基本概念),C-4~C-5
cache performance(缓存性能),B-6
classic pipeline stages(经典流水级),C-6-C-10
code size(代码大小),A-23~A-24
development(发展),2
ISA performance and efficiency prediction (ISA 性能与
效率预测),241
ROB,见 Reorder buffer
Roofline model(屋顶轮廓线模型)
GPU performance(GPU性能),326
memory bandwidth(存储髒带宽),332
Multirnedia SIMD Extensions(多媒体 SID扩展),
285~288,287
Row access strobe(RAS),DRAM(行访问选通),98
Row major order, blocking(行主序,分块),89
Ruby on Rails, hardware impact ot software development
(Ruby on Rails,硬件对软件开发的影响),4
S
S3,见 Amazon Simple Storage Service
SaaS, 见 Software as a Service
Sandy Bridge dies, wafter example (Sandy Bridge 晶片,晶
圆示例),31
SATA (Serial Advanced Technology Attachment) disks(串
行高级技术连接)磁盘)
Google WSC servers (Google WSC服务器),469
SAXPY, GPU raw/relative perforance(SAXPY,GPU 原
始/相对性能),328
Scalability(可扩展性)
cloud computing(云计算),460
coherence issues(一致性问题),378~379
Fermi GPU,295
Java benchmarks(Java 基准测试),402
multicore processor(多核处理器),400
multiprocessing(多重处理),344,395
parallelism(并行度),44
server characteristic(服务器特性),7
transistor perfornance and wires(晶体管性能与连线),
19~21
WSCs, 8, 438
WSCs vs. servers(WSC与服务器),434
Scalar expansion, loop-level parallelism dependences(标量
扩展,循环级并行相关性),321
Scalar Processors,另见 Superscalar processors
Scalar registers(标量寄存器)
GPUsvs. vector architectures(GPU 与向量体系结构),
311
1oop-level parallelism dependences(循环级并行度相
关性),321~322
Multimedia SIMID vs. GPUs(多媒体 SIMD与 GPU),
312
sample reraming code(示例重命名代码),251
vector vs.GPU(向量与GPU),311
vector performance(向量性能),331~332
VMIPS,265~266
Scaled speedup, Amdahl's law and parallel corputers(缩放
加速,Amdabl定律与并行计算机),406-407
Scaling(缩放)
Amdabl's law and parallel computers(Amdahl 定律和
并行计算机),406~407
cloud computing(五汁算),456
DVFS, 25, $2,467
索
引
583
dynamic voltage-fequency(动态电压-频率),25,
$2,467
Intel Core i7, 404
multicore vs. single-core(多核与单核),402
processor perfornance trends(处理器性能趋势),3
transistor performnance and wires(晶体管性能与连线),
19~21
VMIPS,267
SCCC,见 Intel Single-Chip Cloud Computing
Scoreboarding(记分卡)
ARM Cotex-A8,233,234
comaponents(分量),C-76
definition(定义),170
dynamie scheduling(动态调度),171,175
example calculations(示例计算),C-77
MIPS structure (MIPS 结构),C-73
NVIDIA GPU,296
results tables(结果表),C-78~C-79
SIMID thread scbeduler(SIMID线程调度程序),296
Seripting languages, software development impact(脚本语
言,软件开发影响),4
SDRAM,见 Synchronous dynamic random-access tnemory
Second-level caches,另见 L2 caches
Segment descriptor, IA-32 processor(分段描述符,IA-32
处理器),B-52,B-$3
Segmented virtual memory(分段虛拟存储器)
bounds checking(界限检查),B-52
Intel Pentium protection( InteIPentium 保护),
B-51~B-54
memory mapping(存储器映射),B-52
paged(分页),B-43
safe calls(安全调用),B-54
sharing and protection(共享与保护),B-52~B-53
Semantic clash, high-level instraction set(语义冲突,高缎
指令集),A-41
Sernantic gap, high-level instruction set(语义鸿沟,高级指
令集),A-39
Semiconductors(半导体)
DRAM technology(DRAM技术),17
Flash mnemory(闪存),18
GPU vs. MIMD (GPU 与 MIMD),325
manufacturing(制造),3~4
Sequence of SIMD Lane Operations, definition (SIMD 车道
操作序列,定义),292,313
Sequential consistency(顺序一致性)
latency hiding with speculation(通过推测隐藏延退),
396~397
programmer's viewpoint(程序员的观点),394
relaxed consistency models(宽松连贯性模型),
584
索
引
394~395
reguirements and implementation(需求与实现),
392-393
Sequential interleaving, multibanked caches(顺序交错,多
组缓存),86
Serial Advanced Technology Attachment disks, 见 SATA
(Serial Adyanced Technology Attachment)disks
Serialization barier synchronization(序列化屏障同步)
coherence enforcement(一致性实施),354
directory-based cache coherence(目录式缓存一致性),
382
hardware primitives(硬件原语),387
multiprocessor cache coherency(多处理器缓存一致
性),353
page tables(页表),408
smooping coherenice protocols(监听一致性协议),356
write invalidate protocol implementation(写失效协议
实现),356
Servers,另见 Warehouse-scale computers (WSCs)
Server side Java operations per second (ssj_ops)(每秒执
行的服务器端 Java 运算(ssi_ops))
example calculations(示例计算),439
power-performance(功率-性能),54
real-worid considerations(现实考虑事项),$2~55
Service accomplishment, SLAs(服务完成,SLA),34
Service Hlealth Dashboard, AWS(服务健康仪表板,AWS),
457
Service internuption, SLAs(服务中断,SLA),34
Service level agreements (SL.As)
(股务级别协议)
Amazon Web Services (Amazon Web 服务),457
dependability(可靠性),33
wSC efficieney(WSC效率),452
Service level objectives (SLOs)(服务级目标)
dependability(可靠性),33
WSC effciency(WSC效率),452
Set associativity(集相联度)
access time(访问时间),77
address parts(地址部分),B-9
AMD Opteron data cache (AMID Operon 数据缓存),
B-12~B-14
ARM Cortex-A8,114
block placement(块放置),B-7~B-8
cache block(缓存块),B-7
cache misses(缓存缺失),83~84,B-10
cache optimization(缓存优化),79~80,B-33~B-35,
B-38~B-40
commercial workdoad(商业工作负载),371
energy consumption(能耗),81
memory access times(存储器访问时间),77
memory hierarchy basics(存储器层次结构基础),74,
76
nonblocking cache(非阻塞缓存),84
performance equations(性能公式),B-22
pipelined cache access(流水化缓存访问),82
way prediction(路预测),81
Set basics(集基础),B-7
SFF,见 Small form factor SFF disk
SGI,见 Silicon Graphics systems
Shadow page taible, Virtual Machines(影子页表),110
Sharding, WSC memory hierarchy(分片,WSC存储器房
次结构),445
Shared Memory(共享存储器),292,314
Shared-memory multiprocessors(共享存储器多处理器),
346-351
Shared state(共享状态)
cache block(缓存块),357,359
cache coherence(缓存一致性),360
cache miss calculations(缓存缺失计算),366~367
cohereace cxtensions(一致性扩展),362
directory-based cache coherence protocol basics(目标式
缓存一致性协议基础),380,385
private cache(专用缓存),358
Sharing addition, seggmented virtual memory(共享加法,分
段虛拟存储器),B-52~B-$3
Short-circuiting,见 Forwarding
Sign-extended ofiset, RISC(符号扩展偏移量,RISC),
C-4~C-5
Silicon Graphics systems (SGI)(硅图系统)
economies of scale(规模经济),456
miss statistics(缺失统计数字),B-59
multiprocessor sofhware development(多处理器软件
开发),407~409
SIMD (Single Instruction Strear, Multiple Data Stream)(单
推令流,多数据流),10
SIMD Instruction (SIMD 指令),292,313
SIVD Lane Registers, definition (SIMD 车道寄存器,定
义),309,314
SIIMID Lanes(SIMD 车道),292,296,309
SIMID ProcesSOrs, 另见 Multitbreaded SIMID Proce8sor
SIMD Thread (SIMID线程)
GPU conditional branching(GPU条件分支),301~302
Grid mapping(网络映射),293
Multithreaded SIMID processor (多线程 SIMID处理器),
294
NVIDIA GPU, 296
NVIDIA GPU ISA, 298
NVIDIA GPU Memory structures (NVIDIA GPU 存備
器结构),305
scheduling example(调度示例),297
vector vs.GPU(向量与 GPU),308
vector processor(向量处理器),310
SIM.D Thread Scheduler(SIMID线程调度程序),292,
314
SIMT(Single Instruction, Multiple Thread)(单指令,多
线程)
GPU programming(GPU编程),289
SIMD, 314
Warp(交換),313
Simultaneous multithreading(SMT)(同时多线程),
224~225
Single Iostruction, Multiple Thrcad,见 SIMT(Single Instru-
ction, Multiple Thread)
Single Instruction Stream, Single Data Stream, 见 SISD
(Single Instruction Stream, Single Data Stream)
Single-level cache hierarchy, miss rates vs. cache size(单级
缓存层次结构,觖失率与缓存大小),B-33
Single-precision floating point(单精度浮点)
GPU examples(GPU示例),325
GPU vs. MIMD(GPU 与 MIMID),328
MIIPS data types (MIIPS 数据类型),A-34
MIIPS operations(MIPS 操作),A-36
Multimedia SIMID Extensions(多媒体 SIMD扩展),
operand sizes/types(操作数大小/类型),12,A-13
operand type(操作数类型),A-13~A-14
Single-thread (ST)performance(单线程性飴)
IBM eServerP5575,399,399
Intel Core i7, 239
ISA,242
processor comparison(处理器对比),243
SLAs,见 Service level agreements
SLOs,见 Service level objectives
SM,见 Distributed shared memory (DSM)
Small Computer System Interface,见 SCSI
Smartphones(智能手机)
ARM Cortex-A8(ARM Cortex-A8),114
mobile vs. server GPUs(移动与服器 GPU),323~324
SMP,见 Symmetric multiprocessors
SMT,见 Siroultaneous moltithreading
Snooping cache coherence(监听缓存一致性),354~355
Soft enors,definition(软错误,定义),104
Softvare as a Service (SaaS)(软件即服务)
clusters/WSCs(集群/WSC),88
sofitware development(软件开发),4
WSCs,438
WSCs vs.servers(WSC与服务器),433~434
Software development multiprocessor architecture issues(软
索
585
件开发多处理器体系结构问题),407~409
Softwaze prefetching, cache optimization(软件预取,缓存
优化),131~133
Software speculation(软件推测),156
Software technology(软件技术)
ILP approaches(ILP 方法),148
Virtual Machines protection(脆拟机保护),108
WSC runningservice(WSC运行服务),434~435
Solid-state disks(SSDs)(周态硬盘)
processor performance/price/power(处理器性能/价格/
功率),52
server energy efficiency(服务器能量效率),462
WSC cost-performance(WSC成本一性能),474~475
Sortprimitive, GPU vs.MIMD(排序原语,GPU 与MIMD),
329
Sparse matrices(稀疏矩阵)
loop-level parallelism dependences(循环级并行相关
性),318~319
vector architectures(向量体系结构),279-280
vector execution tirne(向量执行时间),271
vector mask registers(向量遮單寄存器),275
Spatial locality(空间局域性),45,B-2
SPEC benchmarks(SPEC 基准测试)
branch predictor correlation(分支预测器相关),
162~164
desktop performance(桌面性能),38~40
evolution(演进),39
fallacies(谬论),56
operands(操作数),A-14
performance(性能),38
performance results reportiog(性能结果报告),41
processor performance growth(处理器性能增长),3
static branch prediction(静态分支预测),C-26-C-27
tournament predictors(竞赛预测器),164
two-bit predictors(两位预测器),165
SPEC89 benchrarks(SPEC89基准测试)
branch-porediction buffiers(分支预測缓冲区),C-28~C-30,
C-30
MIPS FP pipeline performance(MIPS 浮点流水线性
能),C-61~C-62
misprediction rates(错误预测率),166
tournament predictors(竞赛预测器),165~166
SPEC92 benchmarks(SPEC92基准测试)
hardware vs. softwvare speculation(硬件推测与软件推
测),221
ILP hardware model(ILP硬件模型),215
MIPS R4000 performance(MIPS R4000 性能),
C-68~C-69, C-69
misprediction rate(错误预测率),C-27
586
索
引
SPEC95 bemchmarks(SPEC95 基准测试)
retur address predictors(返回地址预测器),206~207,
207
way prediction(略预测),82
SPEC2000 benchmarks(SPEC2000基准测试)
ARM Cortex-A8 memory (ARM Cortex-A8 存储器),
115~116
cache performance prediction(缓存性能预测),125~126
cache size and rnisses per instruction(缓存大小与每条
指令的缺失率),126
compiler optimizations(编译器优化),A-79
compuksory miss rate(强制缺失率),B-23
data reference sizes (数据引用大小),A-44
hardware prefetching(硬件预取),91
instruction misses(指令映失),127
SPEC2006 benchmarks(SPEC2006 基准测试)
evolution(演进),39
SPECCPU2000 benchmarks(SPECCPU2000基准测试)
displacement addressing mode(放置寻址方式),A-12
Intel Core i7,122
server benchmarks(服务器基准测试),40
SPECCPU2006 benchmarks(SPECCPU2006 基准测试)
branch predictors(分支预测器),167
Intel Core i7, 123~124, 240,240~241
ISA performance and efficiency prediction (ISA 性能与
效率預测),241
Virtual Machines protection(虚拟机保护),108
SPECfp benchmarks (SPEC印 基准测试)
hardware prefetching(硬件预取),91
ISA performance and cfficiency prediction(ISA 性能和
效率预测),241~242
MIPS FP pipeline perfornance(MIPS浮点流水线性能),
C-60~C-61
nonblocking caches(非阻塞缓存),84
tournament predictors(竞赛預测器),164
SPECfp92 benchmarks(SPECfp92基准测试)
nonblocking cache(非阻塞缓存),83
SPECp2000 benchmarks(SPECfp2000 基准测试)
hardware prefetching(硬件预取),92
MIPS dynamic instruction mix(MIPs 动态指令混合比),
A42
Sun Ultra S execution tirnes (Sun Ultra S执行时间),43
SPECtp2006 benchmarks(SPECfp2006 基准测试)
Intel processor clock rates(Intel 处理器时钟频率),
244
nonblocking cache(非塞缓存),83
SPECfpRate benchmarks(SPECfpRate 基准测试)
mnulticore processor performance(多核处理器性能),
400
multiprocessor cost effectiveness(多处理器成本效率),
407
SMT,398~400
SMT on superscalar processors(超标处理器上的
SMT),230
Special-purpose register(专用寄存器)
compiler writing-architecture relationship(编译器编写-
体系结构关系),A-30
ISA classification(ISA分类),A-3
VMIPS, 267
SPECINT benchmarks(SPECINT 基准测试)
hardware prefetching(硬件预取),92
ISA per formance and efficiency prediction (ISA 性能与
效率预测),241~242
nonblocking caches(非阻塞缓存),84
nonblocking cache(非阻塞缓存),83
SPECINT2000 benchmarks, MIPS dynamic instruction mix
(SPECINT2000基准测试,MIIPS 动态指令混合比),
A-41
SPECINT2006 benchmarks(SPECINT2006 基准测试)
Intel processor clock rates(Intel 处理器时钟频率),
244
nonblocking cache(非阻塞級存),83
SPECinRate benchrnark(SPECintRate 基准测试)
mnulticore processor performance(多核处理器性飴),
400
multiprocessor cost effectiveness(多处理器成本效率),
407
SMT,398~400
SMT on superscalar processors(超标量处理器上的
SMTT),230
SPEC Java Business Benchmark (JBB)(SPEC Java 业务
基准测试)
mmulticore processor performance(多核处理器性能),
400
molticore processors(多核处理器),402
multiprocessing/ multithreading-based perforance(基
于多重处理/多线程的性能),398
server(服务器),40
Sun T1 multithreading unicore performance(Sun T1 多
线程单核性館),227~229,229
SPECJVM98 benchmarks, ISA performance and efficiency
prediction(SPECJVM98 基准测试,ISA 性能与效率
预测),241
SPECPower benchmarks(SPECPower基准测试)
Google server benchmarks(Google 服务器基准测试),
439~440, 440
multicore processor performance(多核处理器性能),
400
real-world server considerations(现实服务器考虑事项),
$2~55
WSCs,463
WSC server energy efficiency (WSC服务器能量效率),
462~463
SPECRate benchmarks(SPECRate 基准测试)
Intel Core i7, 402
multicore processor pertformance(多核处理器性能),
400
multiprocessor cost effectiveress(多处理器戚本效率),
407
server benchmarks(服务器基准测试),40
SPECRate2000 benchmarks(SPECRate2000基准测试),
398~400
SPECRatios
execution time examples(执行时间示例),43
geometric means calculations(几何均值计算),43~44
SPECSFS benchmarks (SPECSFS 基准測试)
servers(服务器),40
Speculation,另见 Hardware-based speculation:
Software
speculation SPECvirt_Sc2010 benchroarks, server (SP-
ECvirt_Sc2010 基准测试,服务器),40
SPECWeb benchmarks(SPECWeb 基准测试)
parallelism(并行度),44
server benchmarks(服务器基准测试),40
SPECWeb99 benchrnarks(SPECWeb99 基准测试)
multiprocessing/ multithreading-based performance(基
于多重处理/多线程的性飴),398
Sun Tl multithreading unicore performance (Sun T1 多
线程单核性能),227,229
Speedup(加速比)
Amdahl's law(Amdahl 定律),46-47
linear(线性),405~-407
via parallelism(通过并行度),263
pipeline with stalls(具有停顿的流水线),C-12-C-13
relative(相对),406
scaled(縮放),406~407
true(真),406
Spin locks(自旋锁)
via coherence(通过一致性),389~390
SPLASH parallel bsnchrarks, SMT on superscalar proce-
ssors(SPLASH并行基准测试,超标量处理器上的
SMT),230
Split, GPU vs.MIMID(分割,GPU与MIMID),329
SRAM,见 Static random-access memory(SRAM)
SSDs,见 Solid-state disks
SSE,见 Lntel Streaming SIMD Extension
ssij_ops, 见 Server side Java operations per second
Stack architecture(栈体系结构)
索引
587
compiler technology(编译器技术),A-27
flaws vs. success(缺点与成功),A-44~A-45
operanda(操作数),A-3~A-4
Steck or Thread Local Storage, definition(找或线程本地存
储,定义),292
Stale copy, cache cohereney(旧副本,缓存一致性),112
Stall cycles(停顿周期),B-4~B-5
Stalls(停顿)
AMID Opteron data cache(AMID Opleron 数据缓存),
B-15
ARM Cortex-A8,235,235~236
branch hazards(分支冒险),C-42
data hazard minimization(数据冒险最小化),
C-16~C-19,C-18
data hazards requiring(数据冒险需求),C-19~C-21
delayed branch(延迟分支),C-65
Intel Core i7, 239-241
microarchitectural techniques case study(微体系结构技
术案例研究),252
MIPS FP pipeline performance(MIPS 浮点流水线性
能),C-60~C-61,C-61~C-62
MIPS pipeline multicycle operations (MIPS 流水线多間
期操作),C-51
MIPS R4000, C-64, C-67, C-67~C-69,C-69
miss rate calculations(缺失率计算),B-31~B-32
necessity(必要性),C-21
nonblocking cache(非阻塞缓存),84
pipeline performanse(流水线性館),C-12~C-13
fomRAW hazards, FP code(自RAW冒险,浮点代码),
C-55
structural hazatd(结构性冒险),C-15
VLIW sample code(VLIW示例代码),252
VMIPS,268
Stardent-1500, Livermore Fortran kerels (Stardent-1500,
Livermore Fortran 内核),331
Start-up overhead, vs.peak performance(启动开销,与峰
值性能),331
Start-up time(启动时间)
memory banks(存储器组),276
page size selection(页大小选择),B-47
peak performance(峰值性能),331
vector architectures(向量体系结构),331
vector execution time(向量执行时间),270~271
State transition diagram(状态转换圈)
director vs. cache(目录与缓存),385
ditectory-based cache coherence(目录式缓存一致性),
383
Static power(静态功率)
588
索
引
basic equation(静态公式),26
SMT,231
Static random-access memory(SRAM)(静态随机访问
存储器(SRAM))
characteristics(特性),97~-98
dependability(可靠性),104
tault detection pitfalls(错误检测易犯错误),58
power(功率),26
yield(正品率),32
Static scheduling(静态调度),C-71
Storage systems(存储系统)
wSC vs. datacenter costs(WSC 与数据中心威本),
455
WSG, 442~443
Store conditional(存储条件)
locks via coherence(通过一致性锁定),391
synchronization(同步),388-389
Store instructions,另见 Load-store instuction set architecture
Streaming Multiprocessor(流式多处理器)292,313~314
Strided accesses(步幅访问)
Multimsedlia SIMD Extensions(多媒体 SIMID扩展),283
Roofline model(屋顶轮廓线模型),287
TLB interaction(TLB 交互),323
trides(步幅)
gather-scatter(集中-分散),280
highly parallel memory systems(高度并行存储器系统),
133
multidimensional arays in vector architectures(向量体
系结构中的多维数组),278-279
NVIDIA GPU ISA, 300
VMIPS, 266
Strip mining(条带挖掘)
GPU conditional branching(GPU 条件分支),303
GPUsvs.vector architectures (GPU 与向量体系结构),
311
NVIDLA GPU,291
vector(向量),275
VLRS, 274~275
Strong scaling, Amdahl's law and parallel computers(强扩
展,Amdall 定律和并行计算机),407
Structural hazards(结构冒险),C-11
Structural stalls, MIPS R4000pipeline(结构性停顿,MIPS
R4000流水线),C-68~C-69
Subset property,and inclusion(于集特性与包含性),397
Sun Microsystems
cache optimization(缓存优化),B-38
tault detection pitfalls(错误检测易犯错误),58
memory dependability(存储器可靠性),104
Sum Micosyseans Niagaua (T1/T2) procesons(Sun Micnosysens
Niagara(T1/T2)处理器)
characteristics(特性),227
CPI and IPC(CPI和 IPC),399
fine-grained multithreading(细精度多线程),224,
225,226~229
manufacturing cost(制造成本),62
multicore processor perfonnance(多核处理器性能),
400~401
multiprocessing/ multithreading-based perforance(基
于多重处理/多线程的性飴),398~400
T1 multithreading unicore performance (T1 多线程单核
性能),227~229
Sun Microsystems SPARC
ALU operands(ALU操作数),A-6
branch conditions(分支条件),A-19
ISA,A-2
precise exceptions(精确异常),C-60
Sun Microsystems Ultra 5, SPECfp2000 execution times(Sun
Microsystems Ultra S, SPECfp2000执行时间),43
Supercomputers(超级计算机)
WSCs,8
Superlinear performance, multiprocessors(超线性性能,多
处理器),406
Superpipelining(超级流水线),C-61
Superscalar processors(超级处理器)
ideal processors(理想处理器),214~215
ILP,192~197,246
microarchitechural techniques case study(微体系结构
技术案例研究),250~251
mmaltithreading support(多线程支持),225
register renaming code(寄存器重命名代码),251
rename table and register substitution logic(重命名步
及寄存器替换逻辑),251
SMT,230~232
VMIPS, 267
Superscalar registers(超标量寄存器)
sample renaming code(示例重命名代码),251
Supervisor process, virtual memory protection(管理员进程,
逮拟存储器保护),106
SVM,见 Secure Virtual Machine
Swap procedure, VAX(交换进程,VAX)
register preservation(寄存器保留),B-74~B-75
Swim, data cache misses(Swim,数据缓存敏失),B-10
Switches(交換机)
array,WSCs(阵列,WSC),443~444
context(上下文),307,B-49
Ethemnet switches(以太网交換机),16,20,53,
441~444, 464~465,459
processswitch(进程交换),224,B-37,B-49-B-50
WSC hierarcby(WSC层次结构),441~442,442
WSC infastructure(WSC 基础改施),446
WSC network bottleneck(WSC网络瓶颈),461
Switch statements(Switch 语句)
control flow instruction addressing modes(控制流指令
寻址方式),A-18
GPU,301
Symmetric multiprocessors (SMP)(对称多处理器(SMP))
communication calculations(通信计算),350
directory-based cache coherence(目录式缓存一致性),
354
limitations(局限性),363~364
snooping coherence protocols(监听一致性协议),
354~355
TLP,345
Symmetrie shared-mnemory multiprocessors. 另见 Centralized
shared-memory multiprocessors
Synchronization(同步),375
Synchronous dynamnic random-access mnemory (SDRAM)
(同步动态随机访问存储器)
ARM Cortex-A8, 117
DRAM,99
vs.Flash memory(闪存),103
Intel Core i7,121
performance(性能),100
power consumption(功率消耗),102,103
SDRAM timing diagram(SDRAM 时序图),139
Syacbronous event, exception requirements(同步事件,
昇常需求),C-44~C-45
Synonyms(同义词)
address translation(地址转换),B-38
depeadability(可靠性),34
Synthetic benchmarks(合成基准测试),37
System calls(系统调用)
CUDA Thread(CUDA线程),297
mulliprogrammed workload(多重编程工作负载),
378
virtualization/paravirtualization performance(虚拟化/
部分虛拟化性能),141
virtual memory protection(虚拟存储器保护),106
System Performance and Evahuation Cooperative,见 $PEC
benchmarks
System Processor(系统处理器),309
Systems on a chip (SOC), cost trends(片上系统,成本
趋势),28
System Virtual Machines, definition(系统虚拟机,定义),
107
索
引
589
T
T88(标签)
AMD Opteron data cache (AMID)Opteron 数据缓存,
B-12~B-14
ARM Cortex-A8,115
cache optimization(缓存优化),79~80
dynamic scheduling(动态调度),177
invalidate protocols(失效协议),357
memory hierarchy basics(存储器层次结构基础),74
virtual memory fast address translation(虚拟存储器快
速地址转换),B-46
write strategy(写人策略),B-10
Tag check (TC)(标签核对)
MIPS R4000, C-63
R4000 pipeline(R4000流水线),B-62~B-63
R4000 pipeline structire(R4000流水线结构),C-63
write process(写入过程),B-10
Tag fields(标签字段)
block identification(块识别),B-8
dynamic scheduling(动态调度),173,175
Taxget address(目标地址)
branch hazards(分支冒险),C-21,C-42
branch penalty reduction(降低分支代价),C-22-C-23
branch-target buffer(分支目标缓冲区),206
control flow instructions(控制流指令),A-17~A-18
GPU conditional branching(GPU条件分支),301
Iatel Core i7 branch predictor (Intel Core i7 分支预测器),
166
MIPS control flow instructions (MIIPS 控制流指令),
A-38
MIPS implementation(MIPS 实现),C-32
MIPS pipeline(MIPS 流水线),C-36,C-37
MIPS R4000(MIPS R4000),C-25
pipeline branches(流水线分支),C-39
RISC instruction set(RISC 指令集),C-S
Target instructions(目标指令)
branch delay slot schedubing(分支延退时隙调度),
C-24
branch-target buffer variation(分支目标缓冲区变化),
206
GPUconditional branching(条件分支),301
Task-level parallelism(TLP)definition( 任务级并行定义),
9
TB,见 Transkation buffer
TC,见 Tag check
TCO,见 Total Cost of Ownership
TDP,见 Thermal design power
Technology trends(技术趋势)
590
索
引
basic considerations(基本考虑事项)17~18
performance(性館)18~19
Temporal locality(时间局域性),45,B-2
Terminate events(终止事件)
exceptions(异常),C-45~C-46
hardware-based speculation(基于硬件的推测),188
loop unrolling(循环展开),161
Test-and-set operation, synchronization (“測试并置位”操
作,同步),388
Texas Instruments ASC(德州仪器ASC)
peak pertformance vs. start-up overhead(峰值性能与后
动开销),331
Thermal design power (TDP),power trends(散热设计功
率,功率趋势),22
Third-level caches,另见 L3 cachesILP,245
ILP,245
SRAM,98~99
Thrash, memory hierarcby(摆动,存储器层次结构),B-25
Thread Block(线块),292,313
Thread Block Scheduler(线程块调度程序)292,309,
313~314
Thread-level parallelism (TLP)(线程级并行),9
Thread Processor(线程处理器),292,314
Thread Processor Registers,definition(线处理器寄存器,
定义),292
Thread Scheduler in a Multithreaded CPU, definition (多线
程 CPU 中的线程调度程序,定义),292
Thread of SIMID Instructions (SIMD指令线程),292,313
Thread of Vector Instructions, definition(向量指令线程,
定义),292
Three-level cache hierarchy(三级缓存层次结构)
commercial workloads(商业工作负载),368
ILP,245
Intel Core i7,118,118
Throughput,另见 Bandwidth
Ticks(嘀嗒)
cache coherence(缓存一致性),391
processor performance eqjuation(处理器性能公式),
48~49
Time -cost relacionship, components(时间-成本关系,组件),
27~28
TLB,见 Translation lookaside buffer
TLP,见 Task-level parallelism
Tomasulo's algorithm(Tomasulo 算法)
advantages(优势),177-178
dynamic scheduling(动态调度),170~176
FP unit(浮点单元),185
10op-based example(基于循环的示例),179,181~183
MIP FP unit(MIP 浮点单元),173
register renaming vs. ROB(寄存器重命名与ROB),
209
step details(步骤细节),178,180
rop OfStack (TOS) register, ISA operands(栈顶寄存器,
ISA 操作数),A-4
TOS,见 Top Of Stack register
Total Cost of Ownership (TCO),WSC case study (总拥
有成本,WSC案例研究),476~479
Total store ordering, relaxed consistency models(总存储排
序,宽松连贯性模型),395
Touraament predictors(竞赛预测器)
IL.P for realizable processors(可实现处理器的 ELP),
216
local/global predictor combinatioms(局部/全局预测器组
合),164~166
Toy programs, performance benchmarks(玩具程序,性能
基准测试),37
TP,见 Transaction-processing
TPC,见 Transaction Processing Council
Tradebeans benchmark, SMT on superscalar processors
(Tradebens 基准测试,超标量处理器上的SMT),230
Transaction-processitg (TP)(事务处理)
server benchmarks(服务器基准测试),41
Transaction Processing Council (TPC)(事务处理委员会)
parallelism(并行度),44
performance results reporting(性能结果报告),41
server benchmarks(服务器基准测试),41
TPC-B,shared-memory workloads(共享存储器工作负载),
368
TPC-C
IBM eServer p5 processor (IBM eServer p5 处理器),
409
multiprocessing/ multithreading-based performance(基
于多重处理/多线程的性能),398
multiprocessor cost effectiveness(多处理器成本效率),
407
single vs.multiple thread executions(单线程与多线程执
行),228
Sun T1 multithreading unicore performance (Sun T1 多
线程单核性能),227~229,229
WSC services(WSC服务),441
TPC-D, shared-memory workloads(共享存储器工作负载),
368~369
TPC-E, shared-merory workloads(共享存储器工作负载),
368~369
Transfers,另见 Data transfers
Transistors(晶体管)
clock rate considerations(时钟频率考虑事项),244
dependability(可靠性),33~-36
energy and power(能量与功率),23~26
ILP,245
performance scaling(性能扩展),19-21
processor comparisons(处理器对比),324
processor tends(处理器趋势),2
RISC instructions (RISC指令),A-3
shrinking(收缩),SS
static power(静态功率),26
technology trends(技术趋势),17~18
Translation buffer(TB)(转换缓冲区)
virtual memory block identification(虚拟块识别),B-45
virtual memory fast address translation(虚拟存储器快速
地址转换),B-46
Translation lookaside buffer (TLB)(转换旁视缓冲区)
address ttanslation(地址转换),B-39
AMID64 paged virtual memory (AMD64 分页虚拟存储
器),B-56~B-57
ARM Cortex-A8, 114~115
cache optimization(缀存优化),80,B-37
Intel Core i7 (Intel Core i7),118,120~121
memory hierarchy(存储器层次结构),B-48-B-49
memory hierarchy basics(存储器层次结构基础),78
Opteron,B-47
Opteron mernory hierarcby ( Opteron 存储器层次结构),
B-57
RISC code size(RISC 代码大小),A-23
shared-memory workloads(共享存储器工作魚载)
369~370
speculation advantages/ disadvantages(推测优缺点)
210~211
strided access interactions(步幅访何交互),323
Virtual Machines(虚拟机),110
virtual memory block identification(虚拟存储器块识
别),B-45
virtual memory fiast address translation(虚拟存储器快速
地址转換),B-46
virtual memory page size selection(處拟存储器页大小
选择),B-47
virtual memory protection(虚拟存储器保护),106~107
Trojan horses(特洛伊木马),B-5L •
True dependence(真相关)
loop-level parallelism caleulations(循环级并行计算),
320
name dependence(名称相关),153
True sharing misses(真共享缺失),366~367
True speedup,multiprocessor performance(真实加速比,多
处理器性能),406
Turbo mode(Turbo 模式)
hardwace enhancements(硬件增强),56
索
引
591
mnicroprocessors(微处理器),26
Two-level branch predictors(两级分支预测器)
branch costs(分支成本),163
Intel Core i7 (Intel Core i7),166
tournament predictors(竞赛预测器),165
Two-level cache hierarcby(两级缓存层次结构)
cache optimization(缓存优化),B-31
ILP(ILP),245
Two-way conflict misses, definition(两种冲突缺失,定义),
B-23
Two-way set associativity(两路组相联)
ARM Cortex-A8, 233
cache block placement(缓存块放置),B-7,B-8
cache miss rates(缓存敏失率),B-24
cache miss rates vs. size(缓存缺失率与大小),B-33
cache optimization(缓存优化),B-38
cache organization calculations(缀存组织计算),
B-19~B-20
commercial workload(商业工作负载)370~373,371
muitigrogramming workload(多重编程工作负载),
374~375
nonblocking cache(非阻塞缓存),84
Opteron data cache(Opteron 数据缓存),B-13-B-14
2:1 cache rule of thumb (2:1 缓存经验规则),B-29
virtual to cache access scenario(虚拟至缓存访同情景),
B-39
《Typical"program, instruction set considerations (“典型”
程序,指令集考虑事项),A-43
U
U,见 Rack units
UMA,见 Uniform memory access
Uncached state, directory-based cache coherence protocol
basics(未缓存状态,目录式缓存一致性协议基础),
380,384~386
Unconditional branches(无条件分支)
branch folding(分支折合),206
branch-prediction schemes(分支预测方案),C-25~C-26
Unicode character(Unicode 字符)
MIPS data types(MIPS 数据类型),A-34
operand sizes/types(操作数大小/类型),12
popularity(普及性),A-14
Unified cache(统一缓存)
AMD Opteron example (AMD Opteron示例),B-15
performance(性能)B-16~B-17
Uniform memory access(UMA)(一致存储器访问)
multicore single-chip multiprocessor(多核单芯片多处
理器),364
592
索引
SMIP,346~348
Uninterruptible instruction(不可中断措令)
hardware primitives(硬件原语),388
synchronization(同步),386
Uninterruptible power supply (UPs)(不间断电源)
Google WSC (Google WSC),467
WSC calculations(WSC计算),435
WSC infrastructure(WSC基础设计),447
Uniprocessors cache protocols(单处理器缓存协议),359
development views(开发视角),344
linear speedups(线性加速比),407
memory bierarcby design(存储器层次结构设计),73
memorysystem coherency(存储器系统一致性),353,
358
misses(缺失),371,373
multiprogramming workload(多重编程工作负载),
376~377
multithreading(多线程),223~229
Unpacked decimal(非压缩十进制),A-14
UPS,见 Uninterruptible power supply
Use bit(使用位),B-46,B-52
User maskable events, definition(用户可遮罕事件,定义),
C-45~C-46
User nonmaskable events, definition(用户不可逐單事件,
定义),C-45
User-requested events, exception requirerents(用户请求事
件,异常需求),C-45
Utility computing(公用计算),455~461
V
Valid bit(有效位)
acdress translation(地址转换),B-46
block identification(块识别),B-7
Opteron data cache ( Opteron 数据缓存),B-14
paged virtual memory(分页雄拟存储器),B-56
segmented virtual memory(分段虚拟存储器),B-$2
snooping(监听),357
symmetric shared-mernory mmultiprocessors(对称共享存
储器多处理器),366
Value prediction(值预测),202
Variable length encoding(变长编码)
control flow instruction branches(控制流指令分支),
A-18
instruction sets(指令集),A-22
ISAs,14
Variables(变量)
compiler technology(编译器技术),A-27~A-29
CUDA,289
Fermi GPU,306
ISA,A-S,A-12
locks via coherence(通过一致性锁定),389
loop-level parallelism(循环级并行),316
memory consistency(存储器一致性),392
NVIDIA GPU Memory (NVIIA GPU 存储器),
304~305
procedure invocation options(过程调用选项),A-19
register allocation(寄存器分配),A-26~A-27
in tegisters(在寄存器中),A~5
synchronization(同步),375
TLP programmer's viewpoint (TLP 程序员的观点),
394
VCs,见 Virtual channels
Vector architectures(向量体系结构),9
Vector Instraction(向量指令),292,309
Vectorizable Loop(可向量化循环),268,292,313
Vectorized code(向量化代码)
multimedia compiler support(多媒体编译器支持),
A-31
vector architecture programming(向量体系结构编程),
280~282
vector exccution time(向量执行时间),271
VMIPS (VMIPS),268
Vectorized Loop.另见 Body of Vectorized Loop
Vector Lane Kegisters,definition(向量车道寄存器,定义),
292
Vector Lanes(向量车道),292,309
Vector-length register (VLR)(向量长度寄存器)
basic operation(基本操作),274~275
VMIPS,267
Vector load/store unit(向量载人/存储单元)
memory banks(存储器组),276~277
VMIPS, 265
Vector loops(向量循环)
NVIDIA GPU (NVIDIA GPU),294
processor example(处理器示例),267
strip-mining(条带挖掘),303
vector vs. GPU(向量与 GPU),311
vector-length registers(向量长度寄存器),274~275
vectot-mask registers(向量遮單寄存器),275~276
Vector-rnask control, charaeteristics(向量遮罩寄存器,特性),
275~276
Vector-mask registers(向量遮罩寄存器)
basic operation(基本操作),275~276
VMIPS,267
Vector Processor(向量处理器),292,309
Vector Registers(向量寄存器),309
Very Long Instruction Word (VLIW)(超长指令字)
clock rates(时钟頻率),244
ILP,193~196
loop-level parailelism(循环级并行),315
multiple-issue processors(多发射处理器),194,
sample code(示例代码),252
Video(视频)
Amazon Web Services (Amazon Web服务),460
application trends(应用趋势),4
PMID$,6
WSCs,8, 432,437,439
Virtual address(虚拟地址)
address translation(地址转换),B-46
AMD64 paged virtual memory (AMD64分页虚拟存储
器),B-55
AMID Opterton data cache (AMID Opteron 数据缓存),
B-12~B-13
ARM Cortex-A8, 115
cache optimization(缓存优化),B-36~B-39
GPU conditional branching(GPU 条件分支),303
Intel Core i7,120
mapping to physical(映射到物理地址),B-45
memory hierarchy(存储器层次结构),B-39,B-48,
B-48~B-49
memory hierarchy basics(存储器层次结构基础),77~78
miss rate vs. cache size(缺失率与缓存大小),B-37
Opteron mapping (Opteron映射),B-55
Opteron mmemnory management (Opteron 存储器管理),
B-55~B-56
page size(页大小),B-58
page table-based mapping(基于页表的映射),B-45
translation(转换),B-36~B-39
virtual memory(虚拟存储器),B-42,B-49
Virtual acdress space(虚拟地址空间)
exarple(示例),8-41
mnain merory block(主存储器块),B-44
Virtual caches(虚拟缀存),B-36~-B-37
Virtual functions, control flow instructions(虚拟功能,控制
流指令),A-18
Virtualizable architecture(可虚拟化体系结构)
Intel 80x86 issues (Intel 80x86问题),128
system call performance(系统调用性能),141
Virtual Machines support(虚拟机支持),109
VMIM implementation (VMM实现):128~129
Virtualizable GPUS, future technology(可虚拟化 GPU, 未
来技术),333
Virtual machine monitor (VMIM)(虛拟机监视器)
characteristics(特性),108
nonvirtualizable ISA(不可虚拟化 ISA),126,128~129
requirements(需求),108~109
索引
593
Virtual Machines ISA support(虚拟机 ISA 支持),109~110
Virtual Machines (VMs)(虛拟机 (VM))
Amazon Web Services(Amazon Web 服务),456~457
cloud computing costs(云计算威本),471
ISA support(ISA 支持),109~110
protection(保护),107~108
protection and ISA(保护与ISA),112
server benchmarks(服务器基准測试),40
virtual memory and 1/O( 拟存储器与1/O),110~-111
WSCs, 436
Xen VM,111
Virtual memory(虚拟存储器),B-3
Virtual Machines impact(虚拟机影响),110~111
Virtual methods, control flow instructions(虚拟方法,控制
流指令),A-18
VLIW,见 Very Long Iastruction Word
VLR,见 Vector-length register
VMIPS
basic structure(基本结构),265
DLP,265~267
double-precision FP operations(双精度浮点运算),266
gather/scatter operations(集中/分散操作),280
ISA components(ISA分量),264~265
.multidimensional arrays(多维数组),278~279
Multimedia SIMD Extensions(多媒体 SIMID扩展)282
multiple lames(多车道),271~272
vector execution time(向量执行时间),269~270
vector Vs.GPU(向量与 GPU),308
vector-length registers(向量长度寄存器),274
vector load/store unit bandwidth(向量载人/存储单元带
宽),276
vector processor example(向量处理器示例),267~268
VLR,274
Voltage regulator modules (VRMs), WSC server energy
efficiency(电压调节模块,WSC服务器能量效率),
462
Volume-cost relationship,components(体积-成本关系,分
量),27~28
w
Wafers(晶圆)
exarple(示例),31
integrated circuit cost trends(集成电路成本趋势),
28~32
Wafer yield(晶圆正品率),30
Wall-clock time(壁钟时间)
execution timoe(执行时间),36
WAR,见Write after read
594
索引
Warehouse-scale computers(WSCs)(仓库级计算机),
345
Warp Scheduler(Warp 调度程序),292,314
WAW,见 Write after write
Way prediction, cache optimization(路预測,缓存优化),
81~82
Way selection(路选择),82
WB,见 Write-back cycle
Weak ordering, relaxed consistcncy models(弱排序,宽松
连贯性模型),395
Weak scaling, Amdahl'slaw and parallel computers(弱扩展,
Amdahl定律和并行计算机),406~407
Web index search, shared-memory workloads (Web 索引搜
索,共享存储器工作负载),369
Web servers(Web 服务器)
IL.P for realizable processors(可实现处理器的IP),
218
pexformsance benchmarks(性能基准测试),40
Weitek (Weiteck),364
Wet-bulb temperature(湿球温度)
Google WSC(Google WSC),466
WSC cooling systerns(WSC制冷系统),449
Window
latency(延迟),B-21
processor perfornance calculations(处理器性能计算),
218
scoreboarding definition(记分卡定义),C-78
Window size(窗口大小)
ILP lirnitations(TLP 局限性),221
ILP for realizable processors(可实现处理器的 ILP),
216~217
parallelism(并行度),217
Wires(连线)
energy and power(能量与功率),23
scaling(缩放),19~21
Within instruction exceptions (指令内异常),C45
Word count(词数),B-53
Word offiset, MIPS(宇偏移量,MIIPS),C-32
Words(字)
aligned/misaligned addresses(对齐/未对齐地址),4-8
AMID Opteron data cache(AMI Opterot 数据缓存),
B-15
memory address interpretation(存储器地址解释),
A-7~A-8
MIPS dats transfers(MIPS 数据传输),A-34
MIPS data types(MIPS 数据类型),A-34
operand sizes/types(操作数大小/类型),12
operand type(操作数类型),A-13~A-14
Workioads(工作负载)
execution time(执行时间),37
Google search(Google搜索),439
Java and PARSEC without SMT (Java 与 PARSEC,不
采用SMT),403~404
symnmetric shared-memory multiprocessor performance
(系统共享存储器多处理器性能),367~374
WSC goals/reguirements(WSC目标/需求),433
WSC resource allocation case study (WSC资源分配案
例研究),478~479
WSCs,436~441
Write after read(WAR)(读后写)
data hazards(数据冒险),153~154,169
dynamic scheduling with Tomasulo's algorichm(采用
Tomasulo 算法的动态调度),170~171
hazards and forwarding(冒险与转发),C-$5
ILP limitation studies (ILP 局限性研究),220
MIPS scoreboarding(MIPS 记分卡),C-72,C-74~C-75,
C-79
register renaming vs. ROB(寄存器重命名与 ROB),
208
ROB, 192
Tomasulo's advantages(Tomasulo 的优点),177~178
Tomasulo's algorithm(Tomasulo 算法),182~183
Write after write(WAW)(写后写(WAW)),
data hazards(数据冒险),153,169
dynamic scheduling with Tomasulo's algorithm(采用
Tomasulo算法的动态调度),170-171
execution sequences(执行序列),C-80
hazards and forwarding(買险与转发),C55~C-58
ILP limitation studies(ILP 局限性研究),220
microarchitectural techniques case study(微体系结构技
术案例研究),253
MPS FP pipeline performance(MIIPS 浮点流水线性
能),C-60~C-61
MIPS scoreboarding(MIPS 记分卡),C-74,C-79
register renaming vs. ROB(寄存器重命名与ROB),208
ROB, 192
Tomasulo's advantages(Tomnsulo的优势),177~178
Write allocate(写分配),B-11
Write-back cache(写回缓存),B-I1
Write-back cycle (WB)(写回周期)
basic MIPS pipeline(基本MIPS流水线),C-36
data hazard stall mininnization(数据冒险停顿最小化),C-17
execution sequences(执行序列),C-80
hazards and forvarding(冒险与转发),C-55~C-56
MIPS exceptions(MIPS执行),C-49
MIPS pipeline(MIPS 流水线),C-52
MIPS pipeline control (MIPs 流水线控制),C-39
MIPS R4000(MIPS R4000),C-63,C-65
MIPS scorcboarding(MPS 记分卡),C-74
pipeline branch issues(流水线分支问题),C-40
RISC classic pipeline(RISC经典流水线),C-7~C-8,
C-10
simple MIPS implementation(简单MIPS实现),C-33
simple RISC irplementation(简单 RISC实现),C-6
Write broadcast protocol, definition(写广播协议,定义),
356
Write bufter(写缓冲区)
AMID Opteron data cache(AMID Opteron 数据缓存),B-14
Intel Core i7, 118, 121
invalidate protocol(无效协议),356
memnory consistency(存储器一致性),393
memory bierarchy basics(存储器层次结构基础),75
miss penalty reduction(降低续失代价),87,B-32,
B-35~B-36
write merging example(写合并示例),88
write strategy(写策略),B-11
Write hit(写命中)
cache coherence(缓存一致性),358
directory-based coherence(目录式一致性),424
Single-chip multicore mnultiprocessor(单芯片多核心处
理器),414
snooping coberence(监听一致性),359
write process(写进程),B-11
Write invalidate protocol(写失效协议)
directory-based cache coherence protocol example(目录
式缓存一致性协议示例),382~383
example(示例),359,360
irnplemeatation(实现),356~357
snooping coherence(监听一致性),355~356
Write merging(写合并)
example(示例),88
miss penalty reduction(降低缺失代价),87
Write miss(写缺失),385
Write result stage(写结果级)
data hazards(数据冒险),154
dynamic scheduling(动态调度),174~175
hardware-based speculation(基于硬件的推测),192
instruction steps(指令步骤),175
ROB instruction (ROB 指令),186
scoreboarding(记分卡),C-74~C-75,C-78~C-80
status table examples(状态表示例),C-77
索
引
595
Toasulo's algorithm(Tomasulo算法),178,180,
190
Write serialization(写入序列化)
hardware primitives(硬件原语),387
multiprocessor cache coherency(多处理器缓存一致性),
3$3
snooping coherence(监听一致性),356
Write stall, definition(写停顿,定义),B-11
Write strategy(写策略)
memory hierarchy considerations(存储器层次结构考虑
事项),B-6,B-10~B-12
vitual memory(成拟存储器),B-45~B-46
Write-through cache(直写缓存)
average mnemory accese tine(存储器平均访问时间),B-16
coherency(一致性),352
invalidate protocol(失效协议),356
memory hierarchy basics(存储器层次结构基础),74~75
miss penaities(缺失代价),B-32
optimization(优化),B-35
snooping coberence(监听一致性),359
write process(写过程),B-11~B-12
Write update protocol,definition(写更新协议,定义),356
WSCs,见 Warchouse-scale computers
X
Xen Virtual Machine(Xen 虚拟机)
Amazon Web Services(Amazon Web服务),456~457
characteristics(特性),111
Y
Yahoo!,WSCs,465
Yield(正品率)
chip tabrication(芯片制造),61~62
cost trends(成本趋势),27~32
FermiGTX 480 (Fermi GTX 480),324
Z
Zynga, FarmVille,460