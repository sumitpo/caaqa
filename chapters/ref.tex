
参考文献
Adve, S. V. and K. Gharachorloo [1996]. "Shared memory consistency models: A tutorial"JEKE Computer 29:12 (December),
66-76.
Adve, S. V., and M. D. Hill [1990]. “Weak ordering—a new definition,"Proc. 17th Annual Int'. Symposium on Computer
Architecture (ISCA), May 28-31, 1990, Seattle, Wash.,2-14.
Agarwal, A.[1987]. “Analysis of Cache Performance for Operating Systems and Multiprogramming," Ph.D. thesis, Tech. Rep.
No.CSL-TR-87-332, Stanford University, Palo Alto, Calif.
Agarwal, A. [1991]. “Limits on interconnection network performance," JEEE Trans.on Parallel and Distributed Systems 2:4
(April),398-412.
Agarwal, A., and S. D. Pudar [1993]. "Colutan-associative caches: A technique for reducing the miss rate of direct-mapped
caches," 20th Annal Ine '. Sympositm on Computer Architectre (ISCA), May 16-19, 1993, San Diego, Calif. Also appears
in Computer Archilecrure News 21:2 (May), 179-190, 1993.
Agarwal, A., R. Bianchini, D. Chaiken, K. Johnson, and D. Kranz [1995]. "The MIT Alewife machine: Architecture and
performance," Int'Y. Symposium on Computer Architecture (Denver, Colo.), June, 2-13.
Agarwal, A., J. L. Hennessy, R. Simoni, and M. A. Horowitz [1988]. "An evaluation of directory schemes for cache coberence,”
Proc. I5th Int '. Symposium on Computer Architecture (June), 280-289.
Agarwal, A., J. Kubiatowicz, D. Kranz, B.H. Lim, D. Yeung,G. D’'Souza, and M. Parkin [1993]. “Sparcle: An evolutionary
processor design for large-scale multiprocessors”" IEEE Micro 13 (Jume), 48-61.
Agerwala, T., and J. Cocke [1987]. High Perforance Reduced Instruction Set Processors, TBM Tech. Rep. RC12434, 1BM,
Armonk, N.Y.
Akeley, K. and T. Jermoluk [1988]. “High-Performance Polygon Rendering."Proc. 15th Annual Conf. on Computer Graphics
and Interactive Techniques (SIGGRAPH 1988), August 1-5, 1988, Atlanta, Ga., 239-246.
Alexander, W. G., and D. B. Wortman [1975]. "Static and dynamic characteristics of XPL programs" EEE Computer 8:11
(November),41-46.
Alles, A. [1995]. “ATM Internetworking." White Paper (May), Cisco Systems, Inc, San Jose, Calif. (wm.cisco.com/warp/
public/614/12.hmml.
Aliant. [1987].AIliant FX/Series: Product Summary, Alliant Computer Systems Corp., Acton, Mass.
Almasi, G. S. and A. Gottlieb [1989]. Highly Parallel Computing, Benjamin/Cummings, Redwood City, Calif.
Alverson, G., R. Alverson, D. Callahan, B. Koblenz, A. Porterfield, and B. Smith [1992]. "Exploiting heterogeneous parallelism
on a multithreaded multiprocessor,”Proc. ACMIEEE Conf:on Supercomputing, Noverber 16-20, 1992, Minneapolis,
Minn.,188-197.
Amdahl, G. M.[1967]. *Validity of the single processor approach to achieving large scale computing capabilities” Proc. AFIPS
参考文献
519
Spring Joint Computer Conf., April 18-20, 1967, Atlantic City, N.J., 483-485.
Amdahl, G. M., G. A. Blaauw, and F. P. Brooks, Jr. [1964]. "Architecture of the IBM System 360,” IBM J. Research and
Development 8:2 (April),87-101.
Amz,C., A. L. Cox, S. Dwarkadas, P. Keleher, H. Lu, R. Rajamony, W. Yu, and W.Zwaenepoel [1996]. "Treadmarks: Shared
memory corputing on networks of workstations," IEEE Computer 29:2 (February), 18-28.
Anderson, D. [2003]. “You don't know jack about disks." Queue, 1:4 (June), 20-30.
Anderson, D., J. Dykes, and E. Riedel [2003]. “SCSl vs. ATA—More than an interface," Proc. 2nd USENLK Conf. on File and
Storage Technology (FAST 03), March 31-April 2, 2003, San Francisco.
Anderson, D. W., F. J. Sparacio, and R. M. Tomasulo [1967]. "The IBM 360 Model 91: Processor philosopby and instruction
handling"JBMJ. Research and Development 11:1 (January), 8-24.
Anderson, M. H. [1990].“Stength (and safety) in numbers (RAID, disk storage technology).” Byte 15:13 (December), 337-339
Anderson, T. E., D. E. Culler, and D. Patterson [1995]. “A case for NOW (networks of workstations),” IEEE Micro 15:1
(February),54-64.
Ang, B.,D. Chiou, D. Rosenband, M. Ebrlich, L. Rudolph, and Arvind [1998]. “StarTVoyager: A flexible platform for exploring
Scalable SMP issues." Proc. ACM/IEEE Conf. on Supercompuing, November 7-13, 1998, Orlando, FL.
Anjan, K. V., and T. M. Pinkston [1995]. “An efhicient, fully-adaptive deadlock recovery scheme: Disha,” Proc. 22nd Anual
Int '. Symposium on Compater Architectre (ISCA), June 22-24, 1995, Santa Margherita, Italy.
Anon.et al. [1985]. A Measure of Transaction Processing Power, Tandem Tech. Rep. TR85.2. Also appears in Datamation 31:7
(April), 112-118,1985.
Apache Hadoop. [2011].http://hadoop.apache.org.
Archibald, J., and J.-L. Baer [1986}. "Cache coherence protocols: Evabuation using a multiprocessor simulation model” ACM
Trans.on Computer Systems 4:4 (November), 273-298.
Armbrust, M., A. Fox, R. Griffith, A. D. Joseph, R. Katz, A. Konwinski, G. Lee, D. Patterson, A. Rabkin, I. Stoica, and M.
Zaharia [2009]. Above the Clouds: A Berkeley View of Cloud Computing, Tech. Rep. UCB/EECS-2009-28, University of
California, Berkeley (htp:/hww.eecs.berkekey. ec Pubs/TechRpts/2009/EECS-2009-28.html).
Arpaci, R. H., D.E. Culler, A. Krishnamurthy, S. G. Steinberg, and K. Yelick [1995]. "Empirical evaluation of the CRAY-T3D:
A compiler perspective.”22nd Annual Int '. Symposium on Computer Archilecture (ISCA), June 22-24, 1995, Santa
Margherita, Italy.
Asanovic, K. [1998]. “Vector Microprocessors, " Ph.D. thesis, Computer Science Division, University of California, Berkeley.
Associated Press. [2005]. "Gap Inc. shuts down two Internet stores for major overhaul," USATODAY.com, August 8, 2005.
Atanasoff, J. V.[1940]. Compuring Machine for the Solution of Large Systems of Linear Eguations, Interal Report, Iowa State
University,Ames,
Atkins, M.[1991]. Perfornance and the i860 Microprocessor, IEEE Micro, 11:5 (Septemnber), 24-27, 72-78.
Austin, T. M., and G. Sohi [1992]. "Dynamicdependency analysis of ordinary programs”Proc. 19th Anmual Int '.Symposium on
Computer Architecture (ISC.A), May 19-21, 1992, Gold Coast, Australia, 342-351.
Babbay, F, and A. Mendelson [1998]. "Using value prediction to increase the power of speculative execution hardware," ACM
Trans.on Computer Systems 16:3 (August), 234-270.
Baer, J. L., and W.-H. Wang [1988]. "On the inclusion property for multi-level cache hierarchies." Proc. J5th Annual Int1.
Smpositm on Computer Architecture, May 30-June 2,1988, Honolulu, Hawaii, 73-80.
Bailey, D. H., E. Barszcz, J. T. Barton, D. S. Btowning, R. L. Carter, L. Dagum, R. A. Fatoohi, P. O. Frederickson, T. A. Lasinski,
R. S. Schreiber, H. D. Simon, V. Venkatakrishnan, and S. K. Weeratunga [1991]. “The NAS paraliel benchmarks” Int'Y. J.
Supercomputing Applications 5,63-73.
Bakoglu, H. B., G. F. Grohoski, L. E. Thatcher, J. A. Kaeli, C. R. Moore, D. P. Tattle, W.B. Male, W. R. Hardell, D. A. Hicks, M.
Nguyen Phu, R. K. Montoye, W.T. Glover, and S. Dhawan [1989]."TBM second-generation RISC processor organization.”
Proc. IEEE In: '.Conf on Computer Design, September 30-October 4,1989, Rye, N.Y, 138-142.
Balakrishnan, H., V. N. Padmanabhan, S. Seshan, and R. H. Katz [1997]. “A corparison of mechanisms for improving TCP
performance over wireless links,"IEEE/ACM Trans.on Networking 5:6 (December), 756-769.
Ball, T., and J. Larus [1993]. “Branch prediction for ftce," Proc. ACM SIGPLAN'93 Conference on Programming Language
Design and Implementation (PL.DI), June 23-25, 1993, Albuquerque, N.M., 300-313.
520
参考文献
Banerjee, U. [1979]. "Speedup of Ordinary Programs,” Ph. D. thesis, Dept. of Computer Science, University of Illinois at
Urbana-Champaign.
Barham, P., B. Dragovic, K. Fraser, S.Hand, T. Harris, A. Ho, and R. Neugebauer [2003]. "Xen and the art of virtualization,
Proc. ofthe I9th ACM Symposium on Operating Systems Principles, October 19-22, 2003, Bolton Landing, N.Y.
Barroso., L. A. [2010]. "Warehouse Scale Computing [keyzote address]." Proc. ACM SIGMOD, Jue 8-10, 2010, Indianapolis, Ind.
Barroso, L.A., and U. H8lzle [2007], "The case for energy-proportioual computing,"EEE Computer, 40:12 (Decemlber), 33-37.
Barroso, L. A., and U. Holzle [2009]. The Datacenter as a Compuer: An Introduction to the Design of Warehouse-Scale
Machines, Morgan & Claypool, San Rafael, Calif.
Barroso, L. A., K. Gharachorloo, and E. Bugnion [1998]. "Memory system characterization of commercial workloads," Proc.
25th Annual Int ' .Symposium on Computer Architecture (ISC.A), July 3-14, 1998, Barcelona, Spain, 3-14.
Barton, R. S. [1961]. "A new approach to the functional design of a computer," Proc W estern Joint Computer Conf., May 9-11,
1961,Los Angeles, Calif., 393-396.
Bashe,C. J., W. Buchholz, G. V. Hawkins, J. L. Ingram, and N. Rochester [1981]. *The architecture of IBM's carly computers,”
IBMJ. Research and Development 25:5 (September), 363-375.
Bashe, C.J., L. R. Johnson, J. H. Palmer, and E. W. Pugh [1986]. IBM's Early Computers, MIT Press, Carbridge, Mass.
Baskett, F., and T. W. Keller [1977]. “An evaluation of the Cray-1 processor," in High Speed Computer and Algorithm
Organization, D. J. Kuck, D. H. Lawrie, and A. H. Sameh, eds., Academic Press, San Diego, 71-84.
Baskett, F. T. Jermoluk, and D. Solomon [1988]. "The 4D-MP grapbics superworkstation: Computing + graphics=40 MPS+ 40
MFLOPS and 10,000 lighted polygons per second.,” Proc. IFEE COMIPCON, February 29-March 4, 1988, San Francisco,
468-471.
BBN Laboratories. [1986]. Butterfy Parallel Processor Overview, Tech. Rep. 6148, BBN Laboratories, Cambridge, Mass.
Bell, C.G. [1984]. "The mini and micro industries." IEEE Computer 17:10 (October), 14-30.
Bell, C. G. [1985]. "Multis: A new class of mnultiprocessor computers." Science 228(April 26). 462-467.
Bell, C.G. [1989]. "The future of high performance computers in science and engineering,." Communications ofthe ACM 32:9
(Septernber), 1091-110L.
Bell, G., and J. Gray [2001]. Crays, Clusters and Centers, Tech. Rep. MSR-TR-2001-76, Microsoft Research, Redmond, Wash.
Bell, C. G., and J. Gray [2002]. “What's next in high performance computing?”CACM 45:2 (February), 91-95.
Bell, C. G., and A. Newell [1971]. Computer Structures: Readings and Examples, McGraw-Hill, New York.
Bell, C. G., and W. D. Strecker [1976]. “Computer structures: What have we learned from the PDP-11?," Third Anmual Int '.
Symposium on Computer Architecture (USCA), January 19-21, 1976, Tampa, Fla., 1-14.
Bell, C. G., and W. D. Strecker [1998]. "Computer structures: What have we learned from the PDP-117" 25 Years of the
International Symposia on Computer Architecture (Selected Papers), ACM, New York, 138-151.
Bell, C. G., J. C. Mudge, and J. E. McNamara [1978]. A DEC View of Computer Engineering, Digital Press, Bedford, Mass.
Bell, C. G., R. Cady, H. McFarland, B. DeLagi, J. O'Laughlin, R. Noonan, and W. Wulf [1970]. "A new architecture for
mini-computers: The DEC PDP-11,” Proc. AFIPS Spring Joit Comgputer Conf., May S-May 7, 1970, Atlantic City, N.J., 657-675.
Benes, V. E. [1962]. “Rearrangeable three stage connecting networks," Bell Sysrem Technical. Journal 41, 1481-1492.
Bertozzi, D., A. Jalabert, S. Murah, R. Tamhankar, S. Stergiou, L. Benini, and G. De Micheli [2005]. “NoC synthesis flow for
customized dorain specific multiprocessor systems-on-chip," JEEE Trans.on Parallel and Dis tributed Syusiems 16:2
(February),113-130.
Bhandarkar, D. P.[1995]. Alpha Architecture and Implementations, Digital Press, Newton, Mass.
Bhandarkar, D. P., and D. W.Clark [1991]. "Performance from architecture: Comparing a RISC and a CISC with similar
hardware organizations,” Proc. Fourth Int'. Conf:on Architectural Support Jor Programming Languages and Operating
Sustems (ASPLOS), April 8-11, 1991, Palo Alto, Calif., 310-319.
Bhandarkar, D. P., and J. Ding [1997]. “Performance characterization of the Pentium Pro processor,”Proc. Third Int '1.
Symposium on High-Performance Compuer Architecture, February 1-February 5, 1997, San Antonio, Tex., 288-297.
Bhuyan, L. N., and D. P. Agrawal [1984]. "Generalized hyperoube and hyperbus structures for a computer network," IEEE Trans.
on Computers 32:4 (April),322-333.
Bienia, C., S. Kumat, P. S. Jaswinder, and K. Li [2008]. The Parsec Benchmark Suite: Characterization and Architectural
mplicarions,Tech. Rep. TR-811-08, Princeton University, Princeton, N.J.Bier, J.[1997]. “The Evolution of DsP
参考文献
521
Processors" presentation at Univesity of Califoria, Berkeley, November 14.
Bird,S., A. Phansalkar, L. K. John, A. Mericas, and R. Indukuru [2007]."Characterization of performance of SPEC CPU
benchmarks on Intel's Core Microarchitecture based processor,” Proc. 2007 SPEC Benchmark Workshop, January 21, 2007,
Austin, Tex.
Birman, M., A. Samuels, G. Chu, T. Chuk, L. Hu, J. McLeod, and J. Barnes [1990]"Developing the WRI3170/3171 SPARC
floating-point coprocessors,” IEEE Micro 10:1,55-64.
Blackbur, M., R. Garner, C. Hoffoan, A. M. Khan, K. S. MeKinley, R. Bentzur, A. Diwan, D. Feinberg, D. Frampton, S. Z.
Guyer, M.Hirzel,A. Hosking, M. Jump,H. Lee, J. E. B. Moss, A. Phansalkat, D.Stefanovic, T. VanDrunen, D. von
Dincklage, and B. Wiedermann [2006]. "The DaCapo benchmarks: Java benchmarking development and analysis,”ACM
SIGPLAN Conference on Object-Oriented Programming, Systems, Languages, and Applications (OOPSL.A), Oetdber 22-26,
2006,169-190.
Blaum, M., J. Bruck, and A. Vardy [1996]. "MIDS array codes with independent parity symbols.” IEEE Trans. on Information
Theory, IT-42 (March), 529-42.
Blaum, M.,J. Brady, J.Bruck, and J. Menon [1994]. “EVENODD: An optimal scheme for tolerating double disk tailures in RAID
architectures," Proc. 2Ist Annual Int '. Symmposium on Computer Archiecture (LSCA), April 18-21, 1994, Chicago, 245-254.
Blaum, M., J. Brady, J. Bruck, and J. Menon [1995]. “EVENODD: An optimal scheme for tolerating double disk failures in RAID
architectures,"IEEE Trans. on Computers 44:2 (February), 192-202.
Blaum, M., J. Brady, J., Bruck, J. Menon, and A. Vardy [2001]. "The EVENODD code and its generalization," in H. Jin, T. Cortes,
and R. Buyya, eds., High Performance Mass Storage and Parallel I/O: Technologies and Applications, Wiley-IEEE, New
York,187-208.
Bloch, E. [1959].*The engineering desigo of the Stretch computer,” 1959 Proceedings of the Eastern Joint Computer Conf,
December 1-3, 1959, Boston, Mass., 48-59.
Boddie, J. R.[2000]. *Hlistory of DSPs" www.lucent.com/micro/dsp/dsphist.htm!.
Bolt, K. M. [2005]. “Amazon sees sales rise, profit fall," Seattle Post-Intelligencer, October 25 (http://seattlepi.nwsource.com/
business/245943_techearns26.html).
Bordawekar, R., U. Bondhugula, R. Rao [2010]. “Believe It or Notl: Multi-core CPUs can Match GPU Performance for a
FLOP-Intensive Application!" 19th Interational Conference on Parallel Architecture and Compilation Techniques (PACT
2010). Vienna,Austria, September 11-15,2010, 537-538.
Borg, A., R. B. Kessler, and D. W. Wall [1990]. "Generation and analysis of very long address traces,” 19th Annual Int 'Y.
Symposium on Computer Architecture (ISCA), May 19-21, 1992, Gold Coast, Australia, 270-279.
Bouknight, W. J., S. A. Dencberg, D. E. Mclntyre, J. M, Randall, A. H. Sameh, and D.L. Slotnick [1972]. "The Illiac TV system,”
Proc. IEEE 60:4, 369-379.Also appears in D. P. Siewiorek, C. G. Bell, and A. Newell, Computer Structures: Principles and
Examples, McGraw-Hill, New York, 1982, 306-316.
Brady, J. T. [1986]. “A theory of productivity in the creative process," EEE CG&A (May), 25-34.
Brain, M. [2000]. "Inside a Digital Cell Phone," www.howstufhvorks.com/insidecellphone.htm.
Brandt, M., J. Brooks, M. Cahir, T. Hewitt, E. Lopez-Pineda, and D. Sandness [2000]. The Benchmarker's Gride for Cray SV!
Syxstems. Cray Inc., Seattle, Wash.
Brent, R. P. and H.T. Kung [1982]. "A regular Layout for parallel adders." JEEE Trans. on Computers C-31, 260-264.
Brewer, E. A., and B. C. Kuszmani [1994]. "'How to get good performance from the CM-5 data network " Proc. Eighth Int'7.
Parallel Processing Symposium, April 26-27, 1994,Cancun,Mexico.
Brin, S., and L. Page [1998]."The anatomy of a large-scale hypertextual Web search engine"Proc. 7th Int7.World Wide Web
Conf., April 14-18, 1998, Brisbane, Queensland, Australia, 107-117.
Brown,A., and D. A. Patterson [2000]. “Towards maintainability, availability, and growth benchmarks: A case study of software
RAID systems." Proc. 2000 USENLK Annual Technical Conf., Jume 18-23, 2000, San Diego, Calif.
Bucher, 1. V., and A. H. Hayes [1980]. "1O performance measurerent on Cray-1 and CDC 7000 computers,”Proc. Compuer
Performance Evaluation Users Group, I6th Meeting, NBS 500-65,245-254.
Bucher, L. Y.[1983]. "The corputational speed of supercomputers," Proc. Int'Y. Conf. on Measuring and Modeling of Computer
Systems (SIGMETRICS 1983), August 29-31,1983, Minneapolis, Minn.,151-165.
Buchoitz, W.[1962]. Planning a Computer System: Project Stretch, McGraw-Hill, New Yotk.
522
参考文献
Burgess, N., and T. Williams [1995]. “Choices of operand truncation in the SRT division algorithm," IEEE Trans. on Compuuters
44:7,933-938.
Burkhardt II, H., S. Frank, B. Knobe, and J. Rothnie [1992]. Overview ofthe KSRl Computer System, Tech. Rep. KSR-TR-
9202001, Kendall Square Research, Boston, Mass.
Burks,A. W.,H. H.Goldstine, and J. von Neumann [1946]. "Prelimninary discussion of the logical design of an electronic
computing instrument"Report to the U.S. Army Ordnance Department, p. l; also appears in Papers of John von Neumanr, W.
Aspray and A. Burks, eds, MIT Press, Cambridge, Mass., and Tomash Publishers, Los Angeles, Calif.,, 1987, 97-146.
Calder, B., G. Reinman, and D. M. Tullsen [1999]. “Selective value prediction,”Proc. 26th Anmual Int 1.Symposium on
Computer Architecture (ISCA), May 2-4, 1999, Atlanta, Ga.
prediction using machine learning." ACM Tans. Program. Lang. Sys!. 19:l, 188-222.
Callahan, D., J. Dongarra, and D. Levine [1988]. “Vectorizing compilers: A test suite and results,” Proc. ACM/TEEE Conf. on
Szpercomputing, November 12-17, 1988, Orland, Fla., 98-105.
Cantio, J. F., and M. D. Hill [2001]. “Cache Performance for Selected SPEC CPU2000 Benchmarks” www.iffred.org/cache-
data.htm/ (June).
Cantin, J. F, and M. D. Hill [2003]. “Cache Performance for SPEC CPU2000 Benchmarks, Version 3.0," www.cs.wisc.edw/
multifacet/misc/spec2000cache-data/index.html.
Carles, S.[2005]. "Amazon reports Tecord Xras season, top game picks."Gammasura, December 27(http://www.gamasutra.com/
php-bin/hnews_index.php?story=7630.)
Carter, J, and K. Rajamani [2010]. "Designing cnergy-efficient servers and data centers."IEEE Computer 43:7 (July), 76-78.
Case, R. P., and A. Padegs [1978]. “The architecture of the IBM System/370,"Communications of ihe ACM 21:1, 73-96.Also
appears in D. P. Siewiorek, C. G. Bell, and A. Newell, Computer Stuctures: Principles and Examples, McGraw-Hill, New
York,1982, 830-855.
Censier, L. and P. Feautrier [1978]. “A new solution to coberence problemns in multicache systems” IEEE Trans. on Computers
C-27:12 (December), 1112-1118.
Chandra, R., S. Devine, B. Verghese, A. Gupta, and M. Rosenblum [1994]. "Scheduling and page migration for multiprocessor
compute servers," Sixth Int 'Y. Conf. on Architectiral Suppont for Programming Languages and Operating Systems (ASPLOS),
October 4-7, 1994, San Jose, Calif., 12-24.
Chang, F., J. Dean, S. Ghemawat, W. C. Hsieh, D. A. Wallach, M. Burrows, T. Chandra, A. Fikes, and R. E. Gruber [2006].
“Bigtable: A distributed storage system for structured data," Proc. 7th USENLX Symposium on OperatingSystems Design and
Implementation (OSDI '06), November 6-8, 2006, Seattle, Wash.
Chang, J., J. Meza, P. Ranganathan, C. Bash, and A. Shah [2010]. "Green server design: Beyond operational euergy to
sustainability." Proc. Workshop on Power Aware Computing and Systems (HotPower '10), October 3, 2010, Vancouver,
British Columbia.
Chamg, P.P., S. A. Mahlke, W. Y. Chen, N. J. Warter, and W. W. Hwu [1991]. “IMPACT: An architechural framework for
maltipie-instruction-issue processors” ISth Anmual Int '. Symposi on Computer Architecouere (LSCA), May 27-30, 1991,
Toronto, Canada,266-275.
Charlesworth, A. E. [1981]. “An approach to scientific array processing: The architecture design of the AP-120B/FPS-164
famnily." Compuer 14:9(September),18-27.
Charlesworth, A. [1998]. "Starfire: Extending the SMIP envelope." IEEE Micro 18:1 (January/February), 39 49.
Chen, P.M., and E. K. Lee [1995]. “Striping in a RAID level 5 disk array" Proc. ACM SIGMETRICS Conf on Meastirement and
Modeling of Computer Systems, May 15-19, 1995, Ottawa, Canada, 136-145.
Chen, P. M., G. A. Gibson, R. H. Katz, and D. A. Patterson [1990]. “An evaluation of redundant arrays of inexpensive disks using
an Amdahl 5890," Proc.ACM SIGMETRICS Conf. on Measurement and Modeling o/Compuler Systems, May 22-25, 1990,
Boulder, Colo.
Chen, P. M., E. K. Lee, G. A. Gibson, R. H. Katz, and D. A. Patterson [1994]. “RAID: High-per fornance, reliable secondary
storage," ACM Computing Surveys 26:2 (June), 145-188.
Chen, S.[1983]."L.arge-scale and high-speed multiprocessor system for scientific applications," Proc. NATO Acvanced Research
Workshop on High-Speed Computing, June 20-22, 1983, Jtlich, West Germany. Also appears in K. Hiwang, ed.,
参考文献
523
''Superprocessors: Design and applications,”JEEE (August), 602-609, 1984.
Chen, T. C.[1980]. "Overlap and parallel processing." in H.Stone, ed., Introduction to Computer Architecture, Science Research
Associates, Chicago, 427-486.
Chow, F.C. [1983]. “A Portable Machine-Independent Global Optimizer—Design and Measurerents,” Ph.D. thesis, Stanford
University, Palo Alto, Calif.
Chrysos, G. Z., andJ.S. Emer [1998]. "Memory dependence prediction using store sets."Proc. 25th Anmual Int'. Symposiwm on
Computer Architecture (ISC.A), July 3-14, 1998, Barcelona, Spain, 142-153.
Clark, B., T. Deshane, E. Dow, S. Evanchik, M. Finlayson, J. Herne, and J. Neefe Matthews [2004]."Xen and the art of repeated
Tesearch.” Proc. USENIK Anntal Technical Conf., June 27-July 2,2004,135-144.
Clark, D. W. [1983]. “Cache performance of the VAX-11/780,” ACM Trans.on Computer Systems 1:1, 24-37.
Clark, D. W. [1987]“Pipelining and performance in the VAX 8800 processor," Proc. Second Inl'1.Conf. on Architectual
Support for Programming Languages and Operating Systems (ASPLOS, October S-8, 1987, Palo Alto, Calif.,173-177.
Clark, D. W., and J. S. Emer [1985]. "Performance of the VAX-11/780 translation buffer: Simulation and reasurement," ACM
Trans. on Computer Systems 3:1 (February),31-62.
Clark, D., and H. Lcvy [1982]. "Measurement and analysis of instruction set use in the VAX-11/780," Proc. Ninth Anmual Int!.
Sympasium on Computer Architecture (ISCA), April 26-29, 1982, Austin, Tex., 9-17.
Clark, D., and W. D.Strecker [1980]. “Comments on "the case for the recuced instruction set computer,”" Computer Architecture
News 8:6 (October), 34-38.
Clatk, W.A. [1957]. "The Lincoln TX-2 computer development," Proc. Western Joint Computer Conference, February 26-28,
1957, Los Angeles, 143-145.
Clidaras, J., C. Johnson, and B. Felderman [2010]. Private communication.
Climate Savers Computing Initiative.[2007]. “Efficiency Specs." htp://www.climatesaverscomputing.org/.
Clos, C.[1953]. "A study of non-blocking switching networks." Bell Systems Technical Journa/ 32 (March), 406-424.
Cody, W. J., J. T. Coonen, D. M. Gay, K. Hanson, D. Hougb, W. Kahan, R. Karpinski, J. Palmner, F. N. Ris, and D. Stevenson
[1984]. “A proposed radix-and wordlengthindependent standard for floating-point arithmetic,"IEEE Micro 4:4,86-100.
Colwell, R. P., and R. Steck [1995]. "A 0.6 Jmm BiCMOS processor with dynamic execution." Proc. of IEEE Int '1. Symposiam on
Solid State Circuits (ISSCC), February 15-17, 1995, San Francisco, 176-177.
Colwell, R. P., R. P. Nix, J.J. O'Donnell, D.B.Papworth,and P.K.Rodman[1987]. "A VLIW architecture for a trace scheduling
compiler,”Proc. Second Int '.Conf.on Architecmural Support for Programming Languages and Operating Systems (ASPLOS),
October 5-8, 1987, Palo Alto, Calif., 180-192.
Comer, D. [1993]. Insernetworking with TCP/P, 2nd ed., Prentice Hall, Englewood Cliftfs, N.J.
Compaq Computer Corporation. [1999]. Compiler Writer's Guide for the Alpha 21264, Order Number EC-RJ66A-TE, June,
wd.sgpor.compag.com/alpha-tools/documerntation/current/21264_EV67/ec-rj66g_re_comp_writ_gde_jor_alpha21264.pd.
Conti, C., D. H. Gibson, and S. H. Pitkowsky [1968]. “Structural aspects of the Syster/ 360 Model 85. Part I. General
organization," IBM Systems J. 7:1,2-14.
Coonen, J.[1984]. "Contributions to a Proposed Standard for Binary Floating-Point Arithmetic," Ph.D. thesis, University of
Corbett, P.,, B. English, A. Goel, T.Grcanac, S. Kleiman, J. Leong, and S. Sankar [2004]. “Row-diagonal parity for double disk failure
correction” Proc. 3rd USENIX Corf. on File and Storage Technology (FAST "04), March 31-April 2, 2004, Sam Francisco.
Crawford, J., and P.Gelsinger [1988]. Programming the 80386, Sybex Books, Alameda, Calif.
Culler, D. E., J. P. Singh, and A. Gupta [1999]. Parallel Computer Architecrure: A Hardvare/Software Approach, Morgan
Kaufimann, San Francisco.
Curow, H. J., and B.A. Wichmann [1976]. “A synthetic benchmatk ” The Computer d. 19:1, 43-49.
Cvetanovic, Z., and R. E. Kessler [2000]. "Performance analysis of the AJpha 21264- based Compag ES40 system,”Proc. 27th
Annual Int 'Y. Symposium on Computer Architecture (ISC.A), June 10-14, 2000, Vancouver, Canada, 192-202.
Dally, W.J. [1990J. *Performance analysis of k-ary n-cube interconnection networks”LEEE Tramns.on Computers 39:6 (June), 775-785.
Dally, W.J.[1992]. "Virtual channel flow control."EEE Trans.on Parallel and Distributed Systems 3:2 (March), 194-205.
Dally, W.J.[1999]. "Interconnect limited VLSl architecture," Proc. of the Intermational Intercomnect Technology Conference,
May 24-26, 1999, San Francisco.
524
参考文献
Dally, W.J., and C.I. Seitz [1986]. “The torus routing chip," Distributed Computing 1:4, 187-196.
Dally, W. J., and B. Towles [2001].“Route packets, not wires: On-chip interconnection networks" Proc. 38th Design Automation
Conference, June 18-22, 2001, Las Vegas.
Daily, W. J., and B. Towles [2003]. Principles and Practices of Interconnection Networks, Morgan Kaufmann, San Francisco.
Darcy, J. D., and D.Gay [1996]. “FLECKmnatks: Measuring floating point performance using a full IEEE compliant arithmetic
benchmark," cs 252 class project, University of Califoria, Berkeley (see HTTP.CS. Berheley.EDUI/darcy/(Projeccts/es252/ ).
Darley, H. M. et al. [1989]. “Floating Point/Integer Processor with Divide and Square Root Functions,”" U.S. Patent 4,878,190,
October 31.
Davidson, E.S. [1971]. “The design and control of pipelined function generators,”"Proc. IEEE Conf. on Systems, Nerworks, and
Computers, January 19-21, 1971, Oaxtepec, Mexico, 19-21.
Davidson, E.S., A. T. Thomas, L. E. Shar, and J. H. Patel [1975]. “Effective control for pipelined processors,”Proc. IEEE
COMPCON, February 25-27, 1975, San Francisco, 181-184. Colwell, R. P., and R. Steck [1995]. “A 0.6 Jmm BiCMOS
processor with dynamic execution." Proc. of IEEE Int'.Symposium on Solid Slate Circuits (ISSCC), February 15-17, 1995,
San Francisco, 176-177.
Colsell, R.P., R.P.Nix, J.J. O"Donnell, D. B. Papworth, and P. K. Rodman [1987].“A VLIW architecture for a trace scheduling
compiler,” Proc. Second Int'1. Conf. on Architectural Support for Programming Languages and Operating Systems (ASPLOS),
October 5-8, 1987, Palo Alto, Calif., 180-192.
Corner, D. [1993]. Internetworking with TCP/P, 2nd ed., Prentice Hall, Englewood Clifts, N.J.
Compaq Computer Corporation. [1999]. Compiler Writer's Gruide for the Alpha 21264, Order Number EC-RJ66A-TE, June,
ww1.suppor.compag.com/cpha-tooks/ documensation(current/21264_EV67/ec-rj6da._te._comp _wril gade for _obpha21.264.pot.
Conti, C., D. H. Gibson,and S. H. Pitkowsky [1968]. “Structural aspects of the System/ 360 Model 85. Part I. General
organization” IBM Systems J. 7:1,2-14.
Coonen, J.[1984]. “Contributions to a Proposed Standard for Binary Floating-Point Arithmetic,” Ph.D. thesis, University of
California, Berkeley.
Corbett, P., B. English, A. Goel, T. Grcanac, S. Kleiman, J. Leong, and S. Sankar [2004]. "Row-diagonal parity for double disk
failure correction,” Proc. 3rd USENIX Conf on File and Storage Technology (F.AS7 04), March 31-April 2,2004, San Francisco.
Crawford, J., and P. Gelsinger [1988]. Programming the 80386, Sybex Books, Alameda,Calif.
Culler, D. E., J. P. Singh, and A. Gupta [1999]. Parallel Computer Architecture: A Hardare/Softare Approach, Morgan
Kaufmann, San Francisco.
Curow, H. J., and B. A. Wichtoann [1976]. "A synthetic benchmark,” The Computer J. 19:1, 43-49.
Cvetanovic, Z., and R. E. Kessler [2000]. "Performance analysis of the Alpha 21264- based Compaq ES40 system." Proc. 27th
Anmsal Int Y. Simposium on Computer Architecture (ISC.A), June 10-14, 2000., Vancouver, Canada, 192-202.
Dally, W. J. [1990]. "Performance analysis of k-ary n-cube interconnection nerworks," IEEE Trans. on Computers 39:6 (June),
775-785.
Dally, W. J. [1992]. “Virtual channel flow control." IEEE Trans. on Paralle and Distibuted Systems 3:2 (March), 194-205.
Dally, W. J. [1999]. “Interconnect limited VLSl architecture," Proc. of the International Interconnec! Technology Conference,
May 24-26, 1999, San Francisco.
Dally, W. J.,and C. I. Seitz [1986]. *The torus routing chip.” Distributed Computing 1:4, 187-196.
Dally, W.J., and B. Towles [2001]. “Route packets, not wires: On-chip interconnection networks,” Proc. 38th Design Automation
Conference, June 18-22,2001, Las Vegas.
Dally, W. J., and B. Towles [2003]. Principles and Practices of Inlerconnection Networks, Morgan Kaufimann, San Francisco.
Darcy, J. D., and D.Gay [1996]. “FLECKmarks: Measuring tloating point performance using a ful IEEE compliant arithmetic
benchmark,” CS 252 class project, University of California, Berkeley (see HTTP. CS.Berkeley. EDU/darcy/Projects/es2.52/).
Darley, H. M.et al. [1989]. "Floating Point/Integer Processor with Divide and Square Root Functions,” U.S. Patent 4,878, 190,
October 31.
Davidson, E. S. [1971]. "The design and control of pipelined function generators” Proc. IEEE Conf. on Systems, Networks, and
Computers, January 19-21, 1971, Oaxtepec, Mexico, 19-21.
Davidson, E. S., A. T. Thomas, L. E. Shar, and J. H. Patel [1975]. “Effective control for pipelined processors,”" Proc. IEEE
COMPCON, February 25-27, 1975, San Francisco, 181-184.
参考文献
525
Duato, J., and T. M. Pinkston[2001]. “A general theory for deadlock-free adaptive routing using a mixed set of resources,” IEEE
Tans.on Parallel and Distributed Systems 12:12 (December), 1219-1235.
Duato, J., S. Yalamanchili,and L. Ni[2003].Interconnection Networks:An Engineering Approach, 2nd printing, Morgan
Kautmann, San Francisco.
Duato, J., 1. Johnson, J. Flich, F. Naven, P. Garcia, and T. Nachiondo [20058]. “A new scalabile and cost-effective congestion
management strategy for lossless mnultistage interconection networks,” Proe. 11th Int'. Symposium on High-Performance
Computer Architecture, February 12-16,2005, San Francisco.
Duato, J., O. Lysne, R. Pang, and T. M. Pinkston [2005b]. “Part I: A theory for deadlockfiree dynamic reconfiguration of
interconnection netw orks,” JEEE Trans. on Parallel and Distributed S)ystems 16:5 (May), 412-427.
Dubois, M., C. Scheurich, and F. Btiggs [1988]. “Synchronization, coherence, and event ordering,” IEEE Computer 21:2
(February,9-21.
Dunigan, W., K. Vetter, K. Wite, and P. Worley [2005]. "Performance evaluation of the Cray X1 distributed shared memnory
architecture,” IEEE Micro January/February, 30-40.
Eden, A., and T. Mudge [1998]. "The YAGS branch prediction scheme.” Proc. of the 3I stAnnual ACMIEEE IntY. Sympashum
on Microarchitecture, November 30-December 2, 1998, Dallas, Tex., 69.80.
Edmondson, J. H., P.I. Rubinfield, R. Preston, and V. Rajagopalan [1995]. “Superscalar instruction execution in the 21164 Alpha
microprocessor,” IEEE Micro 15:2, 33-43.
Eggers,S. [1989]. "Simulation Analysis of Data Sharing in Shared Memory Multiprocessors,”Ph.D. thesis, University of
California, Berkeley.
Elder, J., A. Gottlieb, C.K. Kruskal, K. P.McAuliffe, L. Randolpb, M. Snir, P. Teller, and J. Wilson [1985]. "Issues related to
MIMD shared-memory computers: The NYU Ultracomputer approach” Proc. 12th Anrual Int 'Y. Symposium on Computer
Architecture (ISCA), June 17-19, 1985, Boston, Mass. 126-135.
Ellis, J.R. [1986]. Bulldog: A Compiler for VL.IW Architectures, MIT Press, Cambridge, Mass.
Emer, J. S., and D. W. Clark [1984]. "A characterization of processor perforance in the VAX-11/780,"Proc. 1Ith Anal Int Y.
Symposium on Computer Architecture (ISCA), June S-7, 1984, Ann Arbor, Mich., 301-310.
Enriquez, P. [2001]. ^What happened to my dial tone? A study of FCC service disrwption reports." poster, Richard Topia
Symposium on the Celebration of Diversity in Computing, October 18-20, Houston, Tex.
Erlichson, A., N. Nuckolls, G. Chesson, and J. L. Hennessy [1996]. “SoftFLASH: Analyzing the performance of clustered
distributed virtual shared memory,"Proc. Seventh Int '. Conf. on Architectural Support jor Programming Languages and
Operating Systems (ASPL.OS), October 1-5, 1996, Cambridge, Mass., 210-220.
Esmaeilzadeh, H., T.Cao, Y. Xi, S. M. Blackburn, and K. S. McK inley [2011]. "Looking Back on the Language and Hardware
Revolution: Measured Power, Performance, and Scaling." Proc. 16th Int'Y. Conf: on Architectural Support for Programming
Languages and Operating Systems (A.SPLOS), March $-11, 2011, Newport Beach, Calif.
Evers, M., S.J. Patel, R. s. Chappell, and Y. N. Patt [1998]. “An analysis of corelation and predictability: What makes two-level
branch predictors work." Proc. 25th Annual Int '. Symposium on Computer Architecturre (ISCA), July 3-14, 1998, Barceloma,
Spain, $2-61.
Fabry, R. s. [1974].“Capability based addressing," Communications ofthe ACM 17:7 (July), 403-412.
Falsafi, B., and D.A. Wood [1997]. “Reactive NUMA: A design for unifying S-COMA and CC-NUMA," Proc. 24th Anmtea/ Int '!.
Symposium on Computer Anchitecture (ISCA), June 2-4, 1997, Denver, Colo.,229-240.
Fan, X., W. Weber, and L. A. Barroso [2007]. *Power provisioning for a warehouse-sized corputer,” Proc. 34th Anmual Int'!.
Symposium on Computer Architectue (ISCA), June 9-13, 2007, San Diego, Calif.
Farkas,K. I., and N. P. Jouppi [1994]. "Complexity/performance trade-ofs with nonblocking loads," Proc. 21st Anmual Int 7.
Symposium on Computer Architecture (ISCA), April 18-21, 1994, Chicago.
Farkas, K. I., N. P.Jouppi, and P.Chow [1995]. "How useful are non-blocking loads, stream buffers and speculative execution in
multiple issue processors?,” Proc. Firs! IEEE Symposium on High-Performance Computer Architecture, January 22-25, 1995,
Raleigh, N.C.,78-89.
Farkas, K. L., P. Chow, N. P. Jouppi, and Z. Vranesic [1997]. "Memory-system design considerations for dynamically-scheduled
processors,” Proc. 24th Anmual Int '. Symposium on Computer Architecture (ISCA), June 2-4, 1997, Denver, Colo., 133-143.
Fazio, D. [1987]. "Tt's really much more fiun building a supercomputer than it is simply inventing one,” Proc. IEEE COMPCON,
526
参考文献
February 23-27,1987, San Francisco, 102-105.
Fisher, J. A. [1981]. "Trace scheduling: A technique for global microcode compaction," IEEE Tans. on Computers 30:7 (July),
478--490.
.Fisher, J. A. [1983]. “Very long instruction word architectures and ELI-512,"1Oth Anmua/ Int'.Sywposium on Computer
Architecture (ISCA), June 5-7, 1982, Stockholm, Sweden, 140-150.
Fisher, J.A., and S. M. Freudenberger [1992]. “Predicting conditional branches fom previous runs of a program,”Proc. Fifth
Int Y.Conf on Architectural Support for Programming Languages and Operating Systems (ASPLOS), October 12-15, 1992,
Boston, Mass., 85-95.
Fisher, J. A., and B. R. Rau [1993]. Joural ofSupercomputing, January (special issuc).
Fisher, J. A., J. R. Ellis, J. C. Ruttenberg, and A. Nicolau [1984] “Parallel processing: A smart compiler and a durb processor,”
Proc. SIGPL.AN Conf on Compiler Construction, June 17-22, 1984, Montreal, Canada, 11-16.
Flemming, P. J., and J. J. Wallace[1986]. "How not to lie with statistics: The correct way to summarize benchmarks results,”
Communications ofthe ACM 29:3 (March), 218-221.
Flyon, M. J.[1966]. “Very bigh-speed computing systems,”" Proc. IEEE54:12 (December), 1901-1909.
Forgie, J.W. [1957]. "The Lincoh TX-2 input-output systemn." Proc. Wesfern Joint Computer Conference (February), Institte of
Radio Engineers, Los Angeles, 156-160.
Foster, C. C., and E. M. Riseman [1972]. "Percolation of code to enhance parallel dispatching and execution”IEEE Trans.om
Computers C-21:12 (December), 1411-1415.
Frank,S.J.[1984]. "Tightly coupled multiprocessor systems speed memory access time," Blectronics 57:1 (January), 164-169.
Freimnan, C. V.[1961]."Statistical analysis of certain binary division algorithms,” Pror. IRE 49:1, 91-103.
Friesenborg, S. E., and R J. Wicks [1985]. DASD Expectations: The 3380, 3380-23, amnd MYS/XA, Tech. Bulletin GG22-9363-02,
IBM Washington Systems Center, Gaithersburg, Md.
Fuller, S. H., and W. E. Bur [1977]. “Measurement and evaluation of alternative computer architectures,”Computer 10:10
(October), 24-35.
Furber, S. B. [1996].ARM System Archilecture, Addison-Wesley, Harlow, England (see www.cs.man.ac.uk/amulet/publications/
bocks/ARMsysArch).
Gagliardi, U. O. [1973]. “Report of workshop 4-software-related advances in computer hardware,"Proc. Symposium on the
High Cost of Software, September 17-19, 1973, Monterey, Calif., 99-120.
Gajski, D., D. Kuck, D. Lawrie, and A. Sameh [1983]. “CEDAR—2 large scale multiprocessor,” Proc. Int1. Conf. on Parallet
Processing (ICPP), August, Columbus, Ohio, 524-529.
Gallagher, D.M., W. Y. Chen, S. A. Mahlke, J. C. Gyllenheal, and W. W.Hiwu [1994]. "Dynamic mnemory disambiguation using
the memory conflict bufTfer," Proc. Sixuth Int'Y.Conf. on Architectural Support for Programming Languages and Operating
Systems (ASPLOS), October 4-7, Santa Jose, Calif., 183-193.
Galles, M. [1996]. “Scalable pipelined interconnect for distributed endpoint routing: The SGI SPIDER chip." Proc. IEEE HOT
Inierconnects"96, August 15-17, 1996, Stanford University, Palo Alto, Calif.
Game, M.,and A. Booker [1999]."CodePack code compression for PowerPC processors," MicroNews, S:1,www.chips.ibm.com/
micronews/vol5_no l/codepack.htm!.
Gao, Q.S. [1993]. "The Chinese remainder theorem and the prime memory system," 20th Annual Ind '. Symposium on Computer
Architecture (ISCA), May 16-19, 1993, San Diego, Calif. (Computer Architecture News 21:2 (May), 337-340).
Gap. [2005]. “Gap Inc. Reports Third Quarter Earnings,"http://gapinc.com/public/ documents/PR_Q405Eammings Feb2306.pdf.
Gap.[2006]. "Gap Inc. Reports Fourth Quarter and Full Year Eacnings,"http://gqpinc.com./mublic/cocuments/Q32005PressRelease_
Final22.pdf.
Garner, R., A. Agarwal, F. Briggs, E. Brown, D. Hough, B.Joy, S. Kleiman, S. Muchnick, M. Narjoo, D. Patterson, J. Pendleton,
and R. Tuck [1988]. “Scalable processor architecture (SPARC)," Proc. IEEE COMPCON, February 29-March 4, 1988, San
Francisco,278-283.
Gebis, J., and D. Patterson [2007]. “Embracing and extending 20th-century instruction set architectures, " IEEE Computer 40:4
(April), 68-75.
Gee,J.D., M. D. Hill, D.N. Pnevmatikatos, and A. J. Smith [1993]. "Cache performance of the SPEC92 benchmark suite," IEEE
Micro 13:4 (August), 17-27.
参考文献
527
Gehringer, E.F, D.P. Siewiorek, and Z. Segall [1987]. Parallel Processing: The Cm*Experience, Digital Press, Bedford, Mass.
Gharachorloo, K., A. Gupta, and J. L. Hennessy [1992].“Thiding memory Lateney using dynamic scheaduling in shared-memory
multiprocessors,” Proc. 19th Anual Int 7. Sympasiam on Computer Architecmere (LSC.A), May 19-21, 1992, Gold Coast, Austrzlia.
Gharachorloo, K., D. Lenoski, J. Laudon, P. Gibbons, A. Gupta, and J. L. Hennessy [1990]. “Memory consistency and event
ordering in scalable shared-mnemory mnuliprocessors,” Proc. 17th Anmual Int '. Symposiu on Computer Architecture (ISC.A),
May 28-31,1990, Seattle, Wash,1$-26.
Ghemawat, S., H.Gobioff,and S.T.Leung[2003]. "The Google file system,"Proc. I9th ACM Symposium on Operating Systems
Principles, October 19-22, 2003, Bolton Landing, N.Y.
Gibson, D. H. [19671. “Considerations in block-oriented systems desigo," AFIPS Conf. Proc. 30, 75-80.
Gibson, G. A. [1992]. Redndant Disk Arrays: Reliable, Parallel Secondary Storage, ACM Distinguished Dissertation Series,
MIT Press, Cambridge, Mass.
Gibson, J.C.[1970]. *The Gibson mix." Rep.TR. 00.2043, IBM Systems Development Division, Poughkeepsie, N.Y. (research
done in 1959).
Gibson, J., R. Kunz, D. Ofelt, M. Horowitz, J. Hennessy, and M. Heinrich [2000].
“FLASH vs. (simulated) FLASH:Closing the simulation loop”Proc. Ninth Int '. Conf:on Architectural Support for
Programming Lamguages umd Operating Systems (ASPLOS), November 12-15, Cambridgc, Mass,, 49-58.
Glass, C. J., and L. M. Ni [1992]. *The Turn Model for adaptive routing," 19th Ammual Int'1. Symposium on Compuser
Anchitecture (USCA), May 19-21, 1992, Gold Coast, Australia.
Goldberg, D. [1991]. ^What every computer scientist should know about floating-point arithmetic." Computing Surveys 23:1,
5-48.
Goldberg, 1. B. [1967]. "27 bits are not enough for 8-digit accuracy,” Communications of theACM10:2, 105-106.
Goldstein, S. [1987]. Storage Performance-An Eight Year Outlook, Tech. Rep. TR 03.308-1, Santa Teresa Laboratory, IBM
Santa Teresa Laboratory, San Jose, Calif.
Goldstine, H. H. [1972]. The Computer: From Pascal to von Neumann, Princeton University Press, Princeton, N.J.
Gonzdlez, J., and A. Gonzflez [1998]. “Limits of instruction level parallelism with data speculation,” Proc. Vector and Paralle!
Processing (VECPAR) Conf., June 21-23, 1998, Porto, Portugal, 585-598.
Goodman, J. R.[1983]. "Using cache memory to reduce processor meraory traffic,”Proc. 10th Anmual Int '.Sympasium on
Computer Architecnure (ISCA), June 5-7, 1982, Stockholm, Sweden, 124-131.
Goralski, W.[1997]. SONET: A Guide to Synchronous Optical Network, McGraw-Hill, New York.
Gosling, J. B.[1980]. Design of Aritmetic Units for Digisal Computers, Springer-Verlag, New York.
Gray, J. [1990]. “A censua of Tandem system availability between 1985 and 1990," EEE Tans.on Relicbility, 39:4 (October)
409-418.
Gray, J. (ed.) [1993]. The Benchmark Handbook for Database ard Transaction Processing Systems, 2nd ed., Morgan Kaufrann,
San Francisco.
Gray, J.[2006]. Sort benchmark home page, http://sortbenchmark.org/.
Gray, J. and A. Reuter [1993]. Transaction Processing: Concept and Techniqgues, Morgan Kaufiman, Sam Francisco.
Gray, J, and D. P.Siewiorek [1991]. "Fiigh-availability computer systems,” Computer 24:9 (September), 39-48.
Gray, J., and C. van Ingen [2005]. Empirical Measurements of Disk Failure Rates and Error Rates, MSR-TR-2005-166,
Microsoft Research, Redmond, Wash.
Greenberg, A., N. Jain, S. Kandula, C. Kim, P. L.ahiri, D.Maltz, P. Patel, and S. Sengupta [2009]."VL2: A Scalable and Flexible
Data Center Network,” in Proc. ACM SIGCOMM, August 17-21, 2009, Barcelona, Spain.
Grice, C., and M. Kanellos [2000]. “Cell phone industry at crossroads: Go high or low2,” CNET Newxs, August 31,
technevs.nerscape.com/news/0-1004-201-2518386-0.htm/?tag=st.ne.1002.4gfsf:
Groe, J. B., and L. E. Larson [2000]. CDMA Mobile Radio Design, Artech House, Boston.
Gurther, K. D. [1981]. “Prevention of deadlocks in packet-switched data transport systems” IEEE Trans.on Communications
COM-29:4 (April), 512-$24.
Hagersten, E., and M. Koster [1998]. "WildFire: A scalable path for SMPs," Proc. Fijfth Int'. Symposium on High-Performance
Computer Architecture, January 9-12, 1999, Otlando, Fla.
Hagersten, E., A. Landin, and S. Haridi [1992]."DDM-a cache-only mernory anchitecture," IEEE Comguter 25:9 (Seplember), 44-54.
528
参考文献
Hamacher, V. C., Z. G. Vranesic, and S. G. Zaky [1984]. Computer Organization, 2nd ed., McGraw-Hill, New York,
Hamilton, J.[2009]."Data center networks are in my way," paper presented at the Stanford Clean Slate CTO Summit, October 23,
2009(hotp://mwdirona.comjrh/TalksAndPapers/James familton_ CleanSlateCTO2009.pdf).
Hamilton, J.[2010]. "Cloud corputing eoonomies of scale," paper presented at the A PS Workshop on Gernomic and Cloud Conpuding,
June 8, 2010, Seattle, Wash. (http://mudirona.com/jirh/TalksAndPapers.lames Hlamilton_GenomicsClowd20100608.pdf).
Handy, J.[1993]. The Cache Memory Book, Academic Press, Boston.
Hauck, E. A., and B. A. Dent [1968]. “Burroughs' B6500/B7500 stack mnecbanism,” Proc. AFIPS Spring Joint Computer Conf.,
April 30-May 2, 1968, Atlantic City, N.J,, 245-251.
Heald, R., K. Aingaran, C. Amir, M. Ang, M. Boland, A. Das, P. Dixit, G. Gouldsberry, J. Hart, T. Horel, W.J. Hsu, J. Kaku, C.
Kim, S. Kim, F.Klass, H. Kwan, R. Lo, H. Mclntyze, A. Mehta, D. Murata, S. Nguyen, Y.-P. Pai, S. Patel, K. Shin, K. Ta, S.
Vishwanthaiah, J. Wu, G. Yee, and H. You [2000]. "Implementation of thirdgeneration SPARC V964-b microprocessor,”
ISSCC Digest of Technical Papers, 412-413 and slide supplement.
Heinrich, J.[1993]. MIPS R4000 User 's Manual, Prentice Hall, Englewood Cliffs, N.J.
Henly, M.,and B.McNatt [1989]. DASD I/O Characteristics: A Comparison ofMVSto Y3,” Tech. Rep. TR 02.1550 (May), IEM
General Products Division, San Jose, Calif.
Hennessy, J.[1984]. "VLSI processor architecture," IEEE Trans. on ComputersC-33:11 (December), 1221-1246.
Hennessy, J. [1985]. “VLSI RISC processors.” VL.SI Systems Destgn 6:10 (October), 22-32.
Hennessy, J., N. Jouppi, F. Baskett, and J. Gill [1981]. “MIPS: A VLSI processor architecture, " in CMU Corference on VI.SI
Systems and/ Compulations, Computer Science Press, Rockville, Md.
Hewlett-Packard. [1994]. PA-RISC 2.0 Architecture Reference Manual, 3rd ed., Hewlett- Packard, Palo Alto, Calif.
Hewlett-Packard.[1998]. “HP's‘SNINES:SMINUTES’Vision Extends Leadership and Redefines High Availabiity in
Mission-Critical Environments," February 10,www.fure.enterprisecomputing.hp.com/ia64/news /Snines_vision_pr.html.
Hill, M. D.[1987].“Aspects of Cache Memory and Instruction Buffer Per formance," Ph.D. tbesis, Tech. Rep. UCB/CSD 87/381,
Computer Scicace Division, University of California, Berkeley.
Hill, M. D.[1988]. “A case for direct mapped caches."Computer 21:12 (December), 25-40.
HilL, M. D. [1998]. "Multiprocessors should support simple memory consistency models,”IEEE Compuler 31:8 (August), 28-34.
Hillis, W. D.[1985]. The Connection Multprocessor, MIT Press, Cambridge, Mass.
Hillis, W. D. and G. L. Steele [1986]. "Data parallel algorithrs,"Communications ofthe ACM 29:12 (December), 1170-1183.
(htp://doi.acm.org/10.1145/7902.7903).
Hinton, G., D. Sager, M. Upton, D. Boggs, D.Carmean, A. Kyker, and P. Roussel [2001]. "The microarchitecture of the Rentiuum
4 processor” Intel Technology Joumal, February.
Hintz, R. G., and D. P. Tate [1972]. "Control data STAR-100 processor desig,” Proc. IEEE COMPCON, September 12-14,1972,
San Francisco,1-4.
Hirata, H., K. Kimnura, S. Nagamine, Y. Mochizuki, A. Nishimura, Y. Nakase, and T. Nishizawa [1992]. "An elementary
processor architecture with simultaneous instruction issuing from multiple threads,"Proc. 19th Anmual Int '.Symposium on
Computer Architecture (ISCA), May 19-21, 1992, Gold Coast, Australia, 136-145.
Hitachi. [1997]. SuperHI RISC Engime SH7700 Series Programming Manual, Hitachi, Santa Clara, Calif. (see ww.haksp.hitcchi.
com/tech_prod/ and search for title).
Ho, R., K. W.Mai, and M. A. Horowitz [2001]. *The future of wires, " Proe. ofthe IEEE 89:4 (April), 490-504.
Hoagland, A. S. [1963]. Digital Magnetic Recording, Wiley, New York.
Hockney, R. W., and C. R. Jesshope [1988]. Parallel Computers 2: Architectures, Programming and Algorithms, Adam Hilger,
Ltd., Bristol, England.
Holland, J. H.[1959]. “A universal corputer capable of executing an arbitrary number of subprograms simnultaneously,” Proc.
Easi Joint Computer Conf. 16, 108-113.
Holt, R. C. [1972]. “Some deadlock properties of computer systems," ACM Computer Surveys 4:3 (September), 179-196.
Hopkins, M. [2000]. “A critical look at IA-64: Massive resources, massive ILP, but can it deliver?" Microprocessor Repor.,
Hord, R. M. [1982]. The Iliac-I, The First Supercompuler, Computer Science Press, Rockville, Md.
Horel,T., and G. Lauterbach [1999]. “UltraSPARC-IIl: Designing third-generation 64-bit performance,” IEEE Micro 19:3
参考文献
529
(May-June),73-85.
Hospodor, A. D., and A. S. Hoagland [1993]. “The changing nature of disk controllers." Proc. IEEE 81:4 (April), 586-594.
Helzie, U. [2010]. "Brawny cores still beat wimpy cores, most of the time,” IEEE Micro 30:4 (July/August).
Hristea, C., D. Lenoski, and J. Keen [1997]. "Measuring memory hierarchy Performance of cache-coherent multiprocessors using
micro benchmarks," Proc. ACM/IEEE Conf. on Swpercomputing, November 16-21, 1997, San Jose, Calif.
Hsu, P. [1994]. "Designing the TFP mnicroprocessor”IEEE Micro 18:2(April), 2333.
Huck, J.et al. [2000]. *Introducing the IA-64 Architecture" IEEE Micro, 20:5 (September-October), 12-23.
Hughes, C.J., P. Kaul, S. V.Adve, R. Jain, C. Park, and J. Srinivasan [2001]. “Variability in the execution of multimedia
applications and implications for architecture," Proc. 28th Anmual Inf'. Sympostum on Computer Architecture (ISCA), June
30-July 4, 2001, Goteborg, Sweden, 254-265.
Hwang, K. [1979]. Computer Arithmetic: Principles, Architecture, and Desig, Wiley, New Yotk.
Hwang, K. [1993]. Advanced Computer Architectre and Parallel Programming, McGraw-Hill, New York.
Hwu, W.-M., and Y. Patt [1986]. “HPSm, a high performance restricted data flow architecture having minimum functionality,”
Proc. 13th Annual Int'.Smmposium on Computer Architecture (ISCA), June 2-5, 1986, Tokyo, 297-307.
Hwa, W. W., S. A. Mahlke, W.Y.Chen, P. P.Chang, N. J. Warter, R. A. Bringmann, R. O. Ouellete, R. E. Hank, T. Kiyohara, G.
E. Haab, J.G. Holn, and D. M. Lavery [1993]. “The superblock: An effective technique for VLIW and superscalar
compilation,"J. Supercomputing 7:1, 2 (March), 229-248.
IBM. [1982]. The Ecomomic Vale of Rapid Response Time, GE20-0752-0, IBM, White Plains, N.Y., 11-82.
IBM. [1990]. “The IBM RISC System/6000 processor" (collection of papers), IBM J. Research and Development 34:1 (January).
IBM.[1994]. The PowerPC Architecture, Morgan Kaufmann, San Francisco.
IBM.[2005]. *Blue Gene" IBM J. Research and Development, 49:2/3 (special issue).
IEEE. [1985]. “TEEE standard for binary floating-point arithmetic,” SIGPLAN Notices 22:2, 9-25.
IEEE. [2005]. "Intel virtualization technology, computer,” IEEE Computer Society 38:5 (May), 48-56.
IEEE. 754-2008 Working Group.[2006]. “DRAFT Standard for Floating-Point Arithmetic 754-2008," http://d.doi.org/10.1109/
IEEESTD.2008.4610935.
Imprimis Product Specjfication, 97209 Sabre Disk Drive IPI-2 Inteface 1.2 GB, Document No. 64402302, Imprirnis, Dallas, Tex.
InfiniBand Trade Association.[2001]. Infin Band Architecture Specifications Release 1.0.a, www.infinibandta.org.
Intel. [2001]. “VUsing MIMX Instructions to Convert RGB to YUV Color Conversion." cedar.intel.com/cgi-bi/tds.dl/conienstcontent jsp?
cntKey=Legacy:irm_AP548_9996& cntTpe=IDS_EDITORIAL
Internet Retailer. [2005]. "The Ghap launches a new site after two weeks of downtime," Internet® Retailer, Septemnber 28,
http://ww.internetretailer.com/2005/09/28/thegap-launches-a-new-site-after-mwo-weeks-of-downtime.
Jain, R. [1991]. The Art of Computer Systems Perforance Analysis: Techniques for Experimental Design, Measurement,
Simulation, and Modeling, Wiley, New York.
Jantsch, A., and H. Tenhunen (eds.) [2003]. Networks on Chips, Kluwer Academic Publishers, The Netherlands.
Jimenez, D. A., and C. Lin [2002]. "Neural metbods for dynamic branch prediction,” ACM Trans.on Computer Systems 20:4
(November),369-397.
Johnson, M.[1990].Superscalar Microprocessor Design, Prentice Hall, Englewood Cliffs, N.J.
Jordan, H. F.[1983]. “Performance measurements on HEP—a pipelined MIMD computer," Proc. 1Oth Annual Int '.Symposium
on Compuler Architecture (ISCA), June S-7, 1982, Stockholm, Sweden,207-212.
Jordan, K. E. [1987].“Performance comparison of large-scale scientific processors: Scalar mainframes, mainframes with vector
facilities, and supercomputers," Computer 20:3 (March), 10-23.
Jouppi, N. P.[1990]. "Improving direct-rapped cache performance by the addition of a stall fully-associative cache and prefetch
buffers,” Proc. I7th Anmual Int 'Y. Symmposium on Computer Architecture (ISCA), May 28-31, 1990, Seattle, Wash., 364-373.
Jouppi, N. P. [1998]. “Retrospective: Improving direct-mapped cache performance by the addition of a small fully-associative
cache and prefetch buffers," 25 Years ofthe International Symposia on Computer Architecture (Selected Papers), ACM, New
York,71-73.
Jouppi, N. P., and D. W. Wall [1989]. "Available instruction-levei parallelisr for superscalar and superpipelined processors,”
Proc. Third Int 1. Conf. on Architectural Support for Programming Languages and Operating Sustems (ASPLOS), April 3-6,
1989, Boston, 272-282.
530
参考文献
Jouppi, N. P., and S. J. E. Wiiton [1994]. “Trade-offs in two-level on-chip caching"Proc. 2Ist Amnual Int 'Y.Symposium on
Computer Architecture (ISC.A), April 18-21, 1994, Chicago, 34-45.
Kaeli, D. R., and P.G. Emma [1991]. "Branch history table prediction of moving target branches due to subroutine returs," Proc.
I8th Anrual Int Y. Symposium on Coimputer Architectre (ISCA), May 27-30, 1991, Toronto, Canada, 34-42.
Kahan, J.[1990]. “On the advantage of the 8087's stack," unpublished course notes, Computer Science Division, University of
California, Berkeley.
Kahan,W.[1968]. *7094-II system support for numerical analysis.” SHARE Secretarial Distribution SSD-159, Department of
Computer Science, University of Toronto.
Kahaner, D. K. [1988]. "Benchmarks for 'real' programs.”SLAM News, November.
Kabn, R. E. [1972]. "Resource-sharing computer communication networks,” Proc. IEEE 60:11 (November), 1397-1407.
Kane, G. [1986]. MIPS R2000 RJSC Architecture, Prentice Hall, Englewood Cliffs, N.J.
Kane, G. [1996]. PA-RISC 2.0 Architecture, Prentice Hall, Upper Saddle River, N.J.
Kane, G., and J. Heinrich [1992]. MIPS RISC Architecrure, Prentice Hlall, Englewood Clifis, N.J.
Katz, R. H., D. A. Patterson, and G. A. Gibson [1989]. "Disk system architectures for high perfommance computing'" Proc. IEEE
77:12 (December),1842-1858.
Keckler, S. W., and W. J. Dally [1992]. "Processor coupling: Iategrating compile time and runtime scheduling for parallelism."
Proc. I9th Anmual Inl '. Symposium on Computer Architecture (ISCA), May 19-21, 1992, Gold Coast, Australia, 202-213.
Keller, R. M. [1975]. “Look-ahead processors," ACM Computing Surveys 7:4 (December), 177-195.
Keltcher, C. N., K. J. McGrath, A. Ahmed, and P. Conway [2003]. "The AMD Opteron processor for mnultiprocessor servers,”
IEEE Micro 23:2 (March-April), 66-76 (dx.doi.org/10.1109.MM.2003.119116).
Kembel, R. [2000]. *Fibre Channel: A comprehensive introduction," Internet Week, April.
Kermani, P., and L. Kleinrock[1979]. "Virtual Cut-Through: A New Corputer Communication Switching Technique,”
Computer Networks 3 (January), 267-286.
Kessler, R. [1999]. "The Alpha 21264 mnicroprocessor," IEEE Micro 19:2 (March/April) 24-36.
Kilbum, T., D. B. G. Edwards, M. J. Lanigan, and F. H. Sumner [1962]. "One-level storage system,” IRE Trans. on Electronic
Computers EC-11 (April) 223-235. Also appears in D. P. Siewiorek, C.G. Bell, and A. Newell, Computer Strctures:
Principles and Examples, McGraw-Hill, New York, 1982, 135-148.
Killian, E.[1991]. "MIPS R4000 technical overview-64 bits/100 MIHIz or bust," Hot Chips III Symposium Record, August 26-27,
1991, Stanford University, Palo Alto, Calif, 1.6-1.19.
Kim, M. Y.[1986]. "Syachronized disk interleaving,"EEE Trans. on Computers C-35:11 (November), 978-988.
Kissell, K. D. [1997]. "MIP$16: High-density for the embedded market,”Proc. Real Time Systems 97, June 15, 1997, Las Vegas,
Nev.(see www.sgi.com/MIPS/arch/MIPS16/ MIPS16. whitepaper.pdf).
Kitagawa, K., S. Tagaya, Y. Hagihara, and Y. Kanoh [2003]. “A hardware overview of SX- 6 and SX-7 supercomputer,” NEC
Research & Development J. 44:1 (January), 2-7.
Knuth, D.[1981]. The Art of Computer Programming, Vol. I,2nd ed., Addison-Wesley, Reading, Mass.
Kogge, P. M. [1981]. The Archiecture ofPipelined Computers, McGraw-Hill, New York.
Kohn, L., and S.-W.Fu [1989]. "A 1,000,000 transistor rnicroprocessor,” Proc. of IEEE Int'. Symposium on Solid State Circuits
(ISSCC), February 15-17, 1989, New York, 54-55.
Kohn, L., and N. Margulis [1989]. "Introducing the Intel 1860 64-Bit Microprocessor,” EEE Micro, 9:4 (July), 15-30.
Kontothanassis, L., G. Hunt, R. Stets, N. Hardavellas, M. Cierniak, S. Parthasarathy, W. Meira, S. Dwarkadas, and M. Scott
[1997]. "vM-based shared memory on lowlatency, remote-memory-access networks." Proc. 24th Anmual Int '.Sympasium on
Computer Architectre (LSCA), June 2-4, 1997, Denver, Colo.
Koren, 1. [1989]. Computer Arithmetic Algorithms, Prentice Hall, Englewood Cliffs, N.J.
Kozyrakis, C. [2000]. “Vector IRAM: A media-oriented vector processor with embedded
DRAM.," paper presented at Hot Chips 12, August 13-15, 2000, Palo Alto, Calif, 13-15.
Kozyrakis, C., and D. Patterson, [2002]. “Vector vs. superscalar and VLIW architectures for embedded multimedia benchmarks,”
Proc. 3Sth Anmual Ine'1. Symposimm on Microarchitecture (MICRO-35), November 18-22, 2002, Istanbul, Turkey.
Krof, D. [1981]. “Lockup-free instruction fetch/prefetch cache organization,"Proc. Eighth Ammual Int'1. Symposium on
Computer Architecture (ISCA), May 12-14, 1981, Minneapolis, Minn.,81-87.
参考文献
531
Smposia on Computer Architecture (Selected Papers), ACM, New York, 20-21.
Kuck, D. P.P. Budnik, S.-C.Chen, D.H. Lawrie, R, A. Towle, R. E. Strebendt, E. W. Davis, Jr., J. Han, P. W. Kraska, and Y.
Muraoka [1974]. “Measurements of parallelism in ordinary FORTRAN programs,”Computer 7:1 (January), 37-46.
Kubn, D. R. [1997]. “Sources of failure in the public switched telephone network,”EEE Computer 30:4 (April), 31-36.
Kumar, A. [1997] "The HP PA-8000 RISC CPU," IEEE Micro 17:2 (March/April), 27-32.
Kuninatsu, A., N. Ide, T. Sato, Y. Endo, H. Murakami, T. Kamei, M. Hirano, F. Ishihara, H. Tago, M. Oka, A. Ohba, T. Yutaka,
T. Okada, and M. Suzuoki [2000]. *Vector unit architecture for emotion synthesis,” IEEE Micro 20:2 (March-April), 40-47.
Kunkel, S. K., and J.E.Smith [1986]. "Optimal pipelining im supercomputers,” Proc. 13th Annal Int'.Symposium on Computer
Architecture (ISCA), June 2-5, 1986, Tokyo, 404-414.
Kurose, J. F., and K. W. Ross [2001]. Computer Nenvorking A Top-Down Approach Feaniring the Interet, Addison-Wesley, Boston.
Kuskin, J., D. Ofelt, M. Heinrich, J. Heinlein, R. Simoni, K. Gharachorloo, J. Chapin, D. Nakahira, J. Baxter, M. Horowitz, A.
Gupta, M. Rosenblum, and J. L. Hennessy [1994]. “The Stanford FLASH multiprocessor,”Proc. 21st Anmual IntY.
Symposium on Computer Architecture (ISCA), April 18-21, 1994, Chicago.
Lam,M. [1988]. “Software pipelining: An effective scheduling technique for VLIW processors,”SIGPL.AN Conf:on
Programming Language Design and Implementation, June 22-24, 1988, Atlanta, Ga., 318-328.
Lam, M.S.,and R. P. Wilson [1992]. "Limits of control flow on parallelism," Proc. I9th Anrual Int'Y. Symposium on Computer
Architecture (ISCA), May 19-21, 1992, Gold Coast, Australia, 46-57.
Lam, M. S., E. E. Rothberg,and M. E. Wolf [1991]. "The cache performance and optimizations of blocked algorithms," Prac.
Fourth Int '. Conf. on Architectural Support for Programming Languages and Operating Systems (ASPLOS, April 8-11,
1991, Santa Clara, Calit. (SIGPLAN Notices 26:4 (April), 63-74).
Lambright, D.[2000]. “Experiences in measuring the reliability of a cache-based storage system"Proc. of First Workshop on
Incustrial Experiences with Systems Software (WIESS 2000), Co-Located with the 4th Symposium on Operating Systems
Design and Implementation (OSD ), October 22, 2000, San Dicgo, Calif.
Lamport, L. [1979]. “How to make a multiprocessor computer that corectly executes multiprocess programs,” IEEE Trans.on
Computers C-28:9 (September), 241-248.
Lang, W.,J. M. Patel,and S. Shankar [2010]. “Wimpy node clusters: What about nonwimpy workloads?”Proc. Sixth
International Workshop on Data Management on New Hardware (DaMoN), June 7, Indianapolis, Ind.
Laprie, J.-C. [1985]. “Dependable computing and fault tolerance: Concepts and terminology." Proc. 15th Anmual Int'?.
Symposizim on Fault-Tolerant Computing, June 19-21, 1985, Ann Arbor, Mich., 2-11.
Larson, E. R. [1973]. “Findings of fact, conclusions of law, and order for judgment." File No. 4-67, Civ. 138, Honeywell v.
Sperry-Rand and IIlinois Scientific Development, U.S. District Court for the State of Minnesota, Fourth Division (October 19).
Laudon,J., and D. Lenoski [1997]. "The SGI Origin: A ceNLMA highly scalable server," Proc. 24th Anmual Int'!. Symposium on
Computer Architecture (ISC.A), June 2-4, 1997, Denver, Colo.,241-251.
andon J.. A. Gunta and M. Horowitz「19941. *Interleaving:A muluthreading technique targeting multiprocessors an
rorkkstations " Proc. Sixth Int', Conf. on Architectural Suppord for Programming Langages and Operatimg Sjstem
(ASPLOS), October 4-7, San Jose, Calif., 308-318.
Lauterbach, G., and T. Horel [1999]. “UltraSPARC-II: Designing third generation 64-bit performance, " IEEE Micro 19:3
(May/June).
Lazowska, E. D, J. Zahoujan, G. S. Grahar, and K. C. Sevsik [L984). Quantiative System Performance: Computer System
Analysis Using Queueing Network Modeks, Prentice Hall, Englewood Cliffs, N.J. (Although out of print, it is available online
at www.cs.washington.ed/homes/lazowska/gsp/.)
Lebeck, A. R., and D. A. Wood [1994]. "Cache profling and the SPEC benchmarks: A case study, ” Computer 27:10 (October), 15-26.
Lee, R. [1989]. “Precision architecture,” Computer 22:1 (January),78-91.
Lee, W. V.et al. [2010]. “Debunking the 100X GPU vs. CPU mnyth: An evaluation of throughput computing on CPU and GPU,”
Proc. 37th Annual Int '. Symposium on Computer Architecrure (ISCA), June 19-23, 2010, Saint-Malo, France.
Leighton, F. T. [1992]. Introduction to Parallel Algorithms and Architectures: Arrays, Trees, Hypercubes, Morgan Kaufmann,
San Francisco.
Leiner, A. L.[1954]. "System specifications for the DYSEAC," J. ACM 1:2(April), $7-81.
532 参考文献
Leiner, A. L., and S. N. Alexander [1954]. "System organization of the DYSBAC,"IRE Trans. ofElectronic Computers EC-3:1
(March),1-10.
Leiserson, C.E. [1985]. “Fat trees: Universal networks for hardware-efticient supercomputing." IEEE Trans. on Computers
C-34:10 (October),892-901.
Lenoski, D. J. Laudon, K. Gharachorloo, A. Gupta, and J.L. Hennessy [1990]. “The Stanford DASH multiprocessor." Proc. 17th
Annual Int '. Symposium on Computer Architecnre (LSCA), May 28-31, 1990, Seattle, Wesh., 148-159.
Lenoski, D., J. Laudon, K. Gharachorloo, W. D. Weber, A. Gupta, J. L. Hennessy, M. A.
Horowitz, and M. Lam [1992]. “The Stanford DASH multiprocessor,”IEEE Computer 25:3(March),63-79.
Levy, H., and R. Eckhouse [1989]. Computer Programming and Architecture: The VAX, Digital Press, Boston.
Li, K. [1988]. "TVY: A shared virtual mnemory system for parallel computing," Proc. 1988 Int'. Conf. on Parallel Processing,
Pennsylvania State University Press, University Park, Penn.
Li, S., K. Chen, J. B. Brockman, and N. Jouppi [2011]. "Performance Topacts of Nonblocking Caches in Out-of-order Processors,”
HP L.abs Tech Report HPL-2011-65 (full text available at http:/.ibrary.hp.com/techpubs/2011/Hipl-2011-65.html).
Lim, K., P. Ranganathan, J. Chang, C. Patel, T. Mudge, and S. Reinhardt [2008]. "Understanding and designing new system
architectures for emerging warehouse-computing eavironments,” Proc. 35th Annual Int'1. Symposium on Computer
Architecture (LSCA), June 21-25, 2008, Beijing, China.
Lincoln, N. R. [1982]. "Technology and design trade offs in the creation of a modern supercomputer,” IEEE Trans. on Corputers
C-31:5 (May), 363-376.
Lindholm, T., and F. Yellin [1999]. The Java Virtual Machine Specification, 2nd edl., Addison-Wesley, Reading, Mass. (also
available online atjava.swn.com/docs/ books/ vispec/).
Lipasti, M. H., and J. P. Shen [1996]. "Exceeding the dataflow limit via value prediction " Proc. 29th Int'1.Symposium on
Microarchitecture, December 2-4, 1996, Paris, France.
Lipasti, M. H., C. B. Wilkerson, and J. P. shen [1996]. "Value locality and load value prediction,”" Proc. Seventh Conf. on
Architectural Support for Programming Languages and Operating Syskems (ASPLOS), October I-S, 1996, Cambridge, Mass.,
138-147.
Liptay, J. S. [1968]. "'Structural aspects of the System/360 Model 85, Part Il: The cache," JBM Systems J. 7:1, 15-21.
Lo, J., L. Barroso, S. Eggers, K. Gharachorloo, H. Levy, and S. Parekh [1998].“An analysis of database workload per fornance on
simnultaneous mnultithreaded processors,” Proc. 25th Anmual Int '!. Siymposium on Computer Architecture (ISCA), July 3-14,
1998, Barcelona, Spain, 39-50.
Lo, J., S. Eggers, J. Emer, H. Levy, R.Stamm, and D. Tulsen [1997]. “Converting threadlevek parallelismn into instruction-level
paralielism via simultaneous multithreading." ACM Trans. on Computer Systems 15:2 (August), 322-354.
Lovett, T. and S. Thakkar [1988]. "The Symnmetry multiprocessor system ” Proc. 1988 Int '1. Conf. of Parallel Processimg,
University Park, Penn., 303-310.
Lubeck, O., J. Moore, and R. Mendez [1985]. “A benchmark comparison of three supercomputers: Fujitsu VP-200, Hitachi
S810/20, and Cray X-MP/2," Computer 18:12 (December), 10-24.
Luk,C.-K., and T. C Mowry [1999]. "Automnatic compiler-inserted prefetching for pointer-based applications," IEEE Trans. om
Computers 48:2 (February), 134-141.
Lunde, A. [1977]. “Empirical evaluation of some features of instruction set processor architecture," Communications of the ACM
20:3 (March), 143-152.
Luszczek, P. J. J. Dongarra, D. Koester, R. Rabenseifer, B. L.ucas, J. Kepner, J. McCalpin, D. Bailey, and D. Takahasbi [2005].
*'Introduction to the HPC cha llenge benchmark suite,"Lawrence Berkeley National Laboratory, Paper LBNL-57493 (April
25), repositories.calib.org/IbnI/L.BNL-57493.
Maberly, N. C. [1966]. Mastering Speed Reading, New American Library, New York.
Magenheimer, D. J., L. Peters, K. W.Pettis, and D. Zuras [1988]. "Integer multiplication and division on the HP precision
architecture,”IEEE Trans. on Computers 37:8, 980-990.
Mahlke, S. A., W. Y.Chen, W.M. Hwu, B. R. Rau, and M. S. Schlansker [1992]. "Sentinel scheduling for VL.IW and superscalar
processors," Proc. Fifith Int '. Conf on Architectura/ Support for Programming Languages and Operating Systems (ASPLOS),
October 12-15, 1992, Boston, 238-247.
Mahike, S. A., R. E. Hank, J. E. McCormick, D.I.August, and W. W.Hw [1995]. "A comparison of full and partial predicated
参考文献
533
execution support for ILP processors,”Proc. 22nd Anmual Int '. Symposium on Computer Architecture (ISCA), June 22-24,
1995,Santa Margherita, Italy, 138-149.
Major, J. B. [1989]. “Are queuing modeis within the grasp of the unwashed?.”Proc. Int'. Conf.on Management amd
Performance Evaluation of Computer Syistems, December 11-15, 1989, Reno, Nev., 831-839.
Markstein, P.W.[1990]."Computation ofelementary functions on the JBM RISC System/ 6000 processot.”BM.J.Research and
Development 34:1, 111-119.
Mathis, H. M., A. E. Mercias, J.D. McCalpin, R. J. Eickemeyer, and S.R. Kunkel [2005]. "Characterization ofthe multithreading
(SMT) efficiency in Power5," IBM.J. Research and Development, 49:4/5 (July/September), SSS-564.
MeCalpin, J. [2005]. "'STREAM: Sustainable Memory Bandwidth in High Performance Computers,” www.cs. virg inia. ed/stream.
McCalpin, J., D. Bailey, and D. Takahashi [2005]. Introcuction to the HPC Challenge Benchmark Stite, Paper LBNL-57493
Lawrence Berkeley National Laboratory, University of California, Berkeley, repositories. cdlib.org/Ibn/LBNL-57493.
MeCormick, J., and A. Knies [2002]. "A brief analysis of the SPEC CPU2000 benchrarks on the Lntel Itaniuo 2 processor,”
paper presented at Hot Chips 14, August 18-20, 2002, Stanford University, Palo Alto, Calif.
McFarling, S. [1989]. “Program optimization for instruction caches," Proc. Third Int'1. Conf. on Archilectural Support for
Progvamming Languages and Operating Systems (ASPLOS), April 3-6, 1989, Boston, 183--191.
McFarling, S.[1993]. Combining Bramch Predictors, WRL. Tecbnical Note TN-36, Digital Westem Research Laboratory, Falo
Alto, Calif.
Architecture (ISCA), June 2-5, 1986, Tokyo, 396-403.
MeGhan, E,, and M. O'Connor [1998]. "PicoJava: A direct execution engine for Java bytecode,” Computer 31:10 (October), 22-30.
McKeeman, W. M.[1967].“Language directed computer design"Proc. AFIPS Fall Joint Computer Conf., November 14-16,
1967, Wasbington, D.C.,413-417.
McMabon, F. M.[1986]. "The Livermore FORTRAN Kernels: A Computer Test ofNumerical Performance Range” Tech. Rep.
UCRL-55745, Lawrence Livermore National Laboratory, University of California, Livermore.
McNairy, C., and D. Soltis [2003]. "Itanium 2 processor mnicroarchitecture,” EEE Micro 23:2 (March-April) 44-55.
Mead, C., and L. Conway [1980]. Introchction to VL.SI Systems, Addison-Wesley, Reading, Mass.
Mellor-Crummey, J. M., and M. L. Scott [1991]. "Algorithms for scalable synchronization on shared-remory multiprocessors,”
ACM Trans.on Computer Systems 9:1 (February), 21-65.
Menabrea, L. F. [1842]. “Sketch of the analytical engine invented by Charles Babbage," Bibliothique Universelle de Geneve, 82
Menon, A., J.Renato Santos, Y. Turner, G. Jamakirtan, and W.Zwaenepoel (2005)."Diagnosing performance overheads in the
xen virtual machine environment,” Proc. First ACM/USENIXK Int '.Conf on Virtual Execution Emvironments, June 11-12,
2005, Chicago, 13-23.
Merlin, P. M., and P. J. Schweitzer [1980]. “Deadlock avoidance innstore-and-forward networks. Part I. Store-and-forward
deadlock, " IEEE Trans. on Communications COM-28:3 (March), 345-354.
Metcalfe, R. M. [1993].*Computer/network interface design: Lessons from Azpanet and Ethernet, ” EEE J. on Selected Areas in
Communications 11:2 (February), 173-180.
Metcalfe, R. M., and D.R. Boggs [1976]. “Ethernet: Distributed packet switching for local corputer networks” Communications
ofthe ACM 19:7 (July), 395-404.
Metropolis, N., J. Howlett, and G. C. Rota (eds.) [1980]. A History ofComputing in the Twentieth Centoy, Acadernic Press, New York.
Meyer, R. A., and L. H. Seawright [1970]. A virtual machine time sharing system, IBM Systems J. 9:3,199-218.
Meyers, G. J. [1978]. “The evaluation of expressions in a storage-to-storage architecture," Computer Architecture News 7:3
(October),20-23.
Meyers, G. J. [1982]. Advances in Computer Architecture, 2nd ed., Wiley, New York.
Micron. [2004]. “Calculating Memory System Power for DDR2,"http://dowmload. micron.com/fpdfipubs/ designlime/dl1004.pdf.
Micron. [2006]. "The Micron® System-Power Calculator," http://www.micron.com/ systemcalc.
MIPS.[1997].“MIPS16 Application Specific Extension Product Description.," www.sgi.com/MIPS/arch/ MIPSL6/mips 16.pdf.
Miranker, G. S.; J. Rubenstein, and J. Sanguinetti [1988]. "Squeezing a Cray-class supercomputer into a single-user package,"
Proc. IEEE COMPCON, February 29-March 4, 1988, San Francisco, 452-456.
534
参考文献
Mitchell, D.[1989]. *The Transputer: The time is now,"Computer Design (RISC suppl.), 40-41.
Mitsubishi. [1996]. Mitsubishi 32-Bil Single Chip Microcompuer M32R Family Software Manual, Mitsubishi, Cypress, Calif.
Miura, K,, and K. Uchida [1983]. "FACOM vector processing system: VP100/200,” Proc. NATO Advanced Research Workshop
on High-Speed Computing, June 20-22, 1983, Julich, West Germany. Also appears in K. Hwang, ed., "Superprocessors:
Design and applications,"IEEE (August 1984), 59-73.
Miya, E. N.[198$]."TMiultiprocessor/distributed processing bibliography," Computer Architecture Nevs 13:1,27-29.
Montoye, R. K., E. Hokenek, and S. L.Runyon [1990]. "Design of the IBM RISC System/ 6000 floating-point exccution,” JBM.J.
Research and Development 34:1, 59-70.
Moore, B., A. Padegs, R. Smith, and W. Bucholz [1987]. “Concepts of the System/370 vector architeeture,” 1 4th Amnual Int'Y.
Symposiam on Computer Architecture (ISCA), June 2-5, 1987, Pittsburgh, Penn., 282-292.
Moore, G. E. [1965]. “Cramming more components onto integrated circuits, ” Electronics, 38:8 (April 19), 114-117.
Morse, S., B. Ravenal, S. Mazor, and W.Pohlman [1980j. "Intel microprocessors-8080 to 8086," Computer 13:10 (October).
Moshovos, A., and G. S. Sohi [1997]. "Streamlining inter-operation memory communication via data dependence prediction,”
Proc. 30th Annual Int'. Symposium on Microarchitecture, December 1-3, Research Triangle Park, N.C., 235-245.
Moshovos, A., S. Breach, T. N. Vijaykumar, and G. S. Sohi [1997]. "Dynamic speculation and synchronization of data
dependences," 24th Annal Int'. Symposium on Computer Architecture (ISCA), June 2-4, 1997, Denver, Colo. Moussouris, J.
L. Crudele, D. Freitas, C. Hansen, E. Hudson, S. PrzybyIski, T. Riordan, and C. Rowen [1986]. "A CMOS RISC processor
with integrated systern functions”Proc. IEEE COMPCON, March 3-6,1986, San Francisco, 191.
Mowry, T. C., S. Lam, and A. Gupta [1992]. "Design and evaluation of a compiler algorithm for prefetching," Proc. Fifth Int'.
Conf.on Archiectural Support for Progyamming Langwages and Operating Systems (ASPL.OS), October 12-15, 1992, Boston
{SIGPLAN Notices 27:9 (September), 62-73),
MSN Money. [2005]. “Amazon Shares Tumble after Rally Fizzles," http://moneycentral .msn.com/content/CNBCTV/Articles/
Dispatches/P133695.asp.
Muchnick, s. S.[1988]. "Optimizing compilers for SPARC,” St Technology 1:3 (Summer), 64-77.
Mueller, M., L. C. Alves, W. Fischer, M. L. Fair, and L. Modi [1999]. "RAS strategy for IBM S/390 G5 and G6," IBMJ. Research
and Development 43:5-6 (September-November), 875-888.
Mukherjee, S. S., C. Weaver, J. S. Emer, S. K. Reinhardt, and T. M. Austin [2003]. "Measuring architectural vulnerability
factors," IEEE Micro 23:6, 70-75.
Murphy, B., and T. Gent [1995]. "Measuring syster and software reliability using an automated data coilleetion process," Quality
and Reliability Engineering International 11:5 (September-October), 341-353.
Myer,T. H., and I. E. Sutherland [1968]. "On the design of display processors," Communications of the ACM 11:6 (June),
410-414.
Narayanan, D., E. Thereska, A. Donnelly, S. Elnikety, and A. Rowstron [2009]. "Migrating server storage to SSDs: Analysis of
trade-ofis" Proc. 4th ACM European Conf. on Computer Systems, April 1-3, 2009, Nuremberg, Germany.
N!ational Research Council. [1997].The Evolntion of Untethered Communications, Computer Science and Telecommunications
Board, National Acadery Press, Washington, D.C.
National Storage Industry Consortium.[1998]. "Tape Roadmap." ww.nsic.org.
Nelson, V. P.[1990]. "Fault-tolerant computing: Fundamental concepts.” Computer 23:7 CTuly), 19-25.
Ngai, T.-F, and M. J. Irwin [1985]. “Regular, area-timne efficient carry-lookahead adders," Proc. Severth IEEE Symposium on
Computer Arithmetic, June 4-6, 1985, University of Illinois, Urbana, 9-15.
Nicolau, A., and J.A. Fisher [1984]. “Measuring the parallelism available for very long instruction word architectures” JEEE
Trans.on Computers C-33:11 (November),968-976.
Nikhil, R.S., G.M. Papadopoulos, and Arvind [1992]. “*T: A multithreaded messively parallel architecture,”Proc. 19th Ammual
Int'. Symposium on Compuer Architecture (ISCA), May 19-21, 1992, Gold Coast, Australia, 156-167.
Noordergraaf, L., and R.van der Pas [1999].“Performance experiences on Sun's WildFire prototype.” Proc. ACM/IEEE Conf. on
Supercomputing, November 13-19, 1999, Portland, Ore.
Nyiberg, C. R., T. Barclay, Z. Cvetanovis, J. Gray, and D. Lomet [1994]. “AlphaSort: A RISC mmachine sort.” Proc. ACM
SIGMOD, May 24-27, 1994, Minneapolis, Minn.
Oka, M., and M. Suzuoki [1999]. "Designing and programming the emotion engine," EEE Micro 19:6 (November-December),
参考文献
535
20-28.
Okada, S., S. Okada,Y. Matsuda, T. Yarada, and A. Kobayashi [1999]. "System on a chip for digital still camera,” JEEE Trans.
on Consumer Electronics 45:3 (August), 584-590.
Oliker, L., A. Canning, J.Carter, J. Shal f, and S. Ethier [2004]. “'Scientific computations on moder parallel vector systems," Proc.
ACMIEEE Conf on.Supercomputing, November 6-12, 2004, Pittsburgh, Penn.,10.
Pabst, T. [2000]. “Performance Showdown at 133 MHz FSBThe Best Platform for Coppermine," www6.tomshar.vare.com/
mainboand/00g 1/000302..
Padua, D., and M. Wolfe [1986]. "Advanced compiler optimizations for supercomputers," Communications ofthe ACM 29:12
(December), 1184-1201.
Palacharla, S., and R. E. Kessler [1994]. "Evaluating stream buffers as a secondary cache replacement," Proc. 2/st Anmual Int'.
Smpasium on Computer Architecte (ISC.A), April 18-21, 1994, Chicago, 24-33.
Palmer, J., and S. Morse [1984]. Te 8087 Primer, John Wiley & Sons, New York, 93.
Pan, S-T., K.So, and J.T. Rameh [1992]. "Tnproving the accuracy of dynamic branch prediction using branch correlation,”" Proc.
Fifh In'1.Conf.on Architectual Support for Programming Languages and Operating Systems (ASPLOS), October 12-15,
1992,Boston, 76-84.
Partridge, C. [1994]. Gigabit Nervorking, Addison-Wesley, Reading, Mass.
Patterson, D.[1985]. *Reduced instruction set computers." Communications ofthe ACM 28:1 (Jamuary), 8-21.
Patterson, D. [2004]. “Latency lags bandwidth,"Communications of the ACM/ 47:10 (October), 71-75.
Patterson, D. A., and D. R. Ditzel [1980]. “The case for the reduced instruction set computer," Computer Architecture News 8:6
(October), 25-33.
Patterson, D. A., and J. L. Hennessy [2004]. Computer Organization and Design: The Hardware/Software Interface, 3rd ed.,
Morgan Kaufmann, San Francisco.
Patterson, D. A., G. A. Gibson, and R. H. Katz [1987]. A Case for Redundant Arrays of Inexpensive Disks (RAID), Tech. Rep.
UCB/CSD 87/391, University of California,Berkeley. Also appeared in Proc. ACM SIGMOD, June 1-3, 1988, Chicago,
109-116.
Patterson, D. A., P. Garrison, M. Hill, D. Lioupis, C. Nyberg, T. Sippel, and K. Van Dyke [1983]. “Architecture of a VLSI
instruction cache for a RISC," 10th Anmual Int'1. Conf. on Computer Architecture Conf. Proc., June 13-16, 1983, Stockholm,
Sweden, 108-116.
Pavan, P., R. Bez, P. Olivo, and E. Zanoni [1997].“Flash memory cells-—an overview." Proc. IEEE 85:8 (August), 1248-1271.
Peh, L. S., and W.J. Dally [2001]. “A delay model and speculative architecture for pipelined routers." Proc. 7th Int'7.Symposium
on High-Performance Compuer Architecture, January 22-24, 2001, Monterey, Mexico.
Peng, V., S. Samudrala, and M. Gavrielov [1987]. “On the implementation of shifters, multipliers, and dividers in VLSl floating
point units," Proc. 8th JEEE Sympasitm on Computer Arithmetic, May 19-21, 1987, Como, Italy, 95-102.
Phister, G. F.[1998]. In Search of Clusters, 2nd ed., Prentice Hall, Upper Saddle River, N.J.
Pfister, G. F., W. C. Brantley, D. A. George, S. L. Harvey, W.J. Kleinfekder, K. P. McAuliffe, E. A. Melton, V. A. Norton, andJ.
Weiss [1985]. "The IBM research parallel processor prototype (KP3): Introduction and architecture," Proc. 1zth Annal lnt'!.
Sywmmposium on Computer Architecture (ISCA), June 17-19, 1985, Boston, Mass., 764-771.
Pinheiro, E. W. D. Weber, and L. A. Barroso [2007]. *Failure trends in a large disk drive population” Proc. 5th USENLX
Conference on Frile and Storage Technologies (FA.ST '07), February 13-16, 2007, San Jose, Calif.
Pinkston, T. M.[2004]. “Deadlock characterization and resolution in interconnection networks,” in M. C. Zhu and M. P. Fanti,
eds., Deadlock Resolution in Computer- Integrated Systems, CRC Press, Boca Raton, FL, 445-492.
Pinkston, T. M.,and J. Shin [2005]. *Trends toward on-chip networked microsysters"" Int '1. J. ofHigh Performance Computing
and Networking 3:1,3-18.
Pinkston, T. M., and S. Warnakulasuriya [1997]. "On deadlocks in interconnection networks" 24th Annual Int'.Symposium on
Computer Architecture (ISCA), June 2-4, 1997, Denver, Colo.
Pinkston,T. M., A. Benner, M. Krause, l. Robinson, and T. Sterling [2003]. “In finiBand: The 'de facto' future standard for system
and local area networks or just a scalable replacement for PCI buses?”Cluster Computing (special issue on communication
architecture for clusters) 6:2 (April), 95-104.
Postiti, M. A., D. A. Greene, G. S. Tyson, and T. N. Mudge [1999]. *The limits of instruction level parallelism in SPEC95
536
参考文献
applications," Computer Architecture News 27:1 (Match),31-40.
Przybylski, S. A. [1990]. Cache Design: A Performance-Direcfed Approach, Morgan Kaufinanr, San Francisco.
Przybylski, S. A., M. Horowitz, and J. L. Hennessy [1988]. “Performance trade-ofis in cache design," ISth Annual Int '.
Symposium on Computer Architecture, May 30-June 2, 1988, Honolulu, Hawaii, 290-298.
Puente, V., R. Bcivide, J. A, Gregorio, J. M. Prellezo, J. Duato, and C. Izu [1999]."Adaptive bubble router: A design to improve
performance in torus networks.” Proc. 28th Int'1. Conference on Parallel Processing, September 21-24, 1999, Aizu-Wakamatsu,
Fukushira, Japan.
Radin, G.[1982]. "The 801 minicomputer"" Proc. Sympasium Architectural Support for Programming Languages and Operating
Systems (ASPLOS), March 1-3, 1982, Palo Alto, Calif., 39-47.
Rajesh Bordawekar, Uday Bondhugula, Ravi Rao: Believe it or not!: mult-core CPUs can match GPU performance for a
FLOP-intensive application! 19th International Conference on Parallel Architecture and Compilation Techniques (PACT
2010), Vienna, Austria, September 11-15, 2010:537-538.
33rd Anma/ Int '. Symposium on Computer Architecture (LSCA), June 17-21, 2006, Boston, Mass., 66-77.
Rau, B. R.[1994]. "Iterative modulo scheduling: An algorithr for software pipelining 1oops,”Proc. 27th Ammual bt !.
Symposium on Microarchitecture, November 30-December 2, 1994, San Jose, Calif.,, 63-74.
Rau, B. R., C. D. Glaeser, and R. L. Picard [1982]. "Efficient code generation for horizontal architectures: Compiler techniques
and architectural support." Proc. Ninth Annual Int '. Symposium on Computer Architectuire (ISCA), April 26-29, 1982, Austim,
Tex-,131-139.
Rau, B. R., D. W. L. Yen, W. Yen, and R. A. Towle [1989]. "The Cydra 5 depatnental supercomputer: Design philosophies,
decisions, and trade-offs”" IEEE Computers 22:1 (January), 12-34.
Reddi, V.J., B.C. Lee, T. Chilimbi, and K. Vaid [2010]. "Web search using mobile cores: Quantifying and mitigating the price of
efficiency" Proc. 37th Annzal Int '. Symposium on Computer Architecture (USCA), June 19-23, 2010, Saint-Malo, France.
Redmond, K. C., and T. M. Smith [1980]. Project Whirlwind-The History of a Pioneer Computer, Digital Press, Boston.
Rcinhardt, S. K., J. R. Larus, and D. A. Wood [1994]. *Tempest and Typhoon: User-level shared memory," 21st Annual It.
Simposium on Computer Architecture (USCA), April 18-21, 1994, Chicag0, 325-336.
Reinman, G., and N. P. Jouppi. [1999]. “Extensions to CACTI" research.compag.com/ wrlpeoplejjouppi/CACTI. html.
Rettberg, R. D., W.R. Crowther, P.P. Carvey, and R. S. Towlinson [1990]. "The Monarch parallel processor hardware design,”
IEEE Comptter 23:4 (April), 18-30.
Riemens, A.,K. A. Vissers, R. J. Schutten, F. W. Sijstermans, G. J. Hekstra, and G. D. La Hei [1999] Trimedia CPU64
application dorain and benchmark suite,”Proc. IEEE Int'1. Conf on Compuler Design: VLS/ in Computers and Processors
(CCD'99), October 10-13, 1999, Austin, Tex., 580-585.
iseman, E. M., and C. C. Foster [1972]. “Percolation of code to enhance paralled dispatching and execution,” IEEE Trans. on
Computers C-21:12 (December), 1411-1415.
sobin, J., and C. Irvine [2000]. “Analysis of the Intel Pentium's ability to support a secure virtual machine monitor.” Proc.
USENIX Security Symposium, August 14-17, 2000, Denver, Colo.
Robinson, B., and L. Blount [1986]. The YM/HPO 3880-23 Performance Resulks, IBM Tech. Bulletin GG66-0247-00, IBM
Washington Systemns Center, Gaithersburg, Md.
Ropers, A., H. W. Lollman, and J. Wellhausen '[1999]. DSPstone: Texas Instruments TMS320C54x, Tech. Rep. IB 315
1999/9-ISS-Version 0.9, Aachen University of Technology, Aaachen, Germany (www.ert.rwh-aachen.de/Projekte/Tools/
coal/dspstone_c54x/index.html).
Rosenblum, M., S. A. Herrod, E. Witchel, and A. Gupta [1995]. "Complete computer simulation: The SimOS approach," in JEEE
Parallel and Distributed Technology (now called Concurrency) 4:3, 34-43.
Rowen, C., M. Johnson, and P. Ries [1988]. “The MIPS R3010 foating-point coprocessor,” IEEE Micro 8:3 (June), $3-62.
Rusgell, R. M. [1978]. "The Cray-1 processor system.” Commuications ofthe ACM21:1 (January), 63-72.
Rymarczyk, J.[1982]. “Coding guideines for pipelined processors,” Proc. Symposium Architectural Support for Programming
Lamnguages and Operating Systems (A.SPL.OS), March 1-3, 1982, Palo Alto, Calif, 12-19.
Ssavedra-Barrera, k. H. [1992). “CPU Performance Evaluation and Execution Time Prediction Using Narrow Spectrum
参考文献
537
Benchmarking,” Ph,D.dissertation, University of California, Berkeley.
Salem, K., and H. Garcia-Molina [1986]. "Disk striping," Proc. 2nd Int 1. IEEE Conf. on Data Engineering, February 5-7, 1986,.
Washington, D.C., 249-259.
Saltzer,J. H., D. P.Reed, and D.D. Clark [1984]. *End-to-end arguments in systemu design,”ACM Trans.on Computer Systems
2:4 (November),277-288.
Samples, A. D., and P. N. Hilfnger [1988]. Code Reorganization for Instruction Caches, Tech. Rep. UCB/CSD 88/447,
University of California, Berkeley.
Santoro,M. R.,G. Bewick, and M. A. Horowitz [1989]. "Rounding algorithms for IEBEE multipliers," Proc. Ninth IEEE
Symposium on Computer Arithmenic, September 6- 8, Santa Monica, Calif.,, 176-183.
Satran, J., D.Smith, K. Meth, C. Sapuntzakis, M, Wakeley, P. Von Stamwitz, R. Haagens, E. Zeidner, L. Dalle Ore, and Y.Klein
[2001]."iSCSL.” IPs Working Group of ETF, Interet draft www.ietforg/internet-drafis/cdrafi-jetf-ips-15cs1-07.t8t.
Saulsbury, A., T. Wilkinson, J. Carter, and A. Landin [1995]. "An argument for Simple COMA," Proc. Firs: IEEE Symposium on
High-Performance Compuier Architectures, January 22-25, 199S, Raleigh, N.C., 276-285.
Schneck, P. B. [1987].Suyperprocessor Architecture, Kluwer Academic Publishers, Norwell, Mass.
Schroeder,B, and G. A. Gibson [2007]. “Understanding failures i petascale computers," J. of Physics Conf. Series 78(1),
188-198.
Schroeder, B., E. Pinheiro, and W.-D. Weber [2009]. "DRAM crrors in the wild: a largescale field study," Proc. Eleventh IntY.
Joint Conf. on Measurement and Modeling of Computer Systems (SIGMETRICS), June 15-19, 2009, Seattle, Wash.
Schuran, E., and J.Brutlag [2009]. "The user and business impact of server delays,” Proc. Velocity: Weh Performance and
Operations Corf, June 22-24, 2009, San Jose, Calif.
Schwartz, J. T. [1980]. “Ultracomputers," ACM Trans. on Programming Languages and Systems 4:2, 484-521.
Scott, N. R. [1985]. Computer Number Systems and Arithmetic, Prentice Hall, Englewood Cliffs, N.J.
Scott, S. L. [1996]. “Synchronization and communication in the T3E multiprocessor.” Seventh Int '.Conf.on Architectra!
Support for Programming Languages and Operating Systems(ASPLOS), October 1-5, 1996, Cambridge, Mass.
Scott, S. L., and J. Goodman [1994]. *The irpact of pipelined channels on k-ary n-cube networks ” IEEE Trans.on Parallel and
Distributed Systems 5:1 (January), 1-16.
Scott, S. L., and G. M. Thorson [1996]. “The Cray T3E network: Adaptive routing in a high performance 3D torus," Proc. IEEE
HOT Interconects '96, August 1$-17, 1996, Stanford University, Palo Alto, Calit., 14-156.
Seranton, R. A., D. A, Thompson, and D. W. Hfunter [1983]. The Access Time Mfth.” Tech. Rep. RC 10197 (45223), IBM,
Yorktown Heights, N.Y.
Seagate. [2000] Seagate Cheetah 73 Family: ST173404LW/ILWVLC/LCY Procct Manual, Vol. 1, Seagate, Scotts Valley, Calif.
(www.seagate.com/suppor/disc/mamuals/ scsi/2947Bb.pdh.
Seitz, C. L.[1985]. *The Costmnic Cube (concurrent computing)" Communications ofthe ACM 28:1 (January), 22-33.
Senior, J. M. [1993]. Optical Fiber Commmunications: Principles and Practice, 2nd ed., Prentice Hall, Hertfordshire, U.K.
Sharangpani, H,, and K. Arora [2000]."Itanium Processor Microarchitecture," IEEE Micro 20:5 (September-October), 24-43.
Shurkin, J. [1984]. Engines of the Mind: A History of the Computer, W. W. Norton, New York.
Shustek, L. J. [1978]. "Analysis and Perfonmance of Computer Instruction Sets;” Ph.D. dissertation, Stanford University, Falo
Alto, Calif.
Silicon Graphics. [1996]. MPS V Instruction Set (see http://www.sgi.com/MS/arch/ ISA5/#MIPSV._indx).
Singh, J. P., J. L. Hennessy, and A. Gupta [1993]. “Scaling parallel programs for multiprocessors: Methodology and examples,”
Computer 26:7 (July), 22-33.
Sinharoy, B., R. N. Koala, J. M, Tendler, R. J. Eickemeyer, and J.B. Joyner [2005]. "POWER5 system microarchitecture,” IBM.J.
Research and Development, 49:4-5, 505-521.
Sites, R. [1979]. Instruction Ordering Jor the CRAY-1 Computer, Tech. Rep. 78-CS-023, Dept. of Computer Science, University
ofCalifornia, San Diego.
Sites, R. L. (ed.) [1992]. Alpha Architecture Reference Mantal, Digital Press, Burlington, Mass.
Sites, R. L., and R. Witek, (eds.) [1995]. Alpha Architecture Refarence Mantal, 2nd ed., Digital Press, Newton, Mass.
High-Performance Computer Archilectue, February 1-5, 1997, San Antonio, Tex., 144-1S5.
538
参考文献
Skadron, K,, P. S. Ahuja, M. Martonosi, and D. W. Clark [1999]. “Branch prediction, instructior-window size, and cache size:
Performnance tradeoffs and sirulation techniques," IEEE Trans. on Computers 48:11 (November).
Slater, R. [1987]. Porraits in Silicon, MIT Press, Cambridge, Mass.
Slotnick, D.L.,W. C. Borck, and R. C. McReynolds [1962]. "The Solomon computer." Proc. AFIPS Fall.Joint Computer Conf,
December 4-6, 1962, Philadelphia, Penn., 97-107.
Smith, A. J.[1982]. “Cache memories." Computing Surveys 14:3 (September), 473-530.
Smith, A., and J. Lee [1984]. "Branch prediction strategies and branch-target buffer design,” Computer 17:1 (January), 6-22.
Smith, B.J. [1978]. “A pipelined, shared resource MIMID computer,” Proc. Int Y.Conf. on Parallel Processing (ICPP), August,
Bellaire, Mich., 6-8.
Smith, B. J.[1981]. “Architecture and applications of the HEP multiprocessor system” Real-Time Signal Processing IV 298
(August), 241-248.
Smith, J. E.[1981]. “A study of branch prediction stategies," Proc. Eighth Anmteal Int '.Siymposium on Computer Architecture
(ISCA), May 12-14, 198l, Minneapolis, Minn., 135-148.
Smith, J. E. [1984]. “Decoupled access/execute computer architectures,” ACM Trans. on Computer Systems 2:4 (November),
289-308.
Smith, J. E.[1988]. "Characterizing computer performance with a single number.”Communications ofthe ACM 31:10 (October),
1202-1206.
Smith, J.E.[1989]."Dynamic instruction scheduling and the Astronautics ZS-1," Computer 22:7 (July), 21-35.
Smnith, J.E.,and J. R.Goodman [1983]."A study of instruction cache organizations and replacemnent policies," Proc. 10th Amual
Int'. Synposium on Computer Architecture (ISC.A), June S-7, 1982, Stockholm, Sweden, 132-137.
Smith, J. E., and A. R. Pleszkun [1988]. "Implementing precise interrupts in pipelined processors” IEEE Trans. on Computers
37:5 (May), 562-573. (This paper is based on an earlier paper that appeared in Proc. 12th Annual Int'.Symposium on
Computer Architecture (ISCA), June 17-19, 1985, Boston, Mass.)
Smith, J. E., G. E. Dermer, B. D. Vanderwamn, S. D. Klinger, C. M. Rozewski, D. L. Fowler, K. R. Scidmore, and J. P.Laudon
[1987]. “The ZS-1 central processor,” Proc. Second Int '.Conf. on Architectal Support for Programming Languages and
Operating Systems (ASPLOS), October 5-8, 1987, Palo Alto, Calif., 199-204.
Smith, M. D.M. Horowitz, and M.S.Lam [1992]. “Efhicient superscalar perforance through boosting." Proc. Fifih Int'. Conf.
on Architectural Support for Programming Languages and Operating Systems (ASPLOS), October 12-15, 1992, Boston,
248-259.
Smith,M. D,M. Johnson, and M. A. Horowitz [1989]. "'Limits on multiple instruction issue,” Proc. Third Int'. Conf.on
Architectural Support for Programming Languages and Operating Systems (ASPLO.S), April 3-6, 1989, Boston, 290-302.
Smotheran, M. [1989]. “A sequencing-based taxonomy of 1/O systems and review of historical machines,” Computer
Architecture Nevs 17:5 (Septenber), $-15. Reprinted in Computer Architecture Readings, M. D. Fill, N. P. Jouppi, and G. s.
Sohi, eds., Morgan Kaufmann, San Francisco, 1999, 451-461.
Sodani, A., and G. Sohi [1997]. "Dynamic instruction reuse,” Proc. 24th Annual Int'V. Symposium on Computer Architecture
(ISCA), June 2-4, 1997, Denver, Colo.
Sohi,G. S. [1990]. “Tnstruction issue logic for high-performance, internuptible, multiple functional unit, pipelined computers”
IEEE Trans.on Computers 39:3 (March), 349-359.
Sohi, G.S., and S. Vajapeyam [1989]. "Tradeofis in instruction format design for horizontal architectures,” Proc. Third Int'. Conf.
on.Architectural Support for Programming Languages and Operating Systems (ASPLOS), April 3-6, 1.989, Boston, 15-25.
Soundararajan, V., M. Heinrich, B. Verghese, K. Gharachorloo, A. Gupta, and J. L. Hennessy [1998]. "Flexible use of memory
for replication/migration in cachecoherent DSM multiprocessors,”Proc. 25th Annual Int '.Symposium on Computer
Archirechure (ISCA), July 3-14, 1998, Barcelona, Spain, 342-35S.
SPEC. [1989]. SPEC Benchmark Suite Release 1.0 (October 2).
SPEC.[1994J.SPEC Newsletter (June).
Sporer, M., F. H. Moss, and C.J. Mathais [1988]. “An introduction to the architecture of the Stellar Graphics supercomputer,”
Proc. IEEE COMPCON, February 29-March 4, 1988, San Francisco, 464.
Spurgeon, C. [2001]. “Charles Spurgson's Etheruet Web Site. " wwwhost.ots.utexas.edhd/ ethere./etheret-home.html.
Spurgeon, C. [2006]. *Charies Spurgeon's Ethermet Web SITE,” www.ethermam.ge.com/ ethernet/ethernet.htmk.
参考文献
539
Stenstrom, P., T. Joe, and A. Gupta [1992]. “Comparative performance evaluation of cache-coherent NUMA and COMA
architectures,” Proc. I9th Anmual Int '. Sumposium on Compurer Architfecture (ISC.A), Miay 19-21, 1992, Gold Coast,
Australia,80-91.
Sterling, T. [2001].Beowwlf PC Cluster Computing with Windors and Beowalf PC Cluster Computing with Limcx, MIT Press,
Stern, N. [1980]. "Who invented the first electronic digital computer?" Annais of the History of Comguting 2:4 (October),
375-376.
Stevens, W. R. [1994-1996]. TCPIP Illustrated (three volumes), Addison-Wesley, Reading, Mass.
Stokes, J. [2000]. "Sound and Vision: A Tecbnical Overview of the Emotion Engine," arstechnica.com./reviews/1q00/playstationz/
ee-1.html.
Stone, H. [1991]. High Performance Computers, Addison-Wesley, New York.
Strauss, W.[1998]. "psP Strategies 2002, www.usadata.com/ market_ research/spr_ 05/spr_r127-005.htm.
Strecker, W. D. [1976]. "Cache memories for the PDP-117," Proc. Third Annual Int '. Sympasium on Computer Architecture
(ISCA), January 19-21, 1976, Tampa, Fla., 155-158.
Strecker, W.D.[1978].“VAX-11/780: A virtual address extension ofthe PDP-11 family,” Proc. AFIPS National Computer Conf,
June 5-8, 1978, Anaheim, Calif., 47, 967-980.
Sugurar, R. A., and S.G.Abraham [1993]. Efficient simulation of caches under optimal replacement with applications to mniss
characterization" Proc. ACM SIGMETRICS Conf:on Measurement and Modeling of Computer Systems, May 17-21, 1993,
Santa Clara, Calif., 24-35.
Su Microsysterns. [1989]. The SPARC Architectura! Mamual, Version 8, Part No. 8001399-09, Sum Microsystems, Santa Clara, Calif.
Sussenguth, E. [1999]. "TBM's ACS-1 Machine," IEEE Computer 22:11 (November).
Swan, R. J.,S. H. Fuller, and D. P. Siewiorek [1977]. "Cm*—a mnodular, multimicroprocessor,” Proc. AFIPS National
Computing Conf., June 13-16, 1977, Dallas, Tex., 637-644.
Swan, R. J., A. Bechtolsheim, K. W. Lai, and J. K. Ousterhout [1977]. “The implementation of the Cm* multi-microprocessor,”
Proc AFIPS National Compuing Conf, June 13-16, 1977, Dallas, Tex., 645-654.
Swartzlander, E. (ed.) [1990]. Computer Arithmeric, JEEE Computer Society Press, Los Alamitos, Calif.
Takagi, N., H. Yasuura,and S. Yajlma [1985]."Figh-speed VI.Sl multiplication algoritbm with a redundant binary addition tree.”
IEEE Trans. on Computers C-34:9, 789-796.
Talagala, N. [2000]. “Characterizing Large Storage Systems: Euror Behavior and Performance Benchmarks,”Ph.D. dissertation,
Computer Science Division, University of California, Berkeley.
Talagala, N., and D. Patterson [1999]. An Analysis of Error Behavior in a Large Slorage System, Tech. Report UCB//CSD-99-1042,
Computer Science Division, University of California, Berkeley.
Talagala, N., R. Arpaci-Dusseau, and D. Patterson [2000].Micro-Benchmark Based Exraction of Local and Global Disk
Characteristics, CSD-99-1063, Computer Science Division, University of Califoria, Berkeley.
Talagala, N., S. Asami, D. Patterson, R. Futernick, and D. Hart [2000]. "The art of massive storage: A case study of a Web image
archive,” Computer (November).
Tamir, Y., and G. Frazier [1992]. "Dynamically-allocated mnwiti-queue buffers for VLSl communication switches,"IEEE Trans.
on Computers 41:6 (June), 725-734.
Tanenbaum, A. S. [1978]. “Tmplications of structured programming for machine architechure," Communications ofthe ACM21:3
(March),237-246.
Tanenbaum, A. S. [1988]. Computer Networks, 2nd ed., Prentice Hall, Englewood Clitis, N.J.
Tang, C. K. [1976]. "Cache design in the tightly coupled multiprocessor system," Proc. AFIPS National Computer Conf., June
7-10,1976, New York, 749-753.
Tangueray, D. [2002]. "The Cray X1 and supercomputer road map." Proc. 13th Daresbury Machine Evaluation Workshop,
December 11-12,2002, Daresbury Laboratories, Daresbury, Cheshire, U.K.
Tarjan, D., S. Thoziyoor,and N. Jouppi [2005]. “HPL Technical Report on CACTI 4.0,”www.hpl.hp.com/techeports/
2006/HPL=2006+86.html.
Taylor, G. S. [1981]. “Compatible hardware for division and square root," Proc. 5th IEEE Symposiu on Compuler Arithmetic,
May 18-19, 1981, University of Michigan, Ann Arbor, Mich., 127-134.
、
540
参考文献
Taylor, G.S. [1985]. “Radix 16 SRT dividers with overlapped quotient selection stages,"Proc. Seventh JEEE Symposium on
Compuler Arithmetie, June 4-6, 1985, University of Illinois, Urbana, Iill., 64-71.
Taylor, G., P. Hilfinger, J. Larus, D. Patterson, and B. Zor [1986]. “Evaluation of the SPUR L.ISP architecture.” Proc. 13th
Annual Int'.Symposium on Computer Architecsure (ISCA), June 2-5, 1986, Tokyo.
Taylor, M. B., W. Lee, S. P. Amarasinghe, and A. Agarwal [2005]. "Scalar operand networks." EEE Trans.on Parallel and
Distributed Systems 16:2 (February), 145-162.
Tendler, J. M., J. S. Dodson, J.S. Fields, Jr. H. Le, and B.Sinharoy [2002].“Power4 system microarchitecture”1BM.J. Research
and Development 46:1,5-26.
Texas Instruments. [2000]. "'History of Innovation: 1980s," www.ti.com/corp/docs/ company/ history/1980s.shtml.
Tezzaron Semniconductor. [2004]. Soft Errors in Electronic Memor, White Paper, Tezzaron Semiconductor, Naperville, Ill.
(http://www.tezzaron.com/aboutpapers/soft_errors_/_1_secure.pdjf).
Thacker, C. P., E. M. McCreight, B. W. Lampson, R. F. Sproull, and D. R. Boggs [1982].
“Alto: A personal computer” in D. P. Siewiorek, C. G. Bell, and A.Newell, eds., Computer Structures: Principles dnd Examples,
McGraw-Hi,New York, 549-572.
Thachani, A. J. [1981]. *Tnteractive user productivity." IBM/ Systems J. 20:4, 407-423.
Thekkath, R., A. P. Singh, J. P. Singh,S. Jobn, and J. L. Hennessy [1997]. “An evaluation of a commercial CC-NUMA
architecture- the CONVEX Exemplar SPP1200,Proc. 11th Int'. Parallel Processing Symmposium (LPPS), April 1-7, 1997,
Gencva, Switzerland.
Thorlin, J.F.[1967]. "Code generation for PIE (parallel instruction execution) computers”Proc. Spring Joint Compuer Conf.,
April 18-20, 1967, Atlantic City, N.J., 27.
Thornton, J. E. [1964]. "Parallel operation in the Control Data 6600," Proc. AFIPS Fall Joint Computer Conf., Part Il, Outober
27-29, 1964, San Francisco, 26,33-40.
Thornton, J. E. [1970]. Design of a Computer, the Control Data 6600, Scott, Foresmzan, Glenview, Ill.
Tjaden, G. S., and M.J. Flyan [1970]. “Tetection and parallel execution of independent instructions " E&E Tans.on Computers
C-19:10 (October), 889-895.
Tomasulo, R. M. [1967]. “An efficient algorithm for exploiting multiple arithmetic units,” BM J. Research and Development
11:1 (January), 25-33.
Torrellas, J., A.Gupta,and J. Hennessy [1992]. "Characterizing the cachingand synchronization perforance of a multiprocessor
operating system," Proc. Fifih Int'1. Conf. on Architectural Support for Programming Languages and Operating Syustems
(ASPLOS), October 12-15, 1992, Boston (SIGPLAIN Notices 27:9 (September), 162-174).
Touma, W. R. [1993]. The Dynamics of the Computer Industry: Modeling the Supply of Workstations and Their Componenks,
Kluwer Acadernic, Boston.
Tuck, N., and D. Tullsen [2003]. "Initial observations of the simultaneous multithreading Pentium 4 processor,” Proc. 12th Int.
Conf. on Parallel Architectures and Compilation Techniques (PACT'03), September 27-Octdber 1, 2003, New Orleans, La,
26-34.
Tullsen, D. M., S.J. Eggers, and H.M.Levy [1995]. “Simultaneous multithreading: Maximizingon-chip parallelism" Proc. 22nd
Annal Int '1. Symposium on Computer Architechure (ISC.A), June 22-24, 1995, Santa Margberita, Italy, 392-403.
Tulsen, D. M., S. J. Eggers, J. S. Fmer, H. M. Levy, J.L. Lo, and R. L. Stamm [1996]. “Exploiting choice: Instruction fetch and
issue on an irplementable simultaneous mmultithreading processor,”Proc. 23rd AnmualInt '. Symposium on Compurer
Archilecture (ISCA), May 22-24, 1996, Philadelphia, Penn., 191-202.
Ungar, D., R. Blau, P. Foley, D. Samples, and D. Patterson [1984]. “Architecture of SOAR: Smalltalk on a RISC,"Proc. 11th
Annual Int 'L. Smmposium on Compuer Architectre (ISCA), June 5-7, 1984, Ann Arbor, Mich., 188-197.
Unger, S. H. [1958]. “A computer oriented towards spatial problems," Proc. Institute of Radio Engineers 46:10 (October ),
1744-1750.
Vahdat, A., M. Al-Fares, N. Farrington, R. Niranjan Mysore, G. Porter, and S. Radhakrishtan [2010]. “Scale-Out Networking in
the Data Center”IEEE Micro 30:4 (July/August), 29-41.
Vaidya, A. S., A Sivasubramaniamn, and C. R. Das [1997]. "Performance benefits of virtual channels and adaptive routing: An
application-driven study," Proc.ACM/EEE Conf. on Supercomputing, November 16-21, 1997, San Jose, Calif.
Vajapeyam, S. [1991]. *Lnstruction-Level Characterization of the Cray Y-MP Processor,” Ph.D. thesis, Computer Sciences
参考文献
541
Department, University of Wisconsin-Madison. van Eijndhoven, J.T.J., F. W. Sijisterans, K. A. Vissers, E. J. D. Pol, M. I. A.
Tromp, P. Struik, R. H. J. Bloks, P. van der Wolt, A. D. Pimentel, and H. P. E. Vranken [1999], "Trimedia CPU64
architecture," Proc: IEEE Int'L.Conf on Computer Design: VSI in Compuers and Processors (ICCD'99), October 10-13,
1999,Austin, Tex., 586-592.
Van Vleck, T. [2005]. "The IBM 360/67 and CP/CMS," http: //www.multicians.org/thvw/ 360-67.html.
von Bicken, T., D. E. Culler, S. C. Goldstein, and K. B. Schauser [1992]. “Active Messages: A mechanism for integrated
Gold Coast, Australia.
Waingold, E., M. Taylor, D.Srikrishna, V. Satkar, W. Lee, V. Lee, J. Kim, M. Frank, R.
Finch, R.Barua, J.Babb, S. Amarasinghe, and A. Agarwal [1997]. "Baring it all to software: Raw Machines," IEEE Computer 30
(September),86-93.
Wakerly, J.[1989]. Microcomputer Archiecne and Programming, Wiley, New York.
Wall,D. W.[1991]. “Limnits of instruction-levelparallelism,”Proc. Fourth Int 'l. Conf.on Architeckural Support for
Programming Languages and Operating Stems (ASPLOS), April 8-11, 1991, Palo Alto, Calif, 248-259.
Wall, D. W. [1993]. Limits of Instruction-Leve/ Parallelism, Research Rep. 93/6, Western Research L.aboratory, Digital
Equipment Corp., Palo Alto, Calif.
Walrand, J. [1991]. Communication Networks: A First Course, Aksen Associates/Irwin, Homewood, Ill.
Wang, W..H., J.-L. Baer, and H. M. Levy [1989]. "Organization and performance of a two-level virtual-real cache hierarchy,"
Proc. 16th Annual Int Y. Symposium on Computer Architecture (ISC.A), May 28-June 1, 1989, Jerusalem, 140-148.
Watanabe, T. [1987]. “Architecture and perfornance of the NEC supercomputer SX system,”" Parallel Compuing 5, 247-255.
Waters, F. (ed.) [1986]. IBM RT Personal Computer Technology, SA 23-1057, IBM, Austin, Tex.
Watson, W. J. [1972]. “The Ti ASC—a highly modular and flexible super processor architecture,” Proc. AFIPS Fall Joint
Computer Conf., December 5-7, 1972, Anaheim, Calif., 221-228.
Weaver, D. L., and T. Germond [1994]. The SP-ARC Architectural Mamual, Version 9, Prentice Hal, Englewood Clifis, N.J.
Weicker, R. P.(1984]. “TDhrystone: A synthetic systers programming benchmark."Communications ofthe ACM27:10 (October),
1013-1030.
Weiss, S., and J. E. Smith [1984]. "Instruction issue logic for pipclined supercomputers," Proc. 1Ith Anmual Int '.Symposizim on
Computer Architectue (ISCA), June 5-7, 1984, Amn Arbor, Mich., 110-118.
Weiss, S., and J.E. Smnith [1987]. "A study of scalar compilation techniques for pipelined supercorputers." Proc. Second IntY.
Conf. on Architectural Support for Programming Languages and Operaring Systems (ASPLOS), October 5-8, 1987, Palo Alto,
Calif.,105-109.
Weiss, S., and J. E. Smith [1994]. Power and PowerPC, Morgan Kaufinann, San Francisco.
Wendel, D., R. Kalla, J. Friedrich, J. Kahle, J. Leenstra, C. Lichtenau, B. Sinharoy, W. Starke, and V. Zyuban [2010]. “The
Power7 processor SoC,” Proc. Int '.Conf on IC Design and Technology, June 2-4, 2010, Grenoble, France, 71-73.
Weste, N., and K. Eshragbian [1993]. Principles of CMOS VL.SI Design: A Systems Perspective, 2nd ed., Addison-Wesley,
Reading, Mass.
Wiecek, C. [1982]. “A case shudy of the VAX 11 instruction set usage for compiler execution," Proc.Synposium on Architectual
Swupport for Programming Languoages and Operating Systems (ASPLOS), March 1-3, 1982, Palo Alto, Calif., 177-184.
Wilkes, M. [1965]. "Slave memnories and dyaamic storage allocation." IFEE Trans. Electr onic Computers EC-14:2 (April),
270-271.
Wilkes, M. V.[1982].“ardware support for memory Protection: Capability implementations,” Proc. Symposium on
Archilectural Support for Programming Languages and Operating Systems (A.SPLOS), March 1-3, 1982, Palo Alto, Calif.,
Wilkes, M. V. [1985]. Memairs of a Computer Pioneer, MIT Press, Cambridge, Mass.
Wilkes, M. V. [1995]. Computing Perspectives, Morgan Kaufmann, San Francisco.
Wilkes, M. V., D.J. Wheeler, and S. Gill [1951]. The Preparation of Programs for an Electronic Digital Computer,
Addison-Wesley, Cambridge, Mass.
Williams,s, A. Waterman, and D.Patterson [2009].“Roofine: An insightful visual performance model for multicore
architectures," Communication of the ACM, 52:4 (April), 65-76.
542
参考文献
Williams,T. E., M.Horowitz, R. L. Alverson, and T.S. Yang [1987]."A self-timed chip for division." in P.Losleben, ed., 1987
Stanford Conference or Advanced Research in VLSI, MIT Press, Cambridge,Mass.
Wilson, A. W., Jr.[1987]. "Fierarchical cache/bus architecture for shared-memory muitiprocessors,” Proc. 14th Anmual Int'!.
Symposium on Computer Architecture (LSCA), June 2-5, 1987, Pittsburgh, Penm., 244-252.
Wilson, R. P., and M. S. Lam [1995]. “Efficient context-sensitive pointer analysis for C programs," Proc. ACM SIGPLAN'95
Conf. on Programming Language Design and Implementation, June 18-21, 1995, La Jolla, Calif., 1-12.
Wolfe, A, and J. P. Shen [1991]. “A variable instruction stream extersion to the VLIW arcbitecture,” Proc. Fourth Inl'.Conf. om
Archilectural Support for Programming Languages and Operating Systems (ASPLOS), April 8-11, 1991, Palo Alto, Calif., 2-14.
Wood, D. A., and M. D. Hill [1995]. “Cost-effective parallel computing"JEEE Computer 28:2 (February), 69-72.
Wulf, W.[1981]. "Compilers and computer anchitecture,” Computer 14:7 (July), 41-47.
Wulf, W., and C. G. Bell [1972]. "C.mmp-A multi-mini-processor,” Proc. AFIPS Fall Joint Computer Conf, Decemiber S-7,
1972, Anaheim, Calif.,76$-777.
Wulf, W., and S. P. Harbison [1978]. "Reflections in a pool of processorsan experieace report on C.mmp/Hydra, ” Proc. AFIPS
National Computing Conf June S-8, 1978, Anaheim, Calif., 939-951.
Wulf,W.A., and S. A. McKee [1995]. Viting the memory wall: Inoplications of the obvious,” ACM SIGARCH Computer
Architecture Nes, 23:1 (March),20-24.
Wulf,W. A., R. Levin, and S. P. Harbison [1981]. KyaraC.mmp: An Experimental Compurer System, McGiraw-Hill, New York.
Yamamoto, W.,, M. J. Serano, A. R. Talcott, R. C. Wood, and M. Nemirosky [1994]. "Terformance estimation of mnultistreamed,
superscalar processors," Proc. 27th Annual Hawaii Int'. Conf. on System Sciences, January 4-7, 1994, Maui, 195-204.
Yang, Y., and G. Mason [1991]. "Nonblocking broadcast switching networks".IEEE Trans. on Computers 40:9 (September),
1005--1015.
Yeager, K. [1996]. “The MIPS R10000 superscalar microprocessor” IEEE Micro 16:2 (April), 28-40.
Yeh, T., and Y. N. Patt [1993a]. "Alternative implementations of two-level adaptive branch prediction," Proc. 19th Ammual Int '.
Symposium on Computer Archilecture (ISCA), May 19-21, 1992, Gold Coast, Australia, 124-134.
Yeh, T., and Y. N. Patt [1993b]. "A comparison of dynamic branch predictors that use two levels of branch history,” Proc. 20th
Annual Int '. Symposium on Compuler Architecture (ISCA), May 16-19, 1993, San Diego, Calif., 257-266.